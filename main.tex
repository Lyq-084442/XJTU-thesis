% !TEX program = latexmk
% !TEX encoding = UTF-8 Unicode

% 注意:请使用 TeXLive 2023 及以上版本
% WARNING: Please use TeXLive version >= 2023

% 注意:请确保自己已经完整阅读了 README.md 这一 markdown 文档,部分功能的使用方法并未在 .tex 中直接给出;如果仍有使用问题,请在 Github 上提出 issue
% WARNING: Pleans read README.md first cause some usages are not given in .tex files.

%%%%%%%%%%%%%%%%%%%%%%%%%%%%%%%%%%%%%%%%%%%%%%%%%%%%%%%%%%%%%%%%%%%%%%%%%%%%%%%%%%%%%%%%%%%%%%%%%%%%%
%% select the basic style of this thesis/dissertation (this document will use thesis for convenience)
%% 选择论文的基本类型

\documentclass[
    master,     % 必选项:   {master, doctor} 此处不区分专业/学术学位,在下面学位类型处区分
                % Mandatory: {master, doctor} No difference between Academic degree and Professional degree,
                %                             but be careful about arguments in `\degree' and `\subject'
    % english,  % 可选项:   英文正文请选择此项
                % Optional:  For english main content, It will change some auto-generated matter into English
    blind,    % 可选项:   论文用于盲审请选择此项,将隐去致谢页内容(但保留标题)、答辩委员会会议决议页内容(同前)、
                %            常规评阅人名单页内容(同前),将隐去题名页、答辩委员会页所有人名,
                %            注意:如果非自动读取成果数据库,请自行修改 `攻读学位期间取得的研究成果' 内的内容格式
                % Optional:  For blind review, It will not generate Acknowledgements(Content), Decision of Defense
                %            Committee(Content), General Reviewers List(Content), hide all names.
                %            NOTICE: CHANGE THE STYLE of Achievements BY YOURSELF
    % plgck,    % 可选项:   论文用于查重请选择此项,将只产生从摘要到附录(含)的内容,且所有图片均不显示
                % Optional:  For plagiarism check, It will only produce content from abstract to appendicies, 
                %            meanwhile all figures will not be displayed.
]{XJTU-thesis}



%%%%%%%%%%%%%%%%%%%%%%%%%%%%%%%%%%%%%%%%%%%%%%%%%
%% fill the each blank for auto-generate contents
%% 填写以下信息用以自动生成

% 论文标题(不超过35个字,英文注意大小写规律)
% Title, Make sure you have an acceptable capitalization
\title{无损网络净空缓存管理策略研究}{Research on Headroom Management Strategy in Lossless Network}

% 学位类型
% 考虑到专业学位的学位名称的少数特例(如专业型法学硕士不是 Master of Juris 而是 Juris Master),此处学位类型请按照案例和文件填写
% 同时为了前向兼容,可选项「A/P」(实际必须正确选择)代表「学术/专业」型学位,默认为学术型
% Type of your degree, Translate it from documents in `Materials/Requirements/2021/01 中英文题名页示例/英文标准翻译/'
\degree[A]{硕士}{Master of Engineering} % 学术型(Academic)硕士请基于 '学术学位名称.txt' 填写
% \degree[P]{工程硕士}{Master of Engineering} % 专业型(Professional)硕士请基于 '专业学位(领域)英文标准翻译.pdf' 填写
%\degree[A]{博士}{Doctor of Philosophy}  % 学术型(Academic Doc)博士请填写 '{博士}{Philosophy}'
% \degree[P]{工程博士}{Doctor of Engineering} % 专业型(Professional Doc)博士请基于 '专业学位(领域)英文标准翻译.pdf' 填写

% 作者姓名(注意:所有人名英文均为「名在前,姓在后」,如果只有外文名,请在两个参数都填写外文名称)
% If you have only foreign name, put it as both first and second argument
\author{刘钰琪}{Yuqi Liu}

% 指导老师姓名(注意:同作者姓名)
% Name of supervisor, It have the same requirments as the author
\advisor{单丹枫}{副教授}{Danfeng Shan}{Asscociate Prof.}

% 合作指导老师姓名 或 老师团队(合作指导老师指:1.与招生简章中一致的合作导师,2.CSC项目的合作导师)
%(校方模板要求只能选择一个,都有则显示合作导师)
% Name of associate advisor or adviror's team, You can use only one of them, and advisorassociate has a higher priority.
% \advisorassociate{陈尘}{副教授}{Chen Chen}{Asscociate Prof.}
% \advisorteam{团队中文名}{English Name of the Team}

% 学科名称,请基于 '学科(专业)英文标准翻译.pdf' 填写
% Name of the subject, also get it from that file
% \subject{航空宇航科学与技术}{Aeronautical and Astronautical Science and Technology}
\subject{计算机科学与技术}{Computer Science and Technology}

% 论文提交日期,不输入参数则默认使用当前日期,如手动指明年月,请在第一个可选参数内填写年份,第二个可选参数填写月份(均为阿拉伯数字)
% Submission date of this thesis, if you not put it manually, it will use the current time
% \submitdate[2021][06]
\submitdate

% 答辩委员会委员 显示的顺序和这里的一样 第一个人是主席
% Committee member of your defence, notice that the order shows in the thesis is same as here, and the first one is the chairperson
% each member should be put as {Organization,Name,Title} split by comma
\addcommitteemember{西安交通大学,张长长,教授}
\addcommitteemember{西安理工大学,王旺旺,教授}
\addcommitteemember{国网陕西经济技术研究院,李力,高工}
\addcommitteemember{西安交通大学,东方不败,副教授}
\addcommitteemember{西安交通大学,赵照,研究员}

% 答辩时间(手动指定)
% Defence date, input manually
\defensedate{2024}{04}{20}

% 答辩地点(涉密论文请手动设置为「西安交通大学」)
% Defence location, default value is 「西安交通大学」
\defenseloc{}

% 常规审阅人 要求和答辩委员会委员一样
% General reviewer list, same requirements as the committee member
\addgeneralreviewer{西安交通大学,张长长,教授}
\addgeneralreviewer{西安理工大学,王旺旺,教授}
\addgeneralreviewer{国网陕西经济技术研究院,李力,高工}
\addgeneralreviewer{西安交通大学,东方不败,副教授}
\addgeneralreviewer{西安交通大学,赵照,研究员}

% 学院
% School or Faculty, unused now
% \school{电气工程学院}{School of Electrical Engineering}

% 专业[学士学位使用]
% Major, unused now
% \stumajor{计算机科学与计数}{Computer Science}

% 学号[学士学位使用]
% Student ID, unused now
% \stuid{}

% 班级[学士学位使用]
% Administrative class, unused now
% \adminclass{电气7xx班}{}

% 参考文献源 参数中不要添加 .bib
% 请使用 \addreferenceresource 添加数据库(可导入多个参考文献数据库)
% 若自动化导入攻读学位期间的成果则使用 \addachivementresource
\addreferenceresource{References/reference}
\addachivementresource{References/achievement}

%%%%%%%%%%%%%%%%%%%%%%%%%%%%%%%%%%%%%%%%%%%%
%% 如果有使用其他包,请在这里添加
%% If you need other packages, use them here

% \usepackage{}


%%%%%%%%%%%%%%%%%%%%%%%%%%%%%%%%%%%%%%%%%%%%%%%%%%%%%%%%%%%%%%%%%%%%%%%%%%%%%%%%%%%
%% 注意:根据校方要求,以下所有页面顺序不可调整
%% Notice: The order of these pages are defined in the requirements of the University.

%% 但在最终提交前,可以通过注释所有 \thesis 开始的命令设置是否生成各个部分
%% 或根据说明调整 latexmkrc 文件,使用 \includeonly 命令导入部分章节。

% \includeonly{
%   Main_Spine/c1,
%   Main_Spine/c2,
%   Main_Spine/c3,
%   Main_Spine/c4,
%   Main_Spine/c5,
%   Main_Spine/c6,
% }

\begin{document}
% [自动生成] 中英题名页
% [Auto Generate] Chinese English Title Page
\thesistitles

% [自动生成] 答辩委员会页
% [Auto Generate] Defense Committee Pages
\thesiscommittes

% 生成摘要页 修改 Main_Miscellaneous/abstract_chs/eng.tex 中的内容
% Abstract, Rewrite your content in Main_Miscellaneous/abstract_chs/eng.tex
\thesisabstract

% [自动生成] 中英目录页
% [Auto Generate] Table of Contents
\thesistableofcontens

% 主要符号表 修改 Main_Miscellaneous/glossary.tex 中的内容
% Glossaries Page, Rewrite your content in Main_Miscellaneous/glossary.tex
% \thesisglossarylist

% 正文 使用 \thesisbodybegin & \include & \thesisbodyend 组合导入正文
\thesisbodybegin
% !TeX root = ../main.tex

\xchapter{绪论}{Introductions}

\xsection{研究背景与意义}{Background and Significance}

% 近年来,在数据中心和集群计算系统中,无损网络已经成为一种流行的趋势。一般来说,数据包丢失引起的重传很容易导致吞吐量减少、完成时间增加,甚至错过应用程序截止日期。

高带宽和低时延是高速网络的两个关键需求。为了满足这些需求,数据中心网络广泛部署远程直接内存访问(Remote Direct Memory Access, RDMA)技术\cite{SIGCOMM16RDMA,JYRJ202103005}。RDMA通过将网络协议栈卸载到硬件,可以在接近于零的CPU开销下实现高带宽和超低时延网络传输。在众多支持RDMA的协议中,包括InfiniBand、iWRAP等,RoCE/RoCEv2\cite{ibaRocev2}(RDMA over commodity Ethernet v2)因其低开销和低复杂度被广泛采用。RDMA网络中的丢包重传会严重损害传输性能,导致吞吐量降低、完成时间增加,甚至错过应用截止时间。因此,RDMA部署需要无损网络支持\cite{SIGCOMM15DCQCN,SIGCOMM15TIMELY,SIGCOMM16RDMA,SIGCOMM19HPCC,SIGCOMM20MasQ,liu2020datacenter}。

无损网络即不因设备缓存溢出而丢包。在RoCE/RoCEv2中,通常通过逐跳基于优先级的流量控制\cite{PFC}(Priority-based Flow Control,PFC)实现无损传输。支持PFC的转发设备在缓存占用超过一定阈值时向其上游设备发送一个暂停帧(PAUSE frame),上游设备收到该暂停帧后会暂停数据包的发送以此避免丢包。由于暂停帧从下游设备触发到被上游设备接收到并发挥作用需要一定的时延,所以转发设备需要预留部分缓存空间存储该时延内接收到的报文\cite{SIGCOMM15DCQCN,SIGCOMM16RDMA},这部分缓存空间称为净空缓存(Headroom)。

PFC本身可能对传输性能造成损害。在大规模网络中,PFC暂停帧会导致严重的性能问题,如头阻(head-of-line blocking)、拥塞传播、附带损害和死锁等问题\cite{SIGCOMM16RDMA,SIGCOMM19HPCC,INFOCOM14TCP-Bolt,CoNEXT17Tagger,SIGCOMM19GFC,TPDS20P-PFC,INFOCOM22ITSY, hu2022load}。因此,无损网络中的一个普遍共识为尽可能避免PFC触发,理想情况下PFC应该仅作为保证无损传输的保底机制。

在保证无丢包的前提下,避免PFC触发的关键在于一个高效的净空缓存管理机制。然而,随着高速网络链路带宽和分布式应用网络服务需求不断增加,现有净空缓存管理机制的低效性和性能问题日益突出。因此,无损网络中净空缓存管理机制正面临严峻挑战,需要进一步改善。

% 无法适应数据中心网络的发展趋势和提高网络性能的需求。

一方面,在数据中心网络内部,链路带宽已经从1Gbps快速增长到40Gbps再到100Gbps,目前仍在持续增长中,即将达到400Gbps\cite{SIGCOMM16RDMA,SIGCOMM15Jupiter}。经过调研发现,在当前的发展趋势下,数据中心网络中PFC的触发频率不断增加。具体地,交换芯片缓存中的净空缓存预留需求量和链路带宽呈正相关关系,随着数据中心网络链路带宽增加,需要预留的净空缓存总量也随之同比例增加。然而,为了高速、低时延的内存访问,现代数据中心交换机中普遍采用片上缓存。受限于芯片面积和成本,交换芯片上的缓存容量并不能以链路带宽增长速度同等程度增加\cite{INFOCOM20BCC,ICNP21FlashPass,BS19BFC},在过去十年的时间里,片上缓存容量和交换带宽的比值已经减小超过四倍。

在数据中心链路带宽的增长趋势下,交换机需要预留更多缓存空间作为净空缓存,净空缓存空间严重挤压正常流量能够占用的缓存空间,使得队列长度很容易到达PFC阈值从而导致频繁的PFC触发。即使最新的拥塞控制算法\cite{SIGCOMM15DCQCN,SIGCOMM15TIMELY,SIGCOMM16RDMA,SIGCOMM19HPCC}可以将交换机缓存占用维持在一个较低的水平,上述问题仍然无法完全解决,其原因在于端到端拥塞控制至少需要一个往返时延(Round-Trip Time,RTT)的时间来响应拥塞,对于短时间内的拥塞无能为力,然而短时间内的拥塞在数据中心网络中是十分常见的。研究表明,在未来的数据中心网络中,大多数流将会在一个RTT的时间内完成\cite{SIGCOMM18Homa,SIGCOMM20Aeolus},并且大多数拥塞将会由持续时间不足一个RTT的突发导致\cite{IMC17microburst}。因此,在一个RTT的时间内,能否避免PFC触发仍然取决于缓存管理策略。

数据中心内部交换机通常部署片上缓存系统。经过调研发现,片上缓存系统中的现行静态净空缓存管理策略SIH存在固有的低效性,根源在于SIH按照最坏情况给每个端口的每个入口队列静态预留净空缓存空间,从而导致严重的缓存空间浪费:

1)并不是所有的入口队列都需要占用净空缓存,当且仅当队列发生拥塞时才开始占用,而且所有队列同时拥塞的可能性极低。所以在大多数情况下,只有少量的净空缓存空间被使用。

2)为每个队列按照最坏情况静态预留净空缓存具有明显的低效性。最坏情况下的净空缓存需求量基于端口线速度计算,然而现实中同一端口下的不同队列天然共享其上行链路带宽。因此,所有队列同时以线速打流的最坏情况不可能发生。

另一方面,随着数据中心网络中业务应用的发展,不只是数据中心内部流量,跨数据中心流量同样需要高带宽、低时延的网络服务来保证性能和可靠性\cite{bai2023empowering},如分布式存储\cite{gao2021cloud,calder2011windows}。具体地,同一个应用的计算和存储集群可能位于不同数据中心内部。因此,将RDMA从数据中心内部扩展到跨数据中心之间成为新的发展趋势\cite{zhao2023deterministic}。

跨数据中心传输的时延增加了净空缓存管理的压力。跨数据中心传输在RTT上具有显著的异构性。具体地,跨数据中心长距离传输的RTT在距离变化范围内可达几毫秒至两微秒\cite{bai2023empowering}。净空缓存需求量正比于链路传播时延,在RTT增加到两微秒时,每队列净空需求量可达25MB,片上缓存系统无法满足跨数据中心传输的缓存容量需求。因此,跨数据中心转发设备通常部署异构缓存系统。利用高带宽缓存(High Bandwidth Memory,HBM)技术,异构缓存系统可以在片外提供比片上缓存大数百倍的缓存空间\cite{jedecHBM2E,kim2019design}。异构缓存系统的复杂度和异构性对净空缓存管理机制提出了更高的要求。

% 将RDMA扩展到区域范围则需要跨数据中心转发路由器提供无损网络支持,不同于交换机,路由器需要更大容量的缓存空间,片上缓存无法满足其容量需求。随着2.5D堆叠技术的发展,HBM(High Bandwidth Memory)可以在片外提供容量比片上缓存大百倍的缓存空间[1],但是其带宽仍然无法满足线速转发,不同物理特性的片上与片外缓存共存形成了异构缓存系统架构。异构缓存系统带来的片外带宽瓶颈、出队保序、数据包交织等新特性,导致针对纯片上缓存系统设计的缓存管理算法无法适用。

经过调研发现,主流转发设备厂商针对异构缓存系统的净空缓存管理机制H-SIH存在显著的低效性和性能损害问题:

1)突发吸纳能力受限于片外带宽。H-SIH在拥塞程度较低时可以片外缓存吸纳足量突发,然而,由于片外缓存带宽受限,H-SIH在片外带宽瓶颈时会面临片外队列积累受阻,从而导致H-SIH突发吸纳能力受限。

2)全局流量控制损害流量吞吐性能。PFC控制帧本身对网络性能有害,H-SIH引入的全局流量控制会造成大量队列暂停帧同时触发,进一步加重流量传输性能损害。

3)长距离传输场景下无法避免丢包。长距离传输存在的时延异构性增加了净空缓存分配的压力,H-SIH采用的片上预留净空缓存方式无法完全避免丢包。

现有净空缓存管理机制的低效性和性能问题主要有两方面来源,一方面在于其静态隔离的净空缓存分配方式,另一方面在于缺乏对缓存结构与流量特性的正确认识。因此,不能简单地调节静态净空缓存大小,需要针对片上缓存系统和异构缓存系统架构分别设计新的净空缓存分配机制从根本上解决上述问题。

% \setcounter{subsubsection}{0}

% \subsubsection{净空缓存分配位置:净空缓存从片上缓存还是片外缓存空间分配}

% \subsubsection{净空缓存分配粒度:净空缓存按照端口粒度还是队列粒度分配}

% 针对片上缓存系统,本文提出了动态共享净空缓存管理机制DSH(Dynamic Shared Headroom);针对异构缓存系统,本文提出了面向异构缓存系统的动态共享缓存管理策略(H-DSH)。动态性在于只有队列发生拥塞时才动态地为其分配净空缓存,避免缓存空间浪费。共享性在于通过在不同入口队列之间共享净空缓存,以统计复用的方式高效利用净空缓存。同时,结合端口级别流量控制和队列级别流量控制:一方面,端口级别流量控制基于对同端口下不同队列天然共享上行链路带宽的认识,不需要为每个入口队列独立地预留净空缓存,而仅为每个端口预留一定量的净空缓存使得同一个端口下的入口队列共享这块缓存即可保证无损传输;另一方面,端口级别流量控制会损害流量之间的隔离性,当一个端口的预留净空缓存开始占用时,上行链路的所有流量都会被暂停发送,在大多数情况下仍然采用队列级别流量控制,端口级别流量控制只是作为一个防止丢包的保险机制。不同于纯片上缓存系统,H-DSH在异构缓存系统中进一步解决了片上缓存与片外缓存定位、各自占用时机以及片外缓存压力减轻措施等问题。

% 通过在ns-3仿真平台对DSH和H-DSH进行测试,结果显示DSH可以在不触发PFC的情况下吸收超过4倍大小的突发流量,有效消除PFC引入的头阻和死锁等网络损害。大规模仿真结果显示DSH最高将突发短流的流完成时间减少57.7\%、其它流的流完成时间减少31.1\%;H-DSH可以

% SIH固有的低效性来源于其队列间独立且静态的净空缓存分配方式,简单地调节静态预留净空缓存大小可能会导致丢包的发生。所以,对于纯片上缓存系统,需要设计一个新的净空缓存分配机制从根本上解决这一问题。


\xsection{国内外研究现状}{Research Status}

如何避免PFC的网络损害一直是无损网络领域研究的热点,对此,国内外学者的普遍共识为尽可能地避免PFC触发,理想情况下PFC应该只被作为保证无损传输的备选措施。关于如何避免PFC触发,学术界和工业界主要有两个研究方向:端到端拥塞控制和缓存管理。其中缓存管理针对不同的缓存系统架构目前存在两个研究分支:片上缓存系统缓存管理和异构缓存系统缓存管理。另外,还有很多研究关注通过其它方式消除PFC损害,如负载均衡、死锁检测等。

\xsubsection{端到端拥塞控制}{End-to-end Congestion Control}

为了减少PFC触发,研究人员提出了一系列端到端拥塞控制算法,旨在通过将队列维持在一个较低的长度,从而避免PFC触发。

Brent Stephens等人为了解决无损网络中PFC导致的性能问题,通过结合DCB和DCTCP提出了一个TCP变种TCP-Bolt\cite{INFOCOM14TCP-Bolt},利用DCB防止短时间内的突发导致吞吐受损,同时借助DCTCP将队列长度维持在较低的水平。TCP-Bolt一定程度上解决了无损网络中的高时延、不公平性和头阻问题。同时为了解决PFC死锁问题,提出了一个无死锁路由机制,基于边不相交生成树(Edge-Disjoint Spanning Tree,EDST)保证生成的路由不存在环路,从而避免死锁。

Microsoft和Mellanox提出的DCQCN\cite{SIGCOMM15DCQCN}标志着RDMA在数据中心网络规模化部署的开端。DCQCN指出基于RoCEv2部署RDMA时PFC可能导致头阻和不公平性等问题,提出导致该问题的根本原因在于粗粒度的端口级别流控,无法区分具体的流,进而导致拥塞扩散,网络性能受损。DCQCN结合数据中心TCP\cite{SIGCOMM10DCTCP}(DCTCP)和量化拥塞通知 (Quantized Congestion Notification, QCN) ,基于速率建立了一个细粒度流级别的拥塞控制算法,在不公平和无辜流场景下可以有效减少PFC控制帧数量,从而避免PFC导致的性能损害,提升网络性能。

与DCQCN同时发表在网络顶级会议Sigcomm,Google也提出了RDMA在数据中心网络部署的拥塞控制解决方案Timely\cite{SIGCOMM15TIMELY}。不同于DCQCN,Timely使用数据包的RTT来检测拥塞,RTT测量端在测量到RTT后将其发送给控制端,控制端运行其中的拥塞控制算法根据RTT计算更新之后的目标速率,调节数据包的发送速率。与DCQCN类似,TIMELY利用RTT的变化获得流级别的拥塞信息从而实现流级别的拥塞控制。在RDMA场景下,Timely可以有效减少PFC控制帧数量。

阿里巴巴在多年运行大规模RoCEv2后发现RDMA网络在协调低延迟、高带宽利用率和高稳定性方面面临着根本性的挑战,存在PFC风暴和尾部时延较长等问题,同时指出DCQCN和Timely存在的收敛速度慢和队列积累等问题。为了解决RDMA网络中时延、带宽和稳定性的权衡问题,阿里巴巴提出新一代拥塞控制算法HPCC\cite{SIGCOMM19HPCC},HPCC第一个提出利用带内网络遥测(Inband Network Telemetry,INT)技术来获得精确的链路拥塞信息,根据INT信息精确地计算出更新后的目标发送速率,而不需要同传统拥塞控制算法一样进行迭代逐步收敛到目标速率。HPCC可以将队列长度维持在一个接近于零的水平,很大程度上避免了队列积累而触发PFC。

Wenxue Cheng等人为了解决PFC带来的头阻、不公平和死锁等问题,重新检视了流量控制和拥塞控制之间的相互作用,发现当PFC不断触发时,局部拥塞会扩散回拥塞源和非拥塞源,严重影响网络吞吐量和流完成时间。同时指出这些性能问题的根本解决方案是通过端到端的拥塞控制方案消除持久的拥塞,避免PFC不断触发。基于以上认识对无损以太网的拥塞控制进行了重构,提出PCN\cite{NSDI20PCN},PCN包括两个主要设计,一是拥塞检测与识别机制,用于识别哪些流是造成拥塞的真正原因;二是接收端驱动的速率调整方案,以尽可能快地在一个RTT内缓解拥塞。PCN可以显著减少无损网络中的PFC暂停帧,提高网络性能。

为了逼近数据中心网络的性能极限,Vamsi Addanki等人通过控制理论将数据中心网络现有的拥塞控制算法统一到一个抽象模型中,将现有的拥塞控制算法分为两类:基于网络状态的拥塞控制算法和基于网络变化的拥塞控制算法。通过论证两类拥塞控制各自固有的缺陷,提出了一个新的概念Power,并基于此提出了一种同时基于网络状态和网络变化的拥塞控制算法PowerTCP\cite{NSDI22PowerTCP}。PowerTCP实现了反应速度和控制精度之间的权衡,在将队列长度维持在接近于零、不损害网络吞吐的同时保证了公平性。

Yiran Zhang等人通过深入观察和理解发现主流无损网络中拥塞检测机制的不合理性,不合理之处在于没有认识到逐跳流量控制和交换机中拥塞检测行为之间的相互作用,具体地,交换机ON-OFF发送模式会对交换机中的拥塞检测行为产生意外影响,包括导致队列累积和影响暂停端口的实际输入速率。基于以上认识提出针对无损网络的三元状态拥塞检测TCD\cite{SIGCOMM21TCD},通过未确定态来区分ON-OFF发送模式中的端口状态,实现无损网络中拥塞的更精确识别。

Cisco针对RDMA提出了一个鲁棒拥塞控制机制RoCC\cite{CoNEXT20RoCC},通过控制理论在交换机处计算队列中每条流的公平共享发送速度,将交换机的队列长度作为PI控制器的输入,同时通过自适应调节的PI参数同时实现稳定性、快速收敛、公平性和接近最佳吞吐。稳定性和快速收敛确保了队列不会过度积累,从而减少PFC触发。

Jiao Zhang等人观测到数据中心网络中的拥塞大多发生在最后一跳,基于此提出一个接收端驱动的快速拥塞控制机制RCC\cite{ICNP21RCC}。RCC将最后一跳拥塞和内部拥塞区分开来,结合显式窗口分配和迭代式窗口调节,对于最后一跳拥塞采用显式窗口拥塞,对于内部拥塞采用迭代式窗口调节,RCC可以同时实现快速收敛和接近于零的排队时延。

华为基于同一机架内聚合流寿命相对较长的认识提出了分层RDMA拥塞控制HierCC\cite{zhang2021hiercc},HierCC将发送到机架中相同IP的各个流聚合在一起,使用基于信用的主动拥塞控制来控制机架之间的聚合流速率,同时利用机架中的聚合流获得的带宽将其分配给该机架中相应的单个流。HierCC通过缩短响应环路实现更及时和准确的速率控制。

Xiaolong Zhong等人发现由于端到端的控制回路过长,现有的拥塞控制算法对拥塞的反应迟缓,对微突发甚至无法感知。进而提出了一个交换机驱动的拥塞控制机制PACC\cite{zhong2022pacc},PACC在交换机处基于PI控制器进行计算、基于阈值进行流量分辨、基于权重进行分配,利用实时队列长度主动产生准确的拥塞反馈,进一步提高了对拥塞的反应速度。


\xsubsection{片上缓存系统缓存管理}{Buffer Management for On-Chip Buffer System}

即使端到端拥塞控制算法可以将缓存占用维持在一个较低的水平,PFC带来的问题仍然无法完全解决。端到端拥塞控制响应环路至少需要一个RTT的时间,然而短时间内的拥塞在数据中心网络中是十分常见的。因此,在第一个RTT的时间内,避免触发PFC的关键在于缓存管理策略。在纯片上缓存系统中,关于净空缓存管理策略的研究一直没有停止过。

Mellanox针对交换机中的共享缓存结构提出了一个灵活的缓存分配方案\cite{flexibleBufferAllocation},将整个缓存在逻辑上划分为共享缓存和净空缓存,收到数据包后首先进入净空缓存,根据净空缓存占用情况决定是否触发流量控制以及是否将数据包移入共享缓存。由于只是逻辑上隔离,共享缓存和净空缓存之间数据包的转移并不会涉及物理内存的移动,只需要更新相应的计数器即可。然而,净空缓存总量仍需要按照最坏情况提前预留。

Broadcom针对独立入口缓存的交换机提出在入口缓存中分配净空缓存的策略\cite{ingressBuffering},同时PFC暂停阈值根据共享缓存的使用情况动态调整。在入口缓存分配净空缓存可以降低交换芯片的空间和能耗压力,进一步地,通过在共享缓存根据出端口分区存储数据包,可以减少输出逻辑的读带宽需求。该策略可以释放共享缓存压力,但是并没有减少净空缓存的静态预留量,只是将其转移到了入口缓存中,同时需要依赖交换机支持入口缓存结构。

根据最坏情况给每个队列预留净空缓存导致缓存的利用率低下,为了减少净空缓存需求量,其中一个研究方向是通过在触发PFC流控时引入随机性来避免多个端口同时出现最坏情况。具体实现有不同的方式,Broadcom提出在PFC暂停阈值计算时引入随机偏移量或触发PFC的概率随着队列长度积累不断增加\cite{reducingHeadroom}。Mellanox同样也是在产生PFC暂停帧的时候引入随机延迟,通过三种颜色来区分数据包的优先级,不同优先级数据包在触发PFC时采用不同的概率曲线\cite{losslessBehavior}。一方面,由于暂停帧的生效时延主要来自于传播时延,所以在时延上引入的随机性效果优先;另一方面,引入随机性可以一定降低同时出现最坏情况的概率,但基于此减少净空缓存预留量无法完全避免丢包。

为了减少净空缓存的净空预留量,很多交换机厂商提出了动态分配解决方案。Dell基于每个队列同时发生最坏情况几乎不出现的认识,提出了一个自适应的动态净空缓存分配策略\cite{selfTuningBufferAllocation},静态预留的净空缓存总量小于所有队列最坏情况下的净空缓存需求总和。每个队列分配到的净空缓存可以动态调整,如果当前净空缓存使用率超过了高利用率阈值则进行再分配,低于低利用率阈值则相应需要进行回收。与此类似,Mellanox提出了一个自调整缓存分配方案\cite{adaptiveHeadroomAllocation},将净空缓存分为逐端口的净空缓存池和全局净空缓存池,全局净空缓存用于动态分配,动态分配方式与Dell类似。Broadcom进一步将缓存划分为三部分\cite{adaptiveThresholding}:保留缓存作为每个队列的最小缓存保证;逐队列净空缓存只保证容纳一个RTT内的数据包;全局净空缓存用于动态分配,基本原理与上述策略类似。综上所述,动态分配方式可以减少净空缓存预留量,但是无法保证极端情况下无丢包。

\xsubsection{异构缓存系统缓存管理}{Buffer Management for Hybrid Buffer System}

近几年,随着高带宽存储技术的发展,片上缓存与片外缓存混合的异构缓存系统新架构逐步应用于路由器中。异构缓存系统对于缓存管理算法提出了新的要求与挑战,目前,关于异构缓存系统缓存管理算法的研究还处于初步阶段,现行的净空缓存管理方案主要来自于主流交换机厂商。

Broadcom提出将异构缓存系统中的片外缓存作为超额订购\cite{BCM88480},数据包到达后优先占用片上缓存,只有在片上缓存紧缺时才将数据包搬移到片外。缓存控制逻辑采用串行工作模型,数据包可以从片上缓存再决策搬移到片外缓存。净空缓存在片上进行预留,但是预留总量少于最坏情况下的需求总量。为了保证在最坏情况下无丢包,进一步引入链路级别流量控制,当净空缓存占用超过一定阈值后,需要触发链路级别流量控制暂停所有可能的流量源,以防止净空缓存溢出而发生丢包。

Cisco针对异构缓存系统提出了一个驱逐模型\cite{CiscoNcs5500},缓存管理策略以片上缓存为中心,只有在需要的时候才占用片外缓存,在检测到拥塞时选择将拥塞队列中的数据包决策进入片外缓存,空对立和轻微拥塞队列的数据包进入片上缓存。不同于Broadcom,缓存控制逻辑采用并行工作模型,在缓存管理单元(Memory Management Unit,MMU)决策缓存位置后不会再改变该数据包的缓存位置。为了应对可能出现的片外带宽瓶颈,该策略引入全局流量控制,通过全局暂停流量源来避免片外带宽用尽而丢包。

\xsubsection{PFC损害消除}{PFC Damage Elimination}

关于PFC损害消除的研究也一直没有停止过,Jinbin Hu等人提出PLB\cite{hu2022load}通过负载均衡消除PFC的头阻问题;Chen Tian等人提出的P-PFC\cite{TPDS20P-PFC}通过预测缓存占用提前触发PFC,从而避免突发流导致的队列积累;关于PFC死锁检测、避免和恢复的研究也有很多,如Brent Stephens等人提出的TCP-Bolt\cite{INFOCOM14TCP-Bolt},Shuihai Hu等人提出的Tagger\cite{CoNEXT17Tagger},Kun Qian等人提出的GFC\cite{SIGCOMM19GFC},Xinyu Crystal Wu等人提出的ITSY\cite{INFOCOM22ITSY}。这些机制可以有效消除PFC的部分损害,但未从根本上解决PFC损害。

\xsubsection{总结}{Summary}
综合以上相关工作,其主要思想和特点总结如图\ref{c1:s4:fig:research status}。其中PFC损害消除相关研究仅关注解决PFC导致的损害,未从根本上避免PFC触发;端到端拥塞控制相关研究通过改进拥塞控制机制,将队列维持在较低的水平以避免PFC触发,但是拥塞响应至少需要一个RTT时间;缓存管理可以决定一个RTT时间内的PFC触发,但是现行缓存管理机制存在固有的低效性。本文关注于设计一种更高效的净空缓存管理机制。

\begin{figure}[H]
  \begin{table}[H]
      \begin{tabularx}{\textwidth}{YYcY}
      \toprule
          研究方向 & 方案/厂商 & 基本思想 & 总结 \\
      \midrule
      \multirow{11}{*}{端到端拥塞控制} & TCP-Bolt & 结合DCB和DCTCP & \multirow{11}{*}{\parbox{3cm}{\centering 通过改进拥塞控制机制, 将队列维持在较短长度避免PFC触发,响应环路至少需要一个RTT}} \\
      & DCQCN & 基于QCN提供流粒度拥塞控制 & \\
      & Timely & 基于RTT提供流粒度拥塞控制 & \\
      & HPCC & 基于INT实现精确拥塞控制 & \\
      & PCN & 识别拥塞发生根源,接收端驱动 & \\
      & TCD & 精确识别无损网络中的拥塞 & \\
      & RoCC & 基于控制论计算流的发送速度 & \\
      & PACC & 基于控制论在交换机处检测拥塞 & \\
      & RCC & 区分最后一跳拥塞和内部拥塞 & \\
      & HierCC & 以聚合流粒度进行拥塞控制 & \\
      & PowerTCP & 同时基于网络状态和网络变化 & \\
      \midrule[0.5pt] 
      \multirow{6}{*}{\parbox{3cm}{\centering 缓存管理 \\(片上缓存系统)}} & Mellanox & 逻辑上划分共享缓存和净空缓存 & \multirow{6}{*}{\parbox{3cm}{\centering 按最坏情况为队列预留存在低效性,减少净空缓存预留总量无法避免丢包}}\\
      & Broadcom & 在入口缓存分配净空缓存 & \\
      & Broadcom & PFC阈值计算引入随机偏移 & \\
      & Mellanox & 产生PFC暂停帧时引入随机延迟 & \\
      & \multirow{2}{*}{\parbox{3cm}{\centering Broadcom \\ Dell \ Mellanox}} & \multirow{2}{*}{\parbox{5.5cm}{\centering 减少净空缓存预留总量,每个队列需要的净空缓存动态分配}} & \\
      & & & \\
      \midrule[0.5pt]
      \multirow{2}{*}{\parbox{3cm}{\centering 缓存管理 \\(异构缓存系统)}} & Cisco & 片上缓存资源紧缺时决策到片外 & \multirow{2}{*}{\parbox{3.2cm}{\centering 片外缓存低效,流控损害性能}} \\  
      & Broadcom & 片上缓存资源紧缺时搬移到片外 & \\
      \midrule[0.5pt]
      \multirow{4}{*}{\parbox{3cm}{\centering PFC损害消除}} & PLB & 利用负载均衡消除PFC头阻 & \multirow{4}{*}{\parbox{3cm}{\centering 仅关注PFC损害消除,未从根本上避免PFC触发}} \\ 
      & P-PFC & 提前触发PFC避免队列积累 & \\
      & \multirow{2}{*}{\parbox{3cm}{\centering Tager \ ITSY \\ GFC \ TCP-Bolt}} & \multirow{2}{*}{关注死锁检测、避免以及恢复} & \\
      & & & \\
      \bottomrule
      \end{tabularx}
  \end{table}
  \caption{国内外研究现状总结}
  \label{c1:s4:fig:research status}
\end{figure}

\xsection{论文主要工作}{Main Research Content}

针对无损网络中净空缓存分配的低效性及其导致的性能问题,本文从统计复用的角度出发,研究设计了面向片上缓存系统的动态共享净空缓存管理机制和面向异构缓存系统的动态共享净空缓存管理机制。本文的研究内容如图\ref{c4:s1:ss1:overall architechture}所示,主要工作和创新性总结如下:

1) 面向片上缓存系统的动态共享净空缓存管理机制。针对片上缓存系统现行净空缓存分配机制SIH静态隔离分配方式的固有低效性,本文基于统计复用的思想提出动态共享净空缓存分配机制DSH。其中,动态性主要体现在DSH不再静态地为每个队列预留净空缓存,而是根据队列拥塞状态动态分配。共享性主要体现在两方面:一方面,DSH为每个端口静态预留保底净空缓存,保底净空缓存可以在端口内不同队列之间共享;另一方面,DSH动态分配的共享净空缓存可以在所有队列之间共享。为了验证DSH的性能,本文通过ns-3搭建片上缓存系统并实现DSH缓存管理模块。实验结果表明,在大规模网络中,DSH最多可以将背景流的流完成时间降低31.1\%,突发流的流完成时间降低57.7\%,同时DSH可以有效减少PFC触发,避免PFC造成网络性能损害。

\begin{figure}[H]
  \centering
  \includegraphics[width=\linewidth]{Figures/overall_architechture.pdf}
  \caption{研究内容基本框架}
  \label{c4:s1:ss1:overall architechture}
\end{figure}

% \todo{修改基本框架}

2)面向异构缓存系统的动态共享净空缓存管理机制。针对异构缓存系统现行净空缓存分配机制H-SIH的突发吸纳受限于片外带宽、全局流控损害吞吐和长距离传输丢包问题,本文将动态共享思想扩展到异构缓存系统,提出一个适用于异构缓存系统的净空缓存管理机制H-DSH。在缓存分配上,H-DSH以片外缓存为中心:一方面,H-DSH将片外缓存作为共享缓存空间,在片外带宽足够时将共享缓存空间扩展到片外缓存容量;另一方面,H-DSH优先从片外缓存动态分配共享净空缓存,从而缓解片上净空缓存分配压力。在缓存决策上,H-DSH进一步结合流量的敏感性特征,将报文划分为流量敏感型和带宽敏感型,基于流量敏感性识别机制进行缓存位置决策,充分发挥片上和片外缓存的带宽和容量优势。在流量控制上,H-DSH基于概率模型主动提前触发部分拥塞流量的暂停帧,避免片外带宽瓶颈导致性能受损。本文通过ns-3搭建异构缓存系统并实现H-DSH缓存管理模块。实验结果表明,H-DSH可以有效减少PFC触发,避免流量吞吐受损,对于片外缓存的突发吸纳量的提升超过3倍,最高可以将平均FCT降低14.7\%,同时对无损传输距离提升超过4倍。

\xsection{论文组织结构}{Thesis Structure}

本文主要分为三个部分:第一部分介绍无损网络、缓存系统和网络仿真的相关技术,主要介绍了无损网络中的PFC机制,缓存系统中的共享缓存结构、架构演进、高带宽缓存技术和异构缓存系统工作模型,以及ns-3网络仿真平台的基本架构。第二部分提出面向片上缓存系统的动态共享缓存管理机制,通过仿真实验验证机制有效性。第三部分提出面向异构缓存系统的动态共享缓存管理机制,通过仿真实验验证机制有效性。本文共有五个章节,以下为各章节具体内容:

第一章 \ 绪论。本章首先介绍本文的研究背景与意义,表明本文工作的必要性。然后详细总结了国内外学术界和工业界在本领域的研究工作。最后阐述本文的主要工作和组织架构。

第二章 \ 相关技术。本章首先对缓存系统进行了简要概述,对不同缓存系统进行阐述和对比。其次针对异构缓存系统进行详细介绍。然后对无损网络中PFC机制进行详细说明。最后简要介绍了ns-3网络仿真平台的架构和功能。

第三章 \ 片上缓存系统净空缓存管理机制。本章首先描述片上缓存系统现行净空缓存管理策略SIH的具体机制,指出其在净空缓存分配上固有的低效性,针对SIH的缓存资源浪费和频繁PFC频繁触发等问题,提出了一种动态共享净空缓存分配机制DSH。通过仿真实验与SIH进行性能对比,验证DSH有效性。

第四章 \ 异构缓存系统净空缓存管理机制。本章首先描述异构缓存系统现行净空缓存管理策略H-SIH具体机制,指出H-SIH的片外缓存利用低效性、全局流量控制损害吞吐和长距离传输丢包问题。针对以上问题提出了一种适用于异构缓存系统架构的动态共享净空缓存管理机制H-DSH。通过仿真实验与H-SIH进行性能对比,验证H-DSH有效性。

第五章 \ 总结与展望。本章首先总结本文工作,然后对本工作改进点和未来研究工作进行展望。


\clearpage

% !TeX root = ../main.tex

\xchapter{相关技术}{Related Technologies}

\xsection{无损网络}{Lossless Network}

\xsubsection{基本介绍}{Basic Introduction}

数据中心通过部署RDMA提供高带宽和超低时延的网络服务\cite{wang2021datacenter}。RDMA利用相关硬件和网络技术,使主机的网卡之间可以直接读取内存,从而实现高带宽、低时延和低CPU开销的网络传输。RDMA提出之初承载在无损的IB(Infinite Band)网络中,专用的IB网络架构封闭,无法兼容现网,使用成本较高。RoCE(RDMA over Converged Ethernet)进一步提出,RoCE使用以太网承载RDMA的网络协议,目前主要有两个版本:RoCEv1和RoCEv2。

\begin{figure}[H]
  \centering
  \includegraphics[width=0.7\linewidth]{Figures/lossless_network.pdf}
  \caption{无损网络部署场景}
  \label{c4:s1:ss1:lossless network}
\end{figure}

无损网络的常见部署场景如图\ref{c4:s1:ss1:lossless network}所示,分布式存储、高性能计算(High Performance Computing,HPC)、分布式机器学习等场景广泛采用RoCEv2协议来提升网络性能\cite{zhao2021intelligent}。RoCEv2是一种基于无连接的UDP协议,缺乏完善的丢包保护机制,对于网络丢包异常敏感。同时,分布式高性能应用通常为多对一通信的Incast流量模型,对于以太网的设备,Incast流量易造成设备内队列的瞬时突发拥塞甚至丢包,导致时延的增加和吞吐的下降,从而损害分布式应用的性能\cite{calder2011windows,zaharia2012resilient}。因此,为了发挥出RDMA的真正性能,突破数据中心大规模分布式系统的网络性能瓶颈,需要为RDMA搭建一套无丢包、低时延、高吞吐的无损网络环境。

无损网络即无丢包的网络\cite{SIGCOMM15DCQCN,lu2018multi,NSDI20PCN},具体地,不因转发设备缓存溢出而导致丢包,其目的是为了提供高带宽和超低时延的网络服务。相对于传统易丢包、高延迟的有损网络而言,无损网络需要在流量控制、拥塞控制、路由选择和缓存管理等方面进行较大改进。目前主流的无损网络架构主要有两类:增强型以太网\cite{reinemo2010ethernet}(Converged Enhanced Ethernet,CEE)和IB网络\cite{SIGCOMM15DCQCN,zhu2015packet}(InfiniBand network)。IB网络主要部署于HPC系统中,由于传统网络中基于TCP/IP的以太网技术占据主导地位,相较于基于无损以太网的CEE,IB网络和原有的网络架构分离,无法兼容现网。因此CEE在数据中心网络应用中得到了更广泛的应用。


\xsubsection{基于优先级流量控制}{Priority-based Flow Control}
\label{c2:s4:priority-based flow control}

基于以太网的数据中心网络通过PFC\cite{PFC}保证无损数据传输。如图\ref{fig:c2:priority-based flow control}所示,PFC是一种逐跳运行的流量控制机制,在支持PFC的转发设备中,如果一个入口队列的长度超过预设的阈值$X_{off}$,交换机将会向上游设备发送一个暂停帧。上游设备收到这个暂停帧后暂停向下游发送报文,暂停帧中携带暂停发送的持续时间。当被暂停入口队列的长度小于阈值$X_{on}$后,交换机会向上游设备发送恢复帧(暂停持续时间为0的暂停帧)以恢复数据包的发送。

\begin{figure}[H]
  \centering
  \includegraphics[width=0.8\textwidth]{priority_based_flow_control.pdf}
  \caption{基于优先级的流量控制}
  \label{fig:c2:priority-based flow control}
\end{figure}

为了防止丢包,$X_{off}$的大小需要保守设置。其原因在于暂停帧需要一定的时延才能到达上游设备并且真正发挥作用,所以需要预留足够的缓存空间用于存储该时延内到达的数据包,这部分缓存空间称为净空缓存。PFC标准支持8个优先级类别\cite{PFC},报文到达后被分类到不同的优先级,每个优先级被映射到一个单独的队列,不同优先级的报文进入不同的队列。PFC控制帧中携带优先级信息,所以仅暂停或者恢复指定的流量类别。

净空缓存大小需要合理地配置以避免丢包。暂停帧需要一定的时延才能发挥其作用,所以MMU需要预留足够地净空缓存空间吸纳这段时延内到达地数据包。参考相关文献\cite{SIGCOMM15DCQCN,SIGCOMM16RDMA,CiscoNexus7000PFC,PFCProposal},净空缓存理论上按照如下公式配置:
\begin{equation}
    \eta = 2(C \cdot D_{prop} + L_{MTU}) + \text{3840B}
    \label{eqn:c2:headroom calculation}
\end{equation}

其中,$C$表示上行链路的带宽,$D_{prop}$表示上行链路的传播时延,$L_{MTU}$为最大传输单元(Maximum Transmission Unit,MTU)的包大小,B为单位,表示字节数。PFC暂停帧发挥作用需要的时延由以下五部分组成:

1)等待时延:暂停帧产生的时候所在端口可能正在发送另一个数据包,需要等待其传输结束才能开始传输暂停帧。最坏情况下,这个端口可能恰好开始传输一个一个MTU大小数据包的第一个比特,因此暂停帧需要等待的时间为$L_{MTU} / C$。

2)传播时延(暂停帧):暂停帧传播到上游设备的时延$D_{prop}$决定于传播距离和信号的传播速度。在数据中心中,两个直接连接的交换机最大可达300米\cite{SIGCOMM16RDMA};对于单模光纤,光信号的传播速度可以达到真空中光速的65\%。

3)处理时延:交换机需要花费一定时间去处理暂停帧之后才能开始暂停数据包的传输,PFC中定义这段时延最大为$3840B/C$\cite{CiscoNexus7000PFC}。

4)响应时延:当上游设备开始执行暂停操作时可能正在发送另一个数据包。最坏情况下,交换机恰好开始发送一个MTU大小数据包的第一个比特,因此,上游设备仍然需要等待$L_{MTU} / C$时延后才能真正开始暂停动作。

5)传播时延(最后一个数据包):当上游设备停止发送数据包时,链路上还有一些正在传输中的数据包将会被下游设备接收到,这些数据包也应该被净空缓存吸纳掉。直到链路上的最后一个数据包到达下游设备仍需要$D_{prop}$的传播时延。 


\xsubsection{PFC转发设备缓存分区}{Buffer Partition in PFC-enabled Device}
\label{c2:buffer partition in pfc-enabled device}

转发设备中等待传输的数据包被暂存在缓存中。为了实现高速低时延的数据包访问,商业高速交换芯片通常采用片上缓存\cite{SIGCOMM16RDMA,SIGCOMM15Jupiter,BroadcomSmartBuffer,ExtremeBuffer,BroadcomTrident3,BroadcomTomahawk4,CiscoNexus9300Buffer,Arista7050X3}。路由器对于时延的容忍度较高,需要大容量缓存空间,通常在片上缓存的基础上搭载片外缓存\cite{CiscoNcs5500,BCM88480}。缓存系统通常采用共享缓存结构,缓存在不同端口或队列之间的分配由缓存管理单元(Memory Management Unit, MMU)管理。如图\ref{fig:c2:buffer partition}所示,在支持PFC的转发设备中,片上缓存通常被划分为两个缓存池:无损缓存池和有损缓存池\cite{CiscoNexus9300Buffer,BCM88800TM,MellanoxRoCEConfig}。无损缓存池专用于无损流量,利用PFC来避免丢包。有损缓存池专用于有损流量,其中的流量允许在缓存溢出时发生丢包。缓存通过硬分区来保证无损流量和有损流量之间的隔离性。本文工作主要针对无损缓存池。

\begin{figure}[H]
  \centering
  \includegraphics[width=0.7\textwidth]{buffer_partition.pdf}
  \caption{PFC转发设备缓存分区}
  \label{fig:c2:buffer partition}
\end{figure}

在无损缓存池中,缓存进一步划分为三个分区:

1)私有缓存:为每个队列预留的缓存空间,保证每个队列的最小缓存资源占用。

2)共享缓存:在所有队列之间共享的缓存空间,根据队列长度动态分配。

3)净空缓存:给每个队列预留的缓存空间,用于存储PFC暂停帧发送后生效前到达的数据包。


\xsection{缓存技术}{Buffer Technology}

缓存在网络中发挥着重要的作用,由于网络中不可避免会存在拥塞和突发,很多协议如TCP正是依靠网络中的拥塞来实现可靠传输,同时TCP本身不可避免地向网络中引入突发,以及设备转发过程中可能存在的调度时延,所以报文在到达转发设备后不一定能够直接发送。为了避免丢包,无法即时发送的数据包需要在转发设备进行暂存。缓存即用来暂存不能立即发送的数据包,历史上关于缓存大小达成的共识为缓存至少需要容纳源节点和目的节点之间的发送中数据包量,即一个时延带宽积(Bandwidth Delay Product,BDP)的大小。另外,缓存需要满足设备的带宽需求以保证所有同时到达的流量可以被缓存完全吸收。

\xsubsection{共享缓存结构}{Shared Buffer Structure}

转发设备(包括交换机和路由器)中最常见的缓存结构为共享缓存结构,在该结构中,所有出端口和入端口共享同一块缓存空间\cite{9776493,SIGCOMM10DCTCP,cummings2010sharedmemory,broadcom2012smartbuffer}。如图\ref{fig:c2:shared buffer structure}所示,当数据包到达入端口后被写入共享缓存中暂存,当该数据包被调度出队时会从共享缓存读出经出端口转发。整个共享缓存被划分为多个队列,每个端口可以有一个或多个队列,由调度策略决定出队队列。

\begin{figure}[H]
  \centering
  \includegraphics[width=0.75\textwidth]{shared_buffer.pdf}
  \caption{共享缓存结构}
  \label{fig:c2:shared buffer structure}
\end{figure}

\xsubsection{缓存系统架构}{Buffer System Architecture}

如图\ref{fig:c2:buffer system architecture}所示,整个缓存系统包括缓存模块、缓存管理模块、队列管理模块和保序模块。缓存模块即存储器,用于暂存数据包;缓存管理模块对缓存的使用情况进行监测并根据监测结果在数据包到达时进行决策;队列管理模块负责维护每个端口的出口队列,其中调度器用于在多个队列之间进行调度决定出队队列;保序模块用于对出队数据包进行重排序,以保证每个队列数据包发送时不会乱序。

数据包在缓存系统中的具体处理流程总结如下:数据包进到达入端口后首先由控制器中的缓存管理模块根据缓存占用情况和缓存管理策略进行决策,决定该数据包在缓存中的存储位置或者直接将其丢弃,数据包决定存入缓存的同时会由队列管理模块决定其入队队列并将其入队,出口队列只有一个逻辑上的队列,其中存放的为数据包描述符,其中包含数据包指针,即数据包在缓存中的地址。队列管理模块通过划分多队列来实现QoS差异化服务,调度器中的调度策略决定从多个队列中挑选出队队列,出队数据包描述符去往保序模块的同时从缓存中读取对应数据包,保序模块为出队数据包申请一个保序标识,当数据包从缓存中读出后根据保序标识对数据包进行重排序,最终按照正确的顺序将数据包送往出端口发出。

\begin{figure}[H]
  \centering
  \includegraphics[width=0.8\textwidth]{buffer_system_architecture.pdf}
  \caption{缓存系统架构}
  \label{fig:c2:buffer system architecture}
\end{figure}

% \todo{报文处理流程}
% \begin{tcolorbox}[height=5cm,colback=black!5!white,colframe=blue!75!black]
%   报文处理流程
% \end{tcolorbox}

\xsubsection{缓存系统类型}{Buffer System Type}

根据缓存系统的缓存组成以及物理特性可以将其划分为不同的类型,常见的缓存系统类型包括纯片上缓存系统、非收敛片外缓存系统和收敛片外缓存系统:

\subsubsection{片上缓存系统}

\begin{figure}[H]
  \centering
  \includegraphics[width=0.8\textwidth]{onchip_buffer_system.pdf}
  \caption{纯片上缓存系统}
  \label{fig:c2:on-chip buffer system}
\end{figure}

纯片上缓存系统的缓存位于交换芯片上,不存在外部存储器。片上缓存采用静态随机存储器(Static Random Access Memory,SRAM)技术,SRAM读取速度快,可以在一个时钟周期内完成完成存取操作,其带宽可以满足交换机线速写入和读出的需求,读写时延仅为纳秒级别。但是受限于交换芯片面积和能耗,片上缓存无法做到较大的容量,通常只有几十兆字节数大小,对于突发吸纳容量敏感。纯片上缓存系统的工作流程如图\ref{fig:c2:on-chip buffer system}所示,一个数据包到达输入端口时,可以被控制器决策直接线速存入片上缓存。在该数据包位于队列首部且所在队列获得调度机会时,控制器将其从片上缓存线速读出经出端口发送。纯片上缓存结构一般应用于交换机中,以满足交换机高速低时延的报文转发需求。

\subsubsection{非收敛片外缓存系统}

非收敛片外缓存系统在芯片上部署片上缓存的同时搭载一块片外缓存。如图\ref{subfig:c2:non-convergent off-chip buffer system}所示,片外缓存系统中新增了在片外缓存进行数据包存取的数据通路,片外缓存采用动态随机存储器(Dynamic Random Access Memory,DRAM)技术,DRAM存储成本较低,因此常被用来作为大容量外部存储器,片外缓存容量可以做到千兆级别。在非收敛片外缓存系统中,片外缓存和片内缓存带宽均大于两倍的芯片吞吐,因此都可以实现和芯片的线速数据包交换。但是相对于时延仅两三百纳秒的片上缓存,片外缓存的读写时延较高,达到微秒级别。所以,相对于纯片上缓存系统,非收敛片外缓存系统通过搭载片外缓存将容量扩展到千兆,可以支持海量队列应用场景,但是同时带来了较大的队列时延和时延抖动,以及片上缓存和片外缓存时延差异导致的队列头阻问题。路由器对于时延有较高的容忍度且需要大容量缓存,因此非收敛片外缓存系统常用于传统路由器。

\begin{figure}[H]
  \begin{subfigure}[b]{0.49\linewidth}
      \centering
      \includegraphics[width=\linewidth]{non_convergent_offchip_buffer_system.pdf}
      \subcaption{非收敛片外缓存系统}
      \label{subfig:c2:non-convergent off-chip buffer system}
  \end{subfigure}
  \begin{subfigure}[b]{0.49\linewidth}
      \centering
      \includegraphics[width=\linewidth]{convergent_offchip_buffer_system.pdf}
      \subcaption{收敛片外缓存系统}
      \label{subfig:c2:convergent off-chip buffer system}
  \end{subfigure}
  \caption{片外缓存系统}
\end{figure}

\subsubsection{收敛片外缓存系统}

收敛片外缓存系统即异构缓存系统,与非收敛片外缓存系统类似,收敛片外缓存系统同样结合SRAM和DRAM技术,利用DRAM扩展缓存容量。不同的是,收敛片外缓存系统存在片外带宽瓶颈,其带宽仅为20\%到40\%的芯片读写带宽。对于带宽的收紧降低了片外缓存系统的成本开销,但是可能增加从片外缓存读取数据包的时延或者加剧队列头阻问题。收敛片外缓存系统中需要将片外缓存带宽和片内缓存容量作为资源进行合理高效地管理,使其性能接近甚至到达与非收敛片外缓存系统同等水平,所以在收敛片外缓存系统中缓存管理策略尤为重要。其数据通路如图\ref{subfig:c2:convergent off-chip buffer system}所示,在片外带宽使用未知的情况下将数据包决策进入片外缓存时无法保证其成功写入到片外缓存,可能由于片外缓存带宽用尽而丢包。


\xsection{异构缓存系统}{Hybrid Buffer System}

异构缓存系统中同时部署片上缓存和片外缓存,片上缓存将SRAM存储器集成在芯片上,片外缓存通过HBM技术\cite{jedecHBM2E,kim2019design}将DRAM存储器在片外堆叠。在异构缓存系统中,一个数据包从到达系统到离开系统完整生命周期内可能经历不同的操作,数据包从入端口进入后需要决策其是否存储以及存储位置,存储缓存位置确定后仍然可能经历搬移或者丢弃,调度出队时从缓存读出经出端口发送。理论上异构缓存系统支持完整的数据包处理流程,但在实际操作中,并不是所有的流程都会利用到。不同的流程选择直接影响异构缓存系统的服务质量(Quality of Service,QoS),根据其流程选择可以将异构缓存系统进一步划分为不同的工作模型。

\xsubsection{高带宽缓存技术}{High Bandwidth Memory Technology}

HBM技术是一种新型的存储技术,相对于传统的双倍速率(Double Data Rate,DDR)内存芯片,HBM可以提供更高的内存带宽、更低的访问延迟、更大的容量以及更低的能耗。目前,HBM被广泛应用于高性能计算、图形处理、人工智能和机器学习以及网络设备等领域。

HBM在架构上采用了3D堆叠封装技术,如图\ref{fig:c2:hbm architecture}所示,HBM通过将DRAM芯片以垂直堆叠的方式组织在一起\cite{jedecHBM2E,kim2019design}。每个HBM存储器堆由多个DRAM芯片组成,通过硅互连通道进行连接,形成一个垂直堆叠的结构。这种堆叠架构使HBM具有高密度和较小的占用空间,同时实现了更短的信号传输距离,从而提供了更高的带宽。尽管每个HBM存储器堆的容量相对较小,但通过堆叠多个存储器堆,使HBM可以实现较大的总容量。同时,HBM通过每个DRAM芯片上的多个通道实现并行数据传输。每个HBM存储器堆可以提供多个通道,每个通道可以同时传输数据,有效地提高了数据传输速率。另外,HBM在设计上优化了功耗效率。得益于其堆叠结构和短距离信号传输,HBM能够以较低的电压和功耗提供高带宽,从而降低整个系统的能耗。同时,HBM具有容量可扩展性。HBM的单层DRAM芯片容量可扩展;HBM通过4层、8层以至12层堆叠的DRAM芯片,可实现更大的存储容量;HBM还可以通过3D系统级封装集成多个HBM叠层DRAM芯片,从而实现更大的内存容量。最新的HBM3堆栈容量可达24GB\cite{DYFZ202302005}。

\begin{figure}[H]
  \centering
  \includegraphics[width=0.9\textwidth]{hbm_architecture.pdf}
  \caption{3D堆叠封装技术}
  \label{fig:c2:hbm architecture}
\end{figure}

在网络设备中,HBM通常作为外部存储器用来构建异构缓存系统。相对于芯片内部的SRAM,HBM可以提供超过百倍大小的缓存容量。通过结合片上存储和片外存储的异构缓存架构,可以提高网络设备的性能和扩展性,充分结合片上存储高带宽和片外存储大容量的优势。

\xsubsection{工作模型}{Work Flow}

在异构缓存系统中,一个数据包从到达系统到离开系统完整生命周期内可能经历不同的处理通路。通用工作模型中包括完整报文处理通路,根据缓存系统支持的报文处理通路可以进一步划分为串行工作模型和并行工作模型。

\subsubsection{通用工作模型}

\begin{figure}[H]
  \centering
  \includegraphics[width=0.75\textwidth]{general_work_flow.pdf}
  \caption{通用工作模型}
  \label{fig:c2:general work flow}
\end{figure}

异构缓存系统通用工作模型的完整数据流程如图\ref{fig:c2:general work flow}所示,一个数据包从入端口进入设备后,首先由控制器进行决策,决策的结果包括丢弃、进入片上缓存和进入片外缓存,在存储转发模型中,数据包必须完整地接收并存储到缓存中才能进行转发,所以不存在直接去往出端口的数据通路。数据包在片上缓存或者片外缓存存储期间,可以对数据包进行丢弃或者片上和片外缓存相互之间的搬移。在调度决策该数据包出队时,数据包可以片上缓存或者片外缓存读出交给出端口进行发送。

\subsubsection{并行工作模型}

\begin{figure}[H]
  \begin{subfigure}[b]{0.49\linewidth}
      \centering
      \includegraphics[height=5.8cm]{parallel_work_flow.pdf}
      \subcaption{数据流}
      \label{subfig:c2:parallel work flow}
  \end{subfigure}
  \begin{subfigure}[b]{0.49\linewidth}
      \centering
      \includegraphics[height=5.8cm]{out_of_order_in_parallel.pdf}
      \subcaption{报文交织}
      \label{subfig:c2:out of order}
  \end{subfigure}
  \caption{并行工作模型}
\end{figure}

异构缓存系统并行工作模型的完整数据流程如图\ref{subfig:c2:parallel work flow}所示,并行工作模型中不会进行数据包搬移,数据包只有在入队决策时才允许进行丢弃操作。数据包入队时决策其是否丢弃以及存储的位置,根据决策结果进入片上或片外缓存,之后不会再进行缓存位置变动,直到数据包出队时从缓存读出经出端口发送。

由于片上缓存和片外缓存之间存在读取时延差异,异构缓存系统中出队数据包可能会产生交织,并行工作模型中可能存在的报文交织场景如图所示,图\ref{subfig:c2:out of order}中显示了一个出口队列中的排队报文,其中P1、P3和P6存储在片外缓存中,P2、P4和P5存储于片上缓存,假设当前所有队列中仅该队列中有报文积累且后续没有新的报文入队,调度策略采用差分轮询(Deficit Round Robin,DRR)。经过六轮调度之后,该队列中所有报文均完成出队,由于片上片外缓存读取时延差异,P2、P4和P5先到达保序模块之后,P1、P3和P6经过一定时延之后才能到达,此时即使P2、P4和P5可以到达,由于保序要求仍需要等待队列前面的数据包到达并发送之后才能进行转发,由此报文交织导致时延抖动。

\subsubsection{串行工作模型}

\begin{figure}[H]
  \centering
  \includegraphics[width=0.75\textwidth]{serial_work_flow.pdf}
  \caption{串行工作模型}
  \label{fig:c2:serial work flow}
\end{figure}

异构缓存系统串行工作模型的完整数据流程如图\ref{fig:c2:serial work flow}所示,串行工作模型允许进行片上缓存向片外缓存的报文搬移,同样报文只有在入队决策时才允许进行丢弃操作。不同于并行工作模型,报文入队时决策丢弃还是存储,如果存储则全部存储于片上缓存,不会在决策进入片外缓存,后续在适当的时机可以将报文从片上搬移到片外,如队列拥塞时,所以片外缓存中的报文只可能从片上缓存搬移后进入。在进行报文搬移时存在两种搬移方式:队头搬移和队尾搬移。

队头搬移过程如图\ref{subfig:c2:move from head}所示,将排在队头的P1、P2和P3搬移进入片外缓存,新到达的报文P7进入片上缓存,队头搬移后出队时,由于产生交织,片上缓存的P4、P5、P6和P7需要等待P1、P2和P3从片外缓存读出后才能进行转发。

队尾搬移过程如图\ref{subfig:c2:move from tail}所示,将排在队尾的P4、P5和P6搬移进入片外缓存。另外,从队尾搬移时需要增加前向链表结构以便找到节点的前趋节点,新到达的报文P7进入片上缓存后产生交织,搬移P7之后不利于下次搬移时查找片上缓存的队尾报文P3。

\begin{figure}[H]
  \begin{subfigure}[b]{0.49\linewidth}
      \centering
      \includegraphics[height=5.8cm]{move_from_head.pdf}
      \subcaption{队头搬移}
      \label{subfig:c2:move from head}
  \end{subfigure}
  \begin{subfigure}[b]{0.49\linewidth}
      \centering
      \includegraphics[height=5.8cm]{move_from_tail.pdf}
      \subcaption{队尾搬移}
      \label{subfig:c2:move from tail}
  \end{subfigure}
  \caption{串行工作模型不同搬移方式}
\end{figure}

\xsection{ns-3网络仿真平台}{ns-3 Network Simulator}

本文基于ns-3网络仿真平台模拟真实网络环境。ns-3是一个由事件驱动的离散事件仿真器,ns-3提供了有关分组数据网络如何工作和执行的模型,并为用户提供了用于进行仿真实验的仿真引擎\cite{ns-3,matthias2010ns3},其系统架构如图\ref{fig:c2:ns3 architecture}所示。

\begin{figure}[H]
  \centering
  \includegraphics[height=6cm]{ns3_architecture.pdf}
  \caption{ns-3系统架构}
  \label{fig:c2:ns3 architecture}
\end{figure}

ns-3能够在一台计算机中模拟出各种类型和规模的网络结构。现实中的计算机网络主要由两部分组成:网络拓扑和网络协议。网络拓扑是由许多结点和连接这些结点的链路组成,网络协议运行在这些结点的协议栈中。ns-3模拟这样的一个网络的基本原理是:首先,ns-3将网络拓扑结构中的结点和链路抽象成C++中的各种类,比如Node类和Channel类,结点和链路之间的连接操作则被抽象为这些对象之间的联系。ns-3通过这些抽象模拟出物理网络中的物理模型、传输协议和网络拓扑结构,比如交换机、点对点协议(Point to Point Protocol, PPP)和基于IEEE 802.11系列标准的无线局域网等。

ns-3使用离散事件来模拟网络中的各种传输协议。通过把真实物理世界中连续的过程抽象成了仿真器中虚拟的离散事件,从而模拟现实计算机网络中的各种协议,ns-3支持的协议如图\ref{fig:c2:protocols supported in ns-3}所示,包括物理层的IEEE 802.11系列、PPP、链路层的ARP、网络层的IPv4、IPv6、传输层的UDP、TCP协议和应用层的ping协议等。

为了方便用户使用,ns-3还提供了诸多辅助功能。如trace功能,用户可以利用wiresharktcpdump和tcpdump等软件简单便捷地分析ns-3仿真中的数据,ns-3中集成的移动模块使得在移动网络中移动结点和分配移动轨迹变得简单方便。除此之外,ns-3模拟出来的虚拟网络可以很好地和现实的物理网络进行融合。一方面,物理网络中的结点设备可以使用ns-3模拟的虚拟链路发送和接收数据包;另一方面,ns-3模拟的结点也可以利用现实中的物理网络链路传输数据。

\begin{figure}[H]
  \begin{table}[H]
      \begin{tabularx}{\textwidth}{cY}
      \toprule
          网络层次 & 网络协议 \\
      \midrule
          应用层 & 分组产生器、应用层协议ping等 \\
          传输层 & UDP、TCP \\
          网络层 & IPv4、IPv6和静态路由、OSPF、BGP等路由协议 \\
          链路层 & Ethernet(IEEE 802.3)、Wi-Fi(IEEE 802.11)、PPP和ARP等 \\
          物理层 & Wi-Fi(IEEE 802.11)、WiMAX(IEEE 802.16)和LTE等 \\
      \bottomrule
      \end{tabularx}
  \end{table}
  \caption{ns-3支持的各层协议}
  \label{fig:c2:protocols supported in ns-3}
\end{figure}

用户可以利用ns-3模拟出现实中的大规模物理网络,并且在其中实现自己的协议和其它控制代码进行测试,这样就不需要搭建真实的网络,既降低了操作难度,也可以降低开发成本。目前,ns-3网络仿真平台仍然在不断的开发和更新中,在物联网、数据中心、软件定义网络、第五代移动通信(5G)等网络中得到了广泛的应用。

本文基于ns-3仿真片上缓存系统和异构缓存系统模型。实现的主要功能模块包括片上缓存、片外缓存、缓存管理模块、队列管理模块和保序模块。片上缓存通过扩展Node类实现,片外缓存由OffChipBuffer类实现,模拟HBM读写带宽、总线冲突、读写时序和批处理等物理特性,缓存管理模块通过Mmu类实现,可以通过组合到Node类中为其提供缓存监测和准入决策功能,队列管理模块通过ns-3内置的排队规则(Queue Discipline,qdisc)实现,基于qdisc实现不同的调度策略,并通过Filter和Class构建多层级调度模型,保序模块通过扩展NetDevice模块的Queue实现保序功能。其中缓存管理策略通过在Mmu中实现,包括面向片上缓存系统的SIH和DSH,以及面向异构缓存系统的H-SIH和H-DSH。


\xsection{本章小结}{Brief Summary}

本章首先对无损网络相关技术进行介绍,对无损网络的部署需求、应用场景、实现技术和主要架构等技术进行简要介绍,主要针对PFC机制工作原理和PFC设备缓存分区结构进行了详细描述。其次对缓存系统进行概述,主要涉及缓存作用、需求和结构,总结并对比了不同类型缓存系统的结构和特点。然后针对异构缓存系统进行详细阐述,介绍HBM缓存技术在异构缓存系统中的应用,总结了串行工作模型和并行工作模型的特点和区别。最后简要介绍了ns-3网络仿真平台的架构和功能。

\clearpage
% !TeX root = ../main.tex

% !TeX root = ../main.tex

\xchapter{片上缓存系统净空缓存管理机制}{Headroom Management Scheme for On-chip Buffer System}

本章针对片上缓存系统现行净空缓存管理策略存在的低效性问题,分析其低效性的固有根源以及寻求更高效缓存管理策略的必要性,提出了一种动态共享净空缓存管理策略DSH,DSH动态地为拥塞队列分配净空缓存,同时在所有队列之间共享分配的净空缓存,通过统计复用的方式减少缓存静态预留量,提高缓存利用率。本章\ref{c3:s1:current buffer management scheme}节阐述片上缓存系统现行缓存管理策略SIH的具体机制。\ref{c3:s2:problem analysic}节分析SIH固有低效性的根本来源。\ref{c3:s3:dsh design}节提出DSH的设计目标、基本思想、主要挑战以及各个功能模块的具体机制。\ref{c3:s4:dsh analysis}节通过理论证明DSH的突发吸纳优势。\ref{c3:s5:dsh implementation}节详细描述DSH各个模块的具体工作方式和算法实现。\ref{c3:s6:dsh evaluation}节测试DSH的基本能力和大规模网络性能表现。\ref{c3:s7:brief summary}节对本章主要工作进行总结。

\xsection{现行缓存管理机制}{Current Buffer Management Scheme}
\label{c3:s1:current buffer management scheme}

片上缓存系统通常部署于交换机中,交换芯片中缓存管理单元(Memory Management Unit,MMU)负责缓存分配与管理,MMU为每个到达的数据包分配缓存空间。对于无损流量,MMU在入口侧为每个端口/队列分配缓存\cite{BCM88800TM,MellanoxRoCEConfig,CiscoNexus9300}。

\xsubsection{缓存分配}{Buffer Allocation}
现行净空缓存分配机制SIH的缓存分区结构同\ref{c2:buffer partition in pfc-enabled device}节所述,无损缓存池进一步划分为私有缓存、共享缓存和净空缓存三个分区。其中私有缓存和净空缓存的大小通常是显式静态配置的,剩余缓存空间作为共享缓存进行动态分配。

\subsubsection{私有缓存}

对于私有缓存,没有明确的规定应该如何配置其大小。交换机通常静态配置一个相对较小的空间\cite{SIGCOMM15DCQCN,SIGCOMM16RDMA},以保证每个队列的最少可用缓存资源,如Arista 7050X3交换机中配置为16\%的缓存空间\cite{Arista7050X3}。

\subsubsection{净空缓存}

对于净空缓存,为了避免缓存溢出而丢包,交换机通常按照最坏情况下的净空缓存需求量静态预留。最坏情况下的净空缓存需求量计算方法如\ref{c2:s4:priority-based flow control}节所述,SIH为每个队列静态预留$\eta$大小的净空缓存,$\eta$由公式(\ref{eqn:c2:headroom calculation})计算得到。

\subsubsection{共享缓存}

对于共享缓存,该空间在所有队列之间共享,在队列需要时动态分配。MMU通过缓存管理机制保证共享缓存空间在所有队列之间公平和高效地共享,动态阈值(Dynamic Threshold, DT)是商用交换芯片中最常用的缓存管理机制\cite{SIGCOMM16RDMA,SIGCOMM19HPCC,broadcom2012smartbuffer,ExtremeBuffer,Arista7050X3,BCM88800TM,MellanoxDT,CiscoNexus9000ConfigGuide,BS19Yahoo}。

DT用同一个阈值限制每个队列的长度,该阈值根据剩余缓存空间的大小动态地调整。具体地,用$T(t)$表示表示$t$时刻的阈值,$\omega_s^{i,j}(t)$表示$t$时刻端口$i$中队列$j$的共享缓存占用量,$B_s$表示共享缓存分区大小,DT阈值可以用下式表示:
\begin{equation}
  T(t)=\alpha \cdot (B_s - \sum_{i} \sum_{j} \omega_s^{i,j}(t))
  \label{eqn:c3:dt threshold}
\end{equation}

其中$\alpha$是DT中的控制参数。DT蕴含的基本原理如下:当网络拥塞程度下降时,缓存占用总量减少即剩余缓存空间增加,DT阈值随之增大,则每个队列允许占用更多缓存从而提高缓存利用率;相反地,当网络拥塞程度增加时,缓存占用总量增加即剩余缓存空间减少,DT阈值随之减小,此时可以限制每个队列的缓存占用从而保证不同队列之间的公平性。

\xsubsection{流量控制}{Flow Control}
\label{c3:s1:ss2:flow control}

在支持PFC的网络中,MMU不仅监测入口队列长度,而且对每个到来的数据包进行决策,同时根据共享缓存占用的情况和当前阈值向上游设备发送PFC暂停/恢复帧,PFC的工作机制可以通过图\ref{c3:s1:ss2:fig:pfc state transition}中的状态机描述。

\begin{figure}[H]
  \begin{subfigure}[b]{0.49\linewidth}
      \centering
      \includegraphics[width=\linewidth]{state_transition_pfc.pdf}
      \subcaption{入口队列}
      \label{c3:s1:ss2:fig:sub1:ingress}
  \end{subfigure}
  \begin{subfigure}[b]{0.49\linewidth}
      \centering
      \includegraphics[width=\linewidth]{state_transition_pfc_out.pdf}
      \subcaption{出口队列}
      \label{c3:s1:ss2:fig:sub2:egress}
  \end{subfigure}
  \caption{PFC状态转换}
  \label{c3:s1:ss2:fig:pfc state transition}
\end{figure}

在下游接收端,每个入口队列存在两种状态:

1)ON:表示该入口队列处于非拥塞状态。在该状态下,上游端口允许向该队列发送对应类别的流量,此时接收到的报文将被存放到私有缓存或共享缓存分区中。

2)OFF:表示该入口队列处于拥塞状态。在该状态下,上游端口对应类别的流量被暂停发送,此时接收到的报文将被存放于净空缓存分区中。

在没有发生拥塞时,入口队列处于ON状态,当该队列的共享缓存占用超过阈值$X_{off}(t)$时,即$\omega_s^{i,j}(t) \geqslant X_{off}$,该队列转换到OFF状态,状态转换的同时需要向上游端口发送一个PFC暂停帧以暂停对应类别流量的发送。当拥塞消除且共享缓存占用降低到阈值$X_{on}(t)$以下时,即$\omega_s^{i,j}(t) \leqslant X_{on}$,则该入口队列转换到ON状态,同时向上游端口发送对应队列的PFC恢复帧以恢复被暂停流量的传输。

出口队列的具体工作流程如\ref{c3:s1:ss2:fig:sub2:egress}所示,除状态转换中的状态与动作以外与入口队列类似,不再展开赘述。

\xsubsection{MMU处理流程}{MMU Workflow}

具体地,SIH中MMU接收到新到达的报文时,对于当前的缓存占用情况可能存在四种不同的处理方式:

1)$q^{i,j}(t)<\phi$:MMU决策该报文存入私有缓存空间。$\phi$表示给每个队列预留的私有缓存空间大小。

2)$\phi \leqslant q^{i,j}(t)< \phi+T(t)$:MMU决策该报文存入共享缓存空间。进一步地,如果入口队列当前处于OFF状态且$\omega_s^{i,j}(t) \leqslant X_{on}$,MMU向上游设备发送该队列的PFC恢复帧同时将队列状态转换为ON状态。

3)$\phi+T(t) \leqslant q^{i,j}(t)<\phi +T(t)+\eta$:MMU决策该报文存入净空缓存空间。进一步地,如果入口队列当前处于ON状态,MMU向上游设备发送该队列的PFC暂停帧同时将该队列状态转换为OFF状态。

4)$q^{i,j}(t) \geqslant \phi +T(t)+\eta$:净空缓存溢出,MMU决策丢弃该报文。


\xsection{问题分析}{Problem Analysis}
\label{c3:s2:problem analysic}

本节提出现有净空缓存管理机制存在的净空缓存所占比重过大和缓存分配低效性问题,结合数据中心网络发展趋势分析上述问题存在的根源以及寻求更高效缓存分配策略的必要性。

\xsubsection{净空缓存所占比重过大}{Headroom Occupies Considerable Memory}

理想情况下大部分缓存空间应该作为共享缓存使用,从而可以在不触发PFC的情况下吸纳更多的突发流量。然而,现有的缓存分配机制预留大量缓存空间作为净空缓存,严重挤压正常流量可以使用的缓存空间,进而导致频繁的PFC触发。

具体地,现有缓存分配机制独立地为每个入口队列保留一个静态大小的净空缓存空间。假设每个入口队列需要的净空缓存大小为$\eta$,则净空缓存预留总量为:
\begin{equation}
  h = N_p \cdot N_q \cdot \eta
  \label{eqn:c3:total headroom}
\end{equation}
\noindent 其中$N_p$为入端口的数量,$N_q$为每个端口中的队列数。

在该分配方式下,MMU为每个入口队列预留最坏情况下的净空缓存需求量$\eta$,静态净空缓存占据大量缓存空间。以Broadcom Trident2交换芯片为例,其缓存总量为12MB,共有32个40GbE端口(即$N_p=32$,$C=40Gbps$)。对于每个端口,PFC标准支持8个队列(即$N_q=8$),现假设$L_{MTU}=\text{1500B}$,$D_{prop}=1.5μs$,则MMU需要预留总量约5.33MB的缓存空间作为净空缓存,占总缓存容量的比重为44.4\%。

随着数据中心网络链路带宽的不断增加,上述情况将会更加严重。在过去的十年时间里,数据中心网络的链路带宽已经从1Gbps增长到40Gbps再到100Gbps\cite{SIGCOMM19HPCC,SIGCOMM15Jupiter},而且还在持续增长中。为了避免丢包,MMU需要预留更多净空缓存空间。然而,受限于芯片面积和开销,缓存容量无法以同等速度增加\cite{INFOCOM20BCC,ICNP21FlashPass,NSDI22BFC}。因此,净空缓存在缓存空间中占据的比重不断增加,严重挤压共享缓存空间。图\ref{c3:s2:ss1:fig:broadcom asic buffer size}显示了Broadcom厂商交换芯片缓存容量和净空缓存所占比重的发展趋势,在过去的十年里,缓存大小和交换带宽的比值的减少超过4倍,净空缓存所占比重增加56\%。

\begin{figure}[H]
  \centering
  \includegraphics[width=0.9\textwidth]{buffer_size.pdf}
  \caption{Broadcom交换芯片缓存发展趋势}
  \label{c3:s2:ss1:fig:broadcom asic buffer size}
\end{figure}

共享缓存空间不足会造成PFC频繁触发,进而导致严重的性能损害,如头阻、拥塞传播和附带损害,严重情况下甚至可能导致网络死锁。本节通过在大规模网络仿真环境验证缓存不足导致的网络性能损害,基于ns-3搭建Spine-Leaf拓扑结构测试环境,其中Spine层16台交换机和16台Leaf层交换机全连接,每个Leaf层交换机连接16台主机。测试环境中拥塞控制算法采用PowerTCP\cite{NSDI22PowerTCP},流大小通过web \ search负载\cite{SIGCOMM10DCTCP}模拟数据中心真实流量,流开始时间服从泊松分布,将网络负载配置为90\%。仿真结果如图\ref{c3:s2:ss1:fig:buffer size effect}所示,平均流完成时间(Flow Completion Time, FCT)随缓存容量减小而不断增大。14MB容量缓存下的平均FCT相对于30MB时增加78.1\%。

\begin{figure}[H]
  \centering
  \includegraphics[width=0.6\textwidth]{buffer_size_effect.pdf}
  \caption{缓存大小对流完成时间的影响}
  \label{c3:s2:ss1:fig:buffer size effect}
\end{figure}

网络运营商通常通过限制优先级队列数量缓解上述问题\cite{SIGCOMM16RDMA},然而,限制队列数量会损害不同网络服务之间的隔离性,导致拥塞更容易传播到整个网络,从而加重头阻问题。同时大量研究\cite{SIGCOMM18Homa,NSDI15PIAS,grosvenor2015queues,NSDI18AFQ,SIGCOMM18AuTO}表明,多服务队列可以极大改善网络性能。因此,限制队列数量无法从根本上解决该问题,而且会导致网络性能损害。

\xsubsection{净空缓存分配低效性}{Current Headroom Allocation Scheme is Inefficient}

在净空缓存所占比重不断增大的发展趋势下,现有的静态隔离净空缓存分配(Static and Isolated Headroom Allocation,SIH)机制具有显著的低效性,主要存在以下三个方面的原因:

\subsubsection{不是所有队列都需要净空缓存}

一个入口队列只有在其发生拥塞(即队列长度超过阈值$X_{off}$)时才需要去占用净空缓存。在现实的网络环境中,所有队列同时发生拥塞是几乎不可能的\cite{bai2023empowering}。然而,SIH为每个入口队列都静态地预留了一个最坏情况下的净空缓存需求量。因此,大部分净空缓存在大多数时间都是没有被利用的。

\subsubsection{同端口不同队列共享上行链路带宽}

同端口中的所有入口队列连接到同一个上行链路,所以这些队列中的流量天然共享上行链路带宽。当一个端口中某个类别的流量需要被PFC暂停时,如果其它类别也有流量到达,则理论上该类别流量的到达速度应当小于链路带宽$C$。因此,被暂停队列实际需要的净空缓存量通常小于最坏情况下的需求大小$\eta$。

\subsubsection{上游设备并不总以线速度打流}

在为入口队列分配净空缓存时,SIH假设在暂停帧生效前上游设备总是以线速度发送流量。然而,上游设备因为队列为空而断流。因此,上游设备的发送速度可能小于链路带宽,此时净空缓存存在超量分配。

\begin{figure}[H]
  \centering
  \includegraphics[width=0.9\linewidth]{headroom_utilization.pdf}
  \caption{净空缓存利用率分布}
  \label{c3:s2:ss1:fig:headroom utilization}
\end{figure}

在大规模ns-3网络仿真环境中验证SIH的低效性,实验环境除拥塞控制算法外同上文配置,拥塞控制算法采用DCQCN\cite{SIGCOMM15DCQCN}。仿真结果如图\ref{c3:s2:ss1:fig:headroom utilization}所示,净空缓存利用率的极大值分布显示,SIH的缓存利用率的中位数为4.96\%,99分位数为25.33\%,该结果表明大部分净空缓存在大多数情况下没有被有效利用。

SIH的低效性是固有的且无法通过简单地调整净空缓存大小来避免。其原因在于SIH需要按照最坏情况预留净空缓存空间,以保证在任何情况下都不会发生丢包。因此需要寻求一个新的缓存分配机制从根本上提高净空缓存的利用效率。

\xsection{DSH机制设计}{DSH Design}
\label{c3:s3:dsh design}

相对于SIH静态隔离的净空缓存分配方式,本文提出了一种动态共享净空缓存分配机制DSH,DSH可以在保证无丢包的基础上高效分配净空缓存。本节主要阐述DSH的基本思想、设计细节及其在交换芯片上的易实现性。

\xsubsection{设计目标}{Design Goals}

考虑到SIH分配方式固有的低效性,简单地减少净空缓存静态预留量会给无损网络带来丢包风险。因此,需要在保证无丢包的基础上设计一个更高效的净空缓存分配和管理机制,该机制需要同时满足以下特性:

\subsubsection{保证无损传输}

RDMA需要无损网络环境保证其性能,丢包引起的重传会导致RDMA性能大幅下降,造成吞吐率降低、完成时间增加,错过应用程序截止时间等问题。因此,避免因缓存溢出而丢包是无损网络对于缓存管理的基本要求,减少净空缓存预留量不能以引入丢包风险为代价。

\subsubsection{资源高效利用}

容量受限的片上缓存空间对于高带宽交换芯片是稀缺资源。实现无损传输不可避免地需要预留一部分缓存空间作为净空缓存,高效的缓存分配和管理机制应该尽可能保证大部分缓存空间在大多数时间内可用,而不应该空闲大量缓存空间仅为应对极少情况下发生的流量场景。

\subsubsection{足够突发容忍}

突发吸纳是缓存最核心的作用之一。突发流量在缓存中的主要表现为短时间内的队列迅速积累,而端到端拥塞控制机制至少需要一个RTT的时间才能做出拥塞响应,如果没有足够缓存空间容纳积累的突发流量则需要触发PFC来避免丢包。在常见的Incast场景下,PFC需要同时暂停多个上游设备,从而更容易导致拥塞传播。因此,缓存管理机制应该提供足够的的突发吸纳能力,以应对数据中心网络中常见的突发流量。

\subsubsection{避免PFC触发}

PFC本身会对网络性能造成一定的损害,其基于队列级别的粗粒度流量控制可能导致头阻问题,造成某条流的性能受损,在大规模网络中还可能存在PFC传播甚至PFC死锁等问题,频繁触发PFC会损害网络中流量的吞吐率和时延等性能。因此,尽可能地避免PFC触发是缓存管理的一个重要目标。

\xsubsection{基本思想}{Key Ideas}
\label{c3:s3:ss2:key ideas}

DSH在设计上主要遵循动态共享的基本思想,在保证无丢包的基础上实现高效净空缓存分配:

\subsubsection{DSH主动预留少量缓存作为保底净空缓存作为无损传输保证。}

同一端口下的不同入口队列共享其上行链路带宽,所以只需要为每个端口预留$\eta$大小的净空缓存即可实现无丢包,因此,为每个入口队列都分配$\eta$大小的净空缓存的方式是没有必要的。当一个端口开始占用保底净空缓存时,整个上游端口中所有队列都应该被PFC暂停发送,以此减少净空缓存的静态预留量。暂停整个端口可能损害不同类别流量之间的隔离性,所以需要额外的机制保证其非必要不触发。

\subsubsection{DSH根据队列拥塞状态动态分配净空缓存,使其在队列之间共享。}

当且仅当一个队列发生拥塞时才需要占用净空缓存空间。因此,DSH仅在一个队列拥塞时开始尝试为其分配净空缓存,而不会为非拥塞队列分配,从而可以在少量队列拥塞的情况下释放大量缓存空间用于吸纳突发流量。此外,当某个队列分配到的净空缓存没有被完全占用的情况下,DSH可以使其剩余空间为其它队列所共享,通过统计复用的方式提高净空缓存利用效率。

\xsubsection{主要挑战}{Main Challenges}

实现\ref{c3:s3:ss2:key ideas}节提出的基本思想首先需要解决其引入的一些问题。一方面,为了在保证无损传输的基础上减少净空缓存静态预留量,DSH需要在队列级别流量控制的基础上引入端口级别流量控制。端口粒度的暂停会降低同一端口中不同队列之间的隔离性;另一方面,以动态方式分配净空缓存及其在不同队列之间的共享可能导致不公平甚至饥饿问题。因此,DSH在具体设计时主要面临以下三个挑战:

\subsubsection{如何尽可能避免触发端口级别流量控制}

理想情况下,端口级别流控应该仅作为队列级别流控的保底措施。绝大多数情况的网络环境下,一个端口中同时只有一个或者少数几个队列处于拥塞状态,所有队列或者大多数队列同时拥塞的情况极少发生。基于以上认识,对于端口中少数队列拥塞的情况,DSH需要实现仅触发队列级别流量控制避免丢包而无需触发端口级别流量控制;对于大多数队列拥塞的极端情况,DSH可以在必要时触发端口级别流量控制,此时暂停整个端口相对于暂停所有拥塞队列差别不大。

\subsubsection{如何保证队列暂停后的净空缓存占用}

DSH不再为每个队列静态预留净空缓存空间,而是仅为每个端口预留用于端口级别流量控制。这意味着在触发队列级别流量控制后可能无法保证足够的净空缓存分配,如果净空缓存分配不足将会进一步触发端口级别流量控制。不合理的分配方式可能会增加端口级别流量控制触发的概率,导致队列之间的隔离性受损。DSH尝试在共享缓存分配足够的净空缓存,首先保证在触发队列暂停时刻该队列仍有足够的可用共享缓存空间,在暂停帧生效之前如果该队列的共享缓存空间无法满足净空缓存需求则进一步尝试从其它队列空闲共享缓存中共享,尽可能保证队列的净空缓存分配。

\subsubsection{如何限制每个端口的共享缓存空间}

限制每个端口的共享缓存空间即确定端口级别流控的触发条件。通过动态方式限制端口的共享净空缓存占用需要做到充分但不过度:一方面,分配空间过小容易导致端口级别流控的频繁触发;另一方面,分配空间过大可能造成其它非拥塞端口的缓存饥饿。DSH需要在设计时权衡避免端口级别流量控制和不同端口之间的公平性。考虑到上述要求与队列之间的共享缓存分配要求基本一致,DSH通过基于端口缓存占用的动态阈值限制每个端口的共享缓存空间,分配和管理逻辑与队列保持统一,同时将动态阈值的公平性继承到端口级别。

\xsubsection{具体机制}{DSH Mechanisms}
\label{c3:s3:ss4:dsh mechanisms}

本节通过设计DSH的具体机制实现\ref{c3:s3:ss2:key ideas}节的基本思想,包括缓存分区结构、缓存分配与管理、流量控制以及MMU处理流程。

\subsubsection{缓存结构}

\begin{figure}[H]
  \centering
  \includegraphics[width=0.7\textwidth]{buffer_partition_dsh.pdf}
  \caption{DSH缓存分区}
  \label{c3:s3:ss4:fig:dsh buffer partition}
\end{figure}


DSH的缓存组织结构如图\ref{c3:s3:ss4:fig:dsh buffer partition}所示,在传统缓存分区的基础上,DSH进一步将净空缓存划分为两部分:共享净空缓存和保底净空缓存。保底净空缓存为每个端口静态预留,作为无丢包保证;共享净空缓存则是在队列需要时动态分配的,同时可以在不同入口队列之间共享。

DSH没有将共享净空缓存单独分区,而是将其作为共享缓存的一部分。这样设计有两个好处:一是可以简化DSH在交换芯片中的实现,缓存分区可以和现有交换芯片保持一致;二是可以提高缓存利用率,从共享缓存分区分配共享净空缓存可以提高统计复用的程度,从而提高缓存可用率。

\subsubsection{缓存分配}

DSH对于私有缓存分区的管理方式与SIH相同。对于保底净空缓存,DSH同样以静态预留的方式为每个端口分配,假设有$N_p$个端口,需要的保底 净空缓存的大小为:
\begin{equation}
  B_i = N_p \cdot \eta
  \label{eqn:c3:total reserved headroom in dsh}
\end{equation}

\noindent 其中$\eta$由公式(\ref{eqn:c2:headroom calculation})计算得到。

私有缓存和保底净空缓存之外的缓存空间作为共享缓存,共享缓存在入口队列之间动态分配。DSH使用阈值$T(t)$限制每个入口队列的共享缓存占用(包括净空缓存和非净空缓存)。与DT相同,DSH同样根据剩余缓存大小动态调整阈值,从而保证在易于实现前提下实现自适应性和高效性。

DSH中的$T(t)$与现有交换芯片中的计算方式相同,即公式(\ref{eqn:c3:dt threshold})所示,唯一的区别在于共享缓存占用的总量
$ω_s^{i,j}(t)$为净空缓存和非净空缓存占用的总和。因此,DSH的阈值计算易于实现,不需要对当前的硬件逻辑进行修改。

\subsubsection{流量控制}

DSH通过结合两种级别的流量控制机制来保证无丢包:队列级别流量控制和端口级别流量控制。

队列级别流量控制和PFC机制类似,当一个入口队列的共享缓存占用总量超过$X_{qoff}$阈值时,MMU向上游端口发送对应队列的PFC暂停帧,上游设备收到该暂停帧后将暂停发送该类别的流量。存在的区别在于$X_{qoff}$阈值的设置,DSH中将其设置为:
\begin{equation}
  X_{qoff}(t) = T(t) - \eta
  \label{eqn:c3:total reserved headroom in dsh}
\end{equation}

其中的蕴含的原理包括两方面:一方面,当一个入口队列变得拥塞时,DSH首先尝试为其预留足够的净空缓存,即$\eta$大小);另一方面,当一个入口队列拥塞程度下降时,未使用的缓存可以被其它拥塞队列占用。具体地,当一个入口队列的共享缓存占用小于阈值$T(t)$时,未占用部分空间可以作为空闲缓存空间用于计算$T(t)$,$T(t)$增大即表示其它拥塞队列可以占用更多的缓存空间。综上,DSH只有在一个队列真正拥塞时才为其预留足够的净空缓存。

由于共享净空缓存是以动态方式分配而不是静态预留的,DSH无法保证每个队列发生拥塞时都能分配到$\eta$大小的净空缓存。因此,DSH引入端口级别流量控制保证在任何情况下都不会发生丢包。

端口级别流量控制会在一个端口中所有队列的共享缓存占用超过$X_{poff}$阈值时触发。当端口级别流量控制触发时,MMU会向上游端口发送一个端口级别暂停帧(即所有优先级计时器均被设置的PFC暂停帧)。上游端口接收到该暂停帧后暂停向下游端口发送所有优先级类别的流量。
$X_{poff}$由下式计算得到:
\begin{equation}
  X_{poff}(t) = N_q \times T(t)
  \label{eqn:c3:port pause threshold}
\end{equation}

DSH为每个入口队列分配$T(t)$大小的共享缓存,其中包括净空缓存和非净空缓存两部分。一个端口下所有入口队列分配到的共享缓存总和即为$N_q \times T(t)$。其中蕴含的原理即为DSH允许一个入端口中的所有入口队列共享其分配到的所有缓存空间。具体地,通过限制端口级别的缓存占用可以使一个拥塞队列占用同一个端口中其它队列的净空缓存空间。由于去往同端口不同队列的流量天然共享其上行链路带宽,端口级别的缓存共享可以同时保证高效性和公平性。综上,DSH可以在高效利用共享缓存的同时保证端口级别流量控制不易触发。

\subsubsection{MMU处理流程}

\begin{figure}[H]
  \begin{subfigure}[b]{0.49\linewidth}
      \centering
      \includegraphics[height=5.3cm]{state_transition_pfc_in_queue.pdf}
      \subcaption{队列级别流量控制}
      \label{c3:s3:ss4:fig:sub1:dsh ingress queue state transition}
  \end{subfigure}
  \begin{subfigure}[b]{0.49\linewidth}
      \centering
      \includegraphics[height=5.3cm]{state_transition_pfc_in_port.pdf}
      \subcaption{端口级别流量控制}
      \label{c3:s3:ss4:fig:sub2:dsh ingress port state transition}
  \end{subfigure}
  \caption{DSH入口侧流量控制状态机}
  \label{c3:s3:ss4:fig:dsh ingress state transition}
\end{figure}

综合以上机制,DSH的具体工作流程可以用图\ref{c3:s3:ss4:fig:dsh ingress state transition}中的状态机描述。在入口侧,每个入口队列存在两种队列状态:

(1)QON:表示该入口队列处于非拥塞状态。在该状态下,上游端口允许向该队列发送对应类别的流量,收到的数据包将被存放到私有缓存空间或共享缓存空间。

(2)QOFF:表示该入口队列处于拥塞状态。在该状态下,上游端口对应类别的流量被暂停发送,收到的数据包将被存放到共享净空缓存空间。

在没有发生拥塞时,入口队列处于QON状态,当该队列的共享缓存占用超过阈值$X_{qoff}(t)$时,该队列转换到QOFF状态,状态转换的同时需要向上游端口发送对应队列级别的暂停帧以暂停相应类别流量的发送。如果拥塞消除并且共享缓存占用降低到阈值$X_{qon}(t)$以下,则该入口队列转换到QON状态,同时向上游端口发送对应队列级别恢复帧以恢复被暂停流量的传输。

此外,入端口同样存在两种端口级别的状态:

\setcounter{paragraph}{0}
(1)PON:表示该入端口处于非拥塞状态。在该状态下,上游端口允许向其发送任意类别的流量(对应类别没有被队列级别流量控制暂停前提下),此时收到的数据包将会被存放到私有缓存或者共享缓存分区中。

(2)POFF:表示该入端口处于拥塞状态。在该状态下,上游端口所有类别的流量均被暂停发送,此时收到的数据包将会被存放到保底 净空缓存中。

在没有发生拥塞时,端口处于PON状态,当该入端口的缓存占用超过阈值$X_{poff}(t)$时,其状态转换为POFF,同时向上游端口发送一个端口级别暂停帧以暂停所有类别流量的发送。当拥塞消除并且该端口的缓存占用降低到$X_{pon}(t)$以下时,该端口转换到PON状态,同时向上游端口发送一个端口级别恢复帧恢复流量发送。

\begin{figure}[H]
  \begin{subfigure}[b]{0.49\linewidth}
      \centering
      \includegraphics[height=5.3cm]{state_transition_pfc_out_queue.pdf}
      \subcaption{队列级别流量控制}
      \label{c3:s3:ss4:fig:sub1:dsh egress queue state transition}
  \end{subfigure}
  \begin{subfigure}[b]{0.49\linewidth}
      \centering
      \includegraphics[height=5.3cm]{state_transition_pfc_out_port.pdf}
      \subcaption{端口级别流量控制}
      \label{c3:s3:ss4:fig:sub2:dsh egress port state transition}
  \end{subfigure}
  \caption{DSH出口侧流量控制状态机}
  \label{c3:s3:ss4:fig:dsh egress flow control}
\end{figure}


出口侧的具体工作流程如图\ref{c3:s3:ss4:fig:dsh egress flow control}所示,除状态转换中的状态与动作以外与入口侧类似,此处不再展开赘述。

\xsection{DSH原理分析}{DSH Analysis}
\label{c3:s4:dsh analysis}

本节在理论上分析DSH的突发流量吸纳能力,通过理论证明DSH相对于SIH的性能优势。

\begin{figure}[H]
  \begin{subfigure}[b]{0.49\linewidth}
      \centering
      \includegraphics[height=5.3cm]{qlen_evolution_1.pdf}
      \subcaption{$R \leqslant \frac{1 - \alpha N}{\alpha M} + 1$}
      \label{c3:s4:ss2:fig:sub1:dsh qlen evaluation 1}
  \end{subfigure}
  \begin{subfigure}[b]{0.49\linewidth}
      \centering
      \includegraphics[height=5.3cm]{qlen_evolution_2.pdf}
      \subcaption{$R > \frac{1 - \alpha N}{\alpha M} + 1$}
      \label{c3:s4:ss2:fig:sub2:dsh qlen evaluation 2}
  \end{subfigure}
  \caption{队列长度和阈值变化过程}
  \label{c3:s4:ss2:fig:dsh qlen evaluation}
\end{figure}

\xsubsection{假设条件}{Assumptions}

考虑如下场景\cite{TON98DT}:$t_0$时刻($t_0 \ll 0$),交换机中$N$个入口队列$Q_0 \sim Q_{N-1}$变为拥塞状态。在$t=0$时,另外$M$个空入口队列$Q_N \sim Q_{N+M-1}$同时开始传输突发流量。上述$N+M$个队列中的流量以平滑负载大小为$R$的速度到达,其中$R$为相对流量离开速度标准化后的值且$R>1$。为了简化分析过程,现做出以下假设:

1)缓存中没有预留私有缓存空间,即$B_p=0$。

2)PFC暂停帧生效的时延无限接近于0。

3)PFC恢复阈值无限接近但小于PFC暂停阈值的大小,即对于任意$\varepsilon > 0$,$0<X_{qon} - X_{qoff}<\varepsilon$。

基于以上假设,DSH和SIH的突发吸纳能力可以分别由定理\ref{thm:c3:burst absorption for DSH}和定理\ref{thm:c3:burst absorption for SIH}表征。

\xsubsection{DSH突发吸纳能力分析}{Analysis of DSH Burst Absorption Ability}

\begin{theorem}
  对于入口队列$Q_N \sim Q_{N+M-1}$,DSH当且仅当满足下式时可以避免触发PFC:
  \label{thm:c3:burst absorption for DSH}
  \begin{equation}
    d < 
    \begin{cases}
      \frac{\alpha(B-N_p\cdot \eta)-\eta}{[1+\alpha(N+M)](R-1)}, & R\leqslant\frac{1-\alpha N}{\alpha M}+1 \\
      \frac{\alpha(B-N_p\cdot \eta)-\eta}{(1+\alpha N)[(1+\alpha M)(R-1)-\alpha N]}, & R>\frac{1-\alpha N}{\alpha M}+1 \\
    \end{cases}
    \label{eqn:c3:port pause threshold}
  \end{equation}
  
  \noindent{其中$d$表示突发流量的持续时间。}  
\end{theorem}

\begin{proof}
  在$t=0$时刻,入口队列$Q_0 \sim Q_{N-1}$的队列长度$q_0(t) \sim q_{N-1}(t)$等于$X_{qoff}$阈值的大小,即:
  \begin{equation}
    q_i(0)=X_{qoff}(0)=T(0)-\eta, \quad 0 \leqslant i < N
    \label{eqn:c3:length of queue i at time 0}
  \end{equation}

  其中阈值$T(t)$由下式计算得到:
  \begin{equation}
    T(t)=\alpha \cdot (B_s - \sum_{i}q_i(t))
    \label{eqn:c3:threshold of pfc pause}
  \end{equation} 
  
  将公式(\ref{eqn:c3:length of queue i at time 0})代入公式(\ref{eqn:c3:threshold of pfc pause})得到:
  \begin{equation}
    \begin{cases}
      T(0)=\frac{\alpha(B_s+N \eta)}{1+\alpha N} \\
      q_i(0)=T(0)-\eta=\frac{\alpha B_s-\eta}{1+\alpha N} \\
    \end{cases}
    \label{eqn:c3:substitute result}
  \end{equation}

  在$t=0^+$时,$M$个入口队列开始活跃。这些队列的队列长度开始增大,同时$T(t)$开始减小。相应地,$X_{qoff}$阈值也会随之减小,进而导致$q_0(t) \sim q_{N-1}(t)$减小。令$T'(t)$表示$T(t)$的导数,$q'_i(t)$表示$q_i(t)$的导数,可以得到下式:
  \begin{equation}
    T'(0^+)=-\alpha \cdot \sum_{i}q'_i(0^+)
    \label{eqn:c3:derivative of T}
  \end{equation}
  \begin{equation}
    q'_i(0^+)=
      \begin{cases}
        \max (T'(0),-1), & 0 \leqslant i < N \\
        R-1, & N \leqslant i < N+M \\
      \end{cases}
    \label{eqn:c3:derivative of queue length}
  \end{equation}  

  将公式(\ref{eqn:c3:derivative of queue length})代入公式(\ref{eqn:c3:derivative of T})可得:
  \begin{equation}
    T'(0^+)=-\alpha \cdot N \cdot \max(T'(0),-1)-\alpha \cdot M
    \cdot (R-1)
  \end{equation}

  此时可能出现两种情况:

  \setcounter{subsubsection}{0}
  \subsubsection{$R \leqslant \frac{1-\alpha N}{\alpha M} + 1$}

  此时,$T'(0^+) \geqslant -1$,因此,$q_i(t)$($0 \leqslant i < N$)与$X_{qoff}$阈值以同等速率减小。队列长度变化曲线如图\ref{c3:s4:ss2:fig:sub1:dsh qlen evaluation 1}所示,$q_i(t)$($N \leqslant i < N + M - 1$)将会保持继续增长。在这段时间内,$T(t)$和$q_i(t)$可由下式计算得到:
  \begin{equation}
    T(t)=\frac{\alpha}{1+\alpha N}[B_s+N\eta-M(R-1)t]
  \end{equation}
  \begin{equation}
    q_i(t)=
    \begin{cases}
      \frac{\alpha B_s-\eta-\alpha M(R-1)t}{1+\alpha N}, 
      & 0 \leqslant i < N \\
      (R-1)t, & N \leqslant i < N + M -1 \\
    \end{cases}
  \end{equation}  

  直到$t=t_1$时,$q_i(t)$($N \leqslant i < N + M - 1$)增长到阈值$X_{qoff}$的大小,即$q_i(t_1)=T(t_1)-\eta$,从中解出$t_1$,可得:
  \begin{equation}
    t_1=\frac{\alpha B_s - \eta}{[1+\alpha (N+M)](R-1)}
  \end{equation}

  在$t=t_1$时刻,队列$Q_N \sim Q_{N+M-1}$开始暂停上游设备。因此,当且仅当$d<t_1$时DSH可以避免触发PFC暂停帧,其中$d$表示队列$Q_N \sim Q_{N+M-1}$中突发流量的持续时间。

  \subsubsection{$R>\frac{1-\alpha N}{\alpha M}+1$}

  在这种情况下,$T'(0^+)<-1$。因此,$q_i(t)(0 \leqslant i < N)$的减小速度小于阈值$X_{qoff}$减小速度。队列长度的变化曲线如图\ref{c3:s4:ss2:fig:sub2:dsh qlen evaluation 2}所示。在$[0,t_2]$时间内,$T(t)$和$q_i(t)$可以由下式得到:
  \begin{equation}
    T(t)=\frac{\alpha(B_s+N\eta)}{1+\alpha N}-[\alpha M(R-1)-\alpha N]t
  \end{equation}
  \begin{equation}
    q_i(t)=
    \begin{cases}
      \frac{\alpha B_s-\eta}{1+\alpha N}-t, 
      & 0 \leqslant i < N \\
      (R-1)t, & N \leqslant i < N + M -1 \\
    \end{cases}
  \end{equation}

  直到$t=t_2$时刻,$q_i(t)$($N \leqslant i < N + M - 1$)增长到$X_{qoff}$阈值,即$q_i(t_2)=T(t_2)-\eta$。从中解出$t_2$,可得:
  \begin{equation}
    t_2=\frac{\alpha B_s - \eta}{(1+\alpha N)[(1+\alpha M)(R-1)-\alpha N]}
  \end{equation}

  在$t=t_2$时刻,队列$Q_N \sim Q_{N+M-1}$开始暂停上游设备。因此,当且仅当$d<t_2$时DSH可以避免触发PFC暂停帧,其中$d$表示队列$Q_N \sim Q_{N+M-1}$中突发流量的持续时间。

  证毕。
\end{proof}

\xsubsection{SIH突发吸纳能力分析}{Analysis of SIH Burst Absorption Ability}

\begin{theorem}
  对于入口队列$Q_N \sim Q_{N+M-1}$,SIH当且仅当满足下式时可以避免触发PFC:
  \label{thm:c3:burst absorption for SIH}  
  \begin{equation}
    d < 
    \begin{cases}
      \frac{\alpha(B-N_p\cdot \eta)}{[1+\alpha(N+M)](R-1)}, & R\leqslant\frac{1-\alpha N}{\alpha M}+1 \\
      \frac{\alpha(B-N_p\cdot \eta)}{(1+\alpha N)[(1+\alpha M)(R-1)-\alpha N]}, & R>\frac{1-\alpha N}{\alpha M}+1 \\
    \end{cases}
    \label{eqn:c3:sih burst obsorption}
  \end{equation}
  
  \noindent{其中$d$表示突发流量的持续时间。}
\end{theorem}

\begin{proof}
  SIH机制下的队列长度变化过程与图\ref{c3:s4:ss2:fig:dsh qlen evaluation}相同,差别仅SIH的$X_{off}$阈值计算,即$X_{off} = T(t)$。因此,可以直接令$\eta = 0$解出$t_1$和$t_2$,结果如下式:
  \begin{equation}
    \begin{cases}
      t_1=\frac{\alpha B_s}{[1+\alpha (N+M)](R-1)} \\
      t_2=\frac{\alpha B_s}{(1+\alpha N)[(1+\alpha M)(R-1)-\alpha N]} \\
    \end{cases}  
    \label{c3:s4:ss2:eqn:t1 and t2 in sih}
  \end{equation}

  因此,当且仅当突发流量持续时间$d$满足下式时SIH可以避免触发PFC:
  \begin{equation}
    d < 
    \begin{cases}
      \frac{\alpha B_s}{[1+\alpha(N+M)](R-1)}, & R\leqslant\frac{1-\alpha N}{\alpha M}+1 \\
      \frac{\alpha B_s}{(1+\alpha N)[(1+\alpha M)(R-1)-\alpha N]}, & R>\frac{1-\alpha N}{\alpha M}+1 \\
    \end{cases}
    \label{eqn:c3:sih:port pause threshold}
  \end{equation}

  证毕。
\end{proof}

\xsubsection{总结}{Summary}

定理\ref{thm:c3:burst absorption for DSH}和定理\ref{thm:c3:burst absorption for SIH}证明:相对于SIH,DSH对于队列数量具有更好的扩展性。具体地,DSH的突发吸纳能力和每个端口的队列数量(即$N_q$)不相关,而SIH的突发吸纳能力和每个端口的队列数量呈负相关关系。这表明DSH可以支持尽可能多的队列数,从而可以提高其在不同网络服务之间的隔离性,同时可以支持部署更多先进的流量优化系统(如PIAS\cite{NSDI15PIAS}和Homa\cite{SIGCOMM18Homa}等)。


\xsection{DSH具体实现}{DSH Implementation}
\label{c3:s5:dsh implementation}

本节讨论DSH在现行交换芯片上的可行性以及其算法实现。DSH不需要对现行缓存结构进行任何修改,只需要对流量控制机制进行适度修改即可实现。

\xsubsection{队列级别流量控制}{Queue-level Flow Control}

如\ref{c3:s1:current buffer management scheme}节所述,现行缓存管理机制在队列长度增长到$T(t)$时触发PFC暂停帧,在队列长度降低到$T(t)-\delta$时触发PFC恢复帧,其中$\delta$为一个可配置参数。DSH中的唯一区别在于PFC触发暂停帧和恢复帧的阈值大小分别为$T(t)-\eta$和$T(t)-\eta-\delta_q$,因此只需要修改PFC控制帧的触发状态即可。具体地,交换芯片需要增加两个减法器,一个以$T(t)$和$\eta$作为输入用于计算阈值$X_{qoff}$,另一个以$X_{qoff}$和$δ_q$作为输入用于计算$X_{qon}$。

\xsubsection{端口级别流量控制}{Port-level Flow Control}

目前,市场上已经有很多交换芯片基于端口的缓存占用端口级别流量控制进行了支持\cite{BCM88800TM,CiscoNexus9300IB},因此DSH同样只需要修改端口级别控制帧的触发状态,其中暂停帧触发阈值为$N_q \times T(t)$,需要为其增加一个乘法器,通常情况下$N_q$的大小为2的幂值,这时只需要一个移位寄存器即可;恢复帧触发阈值大小为$X_{poff}-\delta_p$,$\delta_p$是一个可配置参数,因此需要引入一个减法器。

\xsubsection{两种流量控制整合}{Consolidating Two Kinds of Flow Controls}

对于队列级别和端口级别流量控制的整合,下游入口侧两种流量控制可以独立地工作,所以二者的整合不需要额外的修改;上游出口侧只需要修改暂停传输的状态即可完成整合,具体地,当一个出口队列处于QOFF状态或者其所在出端口处于POFF状态时都会暂停流量传输。因此交换芯片需要为每个出口队列维护一个状态,同时为每个出端口维护一个状态,这样每个队列的暂停状态通过一个或门即可实现。

\xsubsection{交换芯片资源增量}{Overall Resource Increments to Switch ASIC}

基于以上分析,DSH在交换芯片上引入的资源开销是可以接受的,具体原因包括以下几个方面:

1)DSH不需要额外的寄存器。每个队列和端口的缓存占用信息可以在现有的交换芯片中直接获得。

2)DSH不会触及内存分配和管理机制。DSH的缓存分区结构(图\ref{c3:s3:ss4:fig:dsh buffer partition})与现有交换芯片(图\ref{fig:c2:buffer partition})保持一致,而且DSH在没有改变非净空缓存部分的缓存分配和管理机制的基础上增加额外的流控机制即可实现对净空缓存的分配。因此,DSH的实现不需要修改现有的内存分配和管理机制。

3)DSH只需要一些简单易实现的比较和算数运算操作即可实现PFC触发的状态维护。DSH维护的所有状态都是基于缓存占用和阈值比较,在此基础上DSH只需要引入一些额外的比较和简单算数运算操作。

\xsubsection{详细算法实现}{DSH Algorithm}
DSH在算法实现上主要包括两个模块:入队模块和出队模块。入队模块负责报文入队之前的决策以及流量控制暂停帧触发,出队模块负责报文出队后的缓存释放以及流量控制恢复帧触发。

\subsubsection{入队模块}
入队模块算法在报文入队前触发,首先对报文进行决策,根据缓存占用决定其能否进入缓存、缓存位置以及流量控制暂停帧触发,然后根据决策结果更新对应分区的缓存占用计数器。

\begin{algorithm}[H]
  \small
  \SetAlCapFnt{\small}
  \SetAlCapNameFnt{\small}
  \Input{待处理报文$packet$}
  \Output{存储位置决策$decision$}

  \SetAlgoVlined

  \Comment{获取报文的大小,入端口和队列}
  $pkt\_size \leftarrow packet.size$\;
  $pid \leftarrow \text{GetIngressPort}(packet)$\;
  $qid \leftarrow \text{GetQueueIndex}(packet)$\;
  \Comment{判断端口是否处于暂停状态}
  \If{$port\_status[pid]=OFF$}{
    \Comment{判断净空缓存是否溢出}
    \eIf{$headroom\_left[pid] < pkt\_size$}{
      $decision \leftarrow DROP$\;
    }{
      $decision \leftarrow TO\_HEADROOM$\;
    }
  }
  \Comment{判断私有缓存空间是否可以容纳}
  \ElseIf{$pkt\_size+ingress\_bytes[pid][qid]<\varphi$}{
    $decision \leftarrow TO\_RESERVED$\;
  }
  \Comment{判断是否需要触发端口级别流控}
  \ElseIf{$pkt\_size+port\_shared\_bytes[pid] \geqslant N_q \times T(t)$}{
    $port\_status[pid] \leftarrow OFF$\;
    $decision \leftarrow TO\_HEADROOM$\;
    向上游设备发送$pid$的端口暂停帧\;
  }
  \Comment{判断是否需要触发队列级别流控}
  \ElseIf{$pkt\_size+ingress\_bytes[pid][qid]\geqslant\varphi+T(t)-\eta$}{
    $decision \leftarrow TO\_SHARED$\;
    $queue\_status[pid][qid] \leftarrow OFF$\;
    向上游设备发送$qid$的队列暂停帧\;
  }
  \Else{
    $decision \leftarrow TO\_SHARED$
  }
  根据决策结果更新对应的计数器\;
  \Return $decision$\;
  \caption{MMU入队模块决策算法}
  \label{alg:c3:s5:ss5:enqueue module algorithm}
\end{algorithm}

算法\ref{alg:c3:s5:ss5:enqueue module algorithm}描述了DSH入队模块决策算法的伪代码。对于每个新到达报文首先获取其大小、入端口和入口队列(第2-4行),然后检查当前端口是否处于POFF状态,如果处于POFF状态且净空缓存空间足够则进入该端口的保底净空缓存空间(第11行),净空缓存溢出则丢弃当前报文(第9行),正常工作的情况下净空缓存溢出的情况不可能出现。如果当前端口处于PON状态则首先查看私有缓存空间是否可用,可用则优先占用私有缓存(第13行)。不可用则进一步检查共享缓存空间,可用共享缓存空间同时受端口级别和队列级别阈值限制,若端口共享缓存占用超过阈值则需要触发端口暂停帧同时决策占用净空缓存(第19-21行),端口可用则决策进入共享缓存空间,同时进一步判断队列级别限制,队列级别阈值控制队列级别流控的触发,如果超过$X_{qoff}$阈值则触发队列暂停帧(第26行)。最后根据决策结果更新缓存占用(第31行)。

\subsubsection{出队模块}
出队模块算法在报文发送出队之后触发,出队模块不会记录每个报文所在的缓存分区,而是根据缓存使用情况确定出队分区,更新对应计数器释放报文占用的缓存空间,同时决策流量控制恢复帧触发。

\begin{algorithm}[H]
  \small
  \SetAlCapFnt{\small}
  \SetAlCapNameFnt{\small}
  \Input{待处理报文$packet$}

  \SetAlgoVlined

  \Comment{获取报文的大小,入端口和队列}
  $pkt\_size \leftarrow packet.size$\;
  $pid \leftarrow \text{GetIngressPort}(packet)$\;
  $qid \leftarrow \text{GetQueueIndex}(packet)$\;
  \Comment{更新净空缓存占用计数器}
  $from\_headroom \leftarrow \min(headroom\_bytes[pid], pkt\_size)$\;
  $headroom\_bytes[pid] \leftarrow headroom\_bytes[pid] - from\_headroom$\;
  \Comment{更新共享缓存占用计数器}
  $from\_shared \leftarrow \min(shared\_bytes[pid][qid], pkt\_size-from\_headroom)$\;
  $shared\_bytes[pid][qid] \leftarrow shared\_bytes[pid][qid] - from\_shared$\;
  \Comment{更新私有缓存占用计数器}
  $from\_reserved \leftarrow \min(reserved\_bytes[pid][qid], pkt\_size-from\_headroom-from\_reserved)$\;
  $reserved\_bytes[pid][qid] \leftarrow reserved\_bytes[pid][qid] - from\_reserved$\;
  \Comment{更新入口队列缓存占用}
  $ingress\_bytes[pid][qid] \leftarrow ingress\_bytes[pid][qid] - from\_reserved - from\_shared$\;
  \Comment{检查端口级别流控恢复}
  \ForAll{$port\_status[p]=OFF$}{
    \If{$headroom\_bytes[p] \leqslant 0 \And port\_shared\_bytes[p] < N_q \times T(t) - \delta_p$}{
      $port\_status[p] \leftarrow ON$\;
      向上游设备发送端口$p$的恢复帧\;
    }
  }
  \Comment{检查队列级别流控恢复}
  \ForAll{$port\_status[p]=ON \And queue\_status[p][q]=OFF$}{
    \If{$shared\_bytes[p][q]<T(t)-\eta-\delta_q$}{
      $queue\_status[p][q] \leftarrow ON$\;
      向上游设备发送端口$p$队列$q$的恢复\;
    }
  }  

  \caption{MMU出队模块处理算法}
  \label{alg:c3:s5:ss5:dequeue module algorithm}
\end{algorithm}

算法\ref{alg:c3:s5:ss5:dequeue module algorithm}描述了DSH出队模块处理算法的伪代码。对于每个出队报文首先获取其大小、入端口和入口队列(第2-4行),MMU仅维护每个缓存分区的缓存占用的计数器,并不记录每个报文所在的缓存分区,出队模块进行的缓存释放操作为逻辑上的缓存释放,即更新对应分区的计数器。出队时缓存分区释放的顺序依次为:净空缓存$\rightarrow$共享缓存$\rightarrow$私有缓存,出队模块依次计算上述三个分区的出队字节数并更新对应的计数器(第6-7行,第9-10行,第12-13行)。第15行中$ingress\_bytes[pid][qid]$记录端口$pid$中队列$qid$的非净空缓存占用。报文出队后会导致队列长度减小或者暂停阈值增加,此时需要检测端口级别和队列级别的流控状态以便及时恢复发送。出队模块首先对每个处于暂停状态的端口进行检测,如果净空缓存排空且端口共享缓存占用低于$X_{pon}$阈值则向上游设备发送$pid$对应端口的恢复帧同时恢复PON状态(第17-20行);然后检查每个PON状态端口中的QOFF状态队列,如果共享缓存占用低于$X_{qon}$阈值则向上游设备发送$qid$对应队列的恢复帧。


\xsection{DSH性能测试}{DSH Evaluation}
\label{c3:s6:dsh evaluation}

本节通过在ns-3仿真平台\cite{ns-3}测试DSH的性能,实验结果总结如下,与现有机制SIH相比:

1)DSH在不触发PFC的前提下可以实现超过4倍大小的突发吸纳量。

2)DSH可以有效消除PFC带来的性能损害。

3)在大规模数据中心网络中,DSH最高可以将突发短流的FCT减少57.7\%,背景长流的FCT减少31.1\%。

\xsubsection{基本性能测试}{Basic Performance Evaluation}
\label{c3:s5:ss1:basic performance evaluation}

本节通过ns-3搭建小型测试环境模拟Broadcom Tomahawk交换芯片,该交换芯片具有32个100Gbps端口和16MB共享缓存空间,每个端口支持8个队列,其中最高优先级队列预留给ACK确认和控制帧,其余7个队列之间采用DWRR调度策略。网络拓扑中的链路传播时延为$2 \mu s$,可得$\eta=\text{56840B}$。所以,SIH中需要预留的净空缓存总量为$\text{56840B} \times 32 \times 7 = \text{12MB}$,私有缓存总大小为$\text{3KB} \times 32 \times 7 = \text{672KB}$。DT中的$\alpha$设置为1/16\cite{SIGCOMM16RDMA},$X_{qon} / X_{pon}$阈值设置与$X_{qoff} / X_{poff}$相同。

\begin{figure}[H]
  \begin{subfigure}[b]{0.49\linewidth}
      \centering
      \resizebox{0.95\linewidth}{!}{\begin{tikzpicture}[font=\small]
    \coordinate (south-west) at (0, 0);
    \coordinate (north-east) at (6, 8);
    \coordinate (north-west) at (south-west |- north-east);
    \coordinate (south-east) at (south-west -| north-east);
    \coordinate (east) at ($(south-east)!0.5!(north-east)$);
    \coordinate (west) at ($(south-west)!0.5!(north-west)$);
    \coordinate (south) at ($(south-east)!0.5!(south-west)$);
    \coordinate (north) at ($(north-east)!0.5!(north-west)$);
    \draw[line width=2pt, gray, opacity=0.5] (south-west) rectangle (north-east);
    \begin{scope}[gray, line width=6pt, opacity=0.3]
        \draw
            ($(north-west)+(1cm,-1cm)$)
            -- ($(north-west)+(2cm,-1cm)$)
            -- ($(south-east)+(-2cm,1cm)$)
            -- ($(south-east)+(-1cm,1cm)$)
        ;
        \draw
            ($(south-west)+(1cm,1cm)$)
            -- ($(south-west)+(2cm,1cm)$)
            -- ($(north-east)+(-2cm,-1cm)$)
            -- ($(north-east)+(-1cm,-1cm)$)
        ;
    \end{scope}
    \path (north-west) -- (south-west)
        node[pos=0.7, left] (ingress-4) {\bf \huge $\vdots$}
        \foreach \i/\c in {0.1/0,0.3/1,0.5/2,0.88/17} {
            node[pos=\i, rectangle, draw, minimum width=8mm, minimum height=8mm] (ingress-\c) {}
        };
    \path (north-east) -- (south-east)
        \foreach \i/\c in {0.2/31,0.7/30} {
            node[pos=\i, rectangle, draw, minimum width=8mm, minimum height=8mm] (egress-\c) {}
        };
    \foreach \c in {0, 1, 2, 17} {
        \node [below=0 of ingress-\c] {\bf Port \c};
    }
    \foreach \c in {31, 30} {
        \node [below=0 of egress-\c] {\bf Port \c};
    }
    \begin{scope}[darkblue, line width=3pt, ->]
        \draw
            ([xshift=-1cm] ingress-0.center)
            -- ([xshift=1cm] ingress-0.center)
            to[out=0, in=180] ([xshift=-1cm, yshift=1mm] egress-31.center)
            -- ([xshift=1cm, yshift=2mm] egress-31.center);
        \draw
            ([xshift=-1cm] ingress-1.center)
            -- ([xshift=1cm] ingress-1.center)
            to[out=0, in=180] ([xshift=-1cm, yshift=-2mm] egress-31.center)
            -- ([xshift=1cm, yshift=-2mm] egress-31.center);
        \node [right=5mm of egress-31] {背景流};
    \end{scope}
    \begin{scope}[darkred, line width=3pt, ->]
        \draw
            ([xshift=-1cm] ingress-2.center)
            -- ([xshift=1cm] ingress-2.center)
            to[out=0, in=180] ([xshift=-1cm, yshift=1mm] egress-30.center)
            -- ([xshift=1cm, yshift=2mm] egress-30.center);
        \draw
            ([xshift=-1cm] ingress-17.center)
            -- ([xshift=1cm] ingress-17.center)
            to[out=0, in=180] ([xshift=-1cm, yshift=-2mm] egress-30.center)
            -- ([xshift=1cm, yshift=-2mm] egress-30.center);
        \node [right=5mm of egress-30] {突发流};
    \end{scope}
\end{tikzpicture}
}
      \subcaption{实验场景}
      \label{c3:s6:ss1:fig:sub1:pfc avoidance scenario}
  \end{subfigure}
  \begin{subfigure}[b]{0.49\linewidth}
      \centering
      \includegraphics[width=\linewidth]{pfc-avoidance.pdf}
      \subcaption{总暂停时长}
      \label{c3:s6:ss1:fig:sub1:total pause duration}
  \end{subfigure}
  \caption{PFC避免}
  \label{c3:s6:ss1:fig:sub1:pfc avoidance}
\end{figure}

\subsubsection{PFC避免}

为了测试DSH在PFC避免方面的性能,测试场景如图\ref{c3:s6:ss1:fig:sub1:pfc avoidance scenario}所示,初始时刻两条背景流分别从入端口0和入端口1去往出端口31,在$t=0.1s$时开启16条突发流,分别从入端口2-17同时去往出端口30。所有突发流的暂停总时间结果如图\ref{c3:s6:ss1:fig:sub1:total pause duration}所示,在突发大小不超过缓存大小的40\%的情况下,DSH不会触发PFC,相对于SIH,不触发PFC情况下的突发吸纳量增加超过4倍。

\subsubsection{死锁避免}

引入PFC可能会发生网络死锁\cite{SIGCOMM16RDMA,INFOCOM14TCP-Bolt,CoNEXT17Tagger,SIGCOMM19GFC,INFOCOM22ITSY},导致部分网络瘫痪不可用。为了测试DSH在死锁避免方面的性能,搭建如图\ref{c3:s6:ss1:fig:sub1:deadlock scenario}所示的Spine-Leaf型网络拓扑,其中虚线标记的两条链路$S_0-L_3$和$S_1-L_0$因故障断开,拓扑中Spine层有2台交换机$S_0$和$S_1$,Leaf层有4台交换机$L_0-L_3$,每个Leaf交换机通过100Gbps下行链路连接16个主机,通过400Gbps上行链路和两个Spine交换机连接,链路时延均为$2\mu s$。该实验生成4条突发流,分别从$L_0-L_3$去往$L_3-L_0$,该场景下会存在循环缓存依赖:$S_0 \rightarrow L_1 \rightarrow S_1 \rightarrow L_2 \rightarrow S_0$,如图中红色链路标记所示。实验中突发流流数在1-15范围内随机选取,流大小基于Hadoop\cite{SIGCOMM10DCTCP}负载随机选取,流开始时间服从泊松分布并控制链路负载为50\%。不同机制下各自重复100次实验,每次实验持续$100ms$。

\begin{figure}[H]
  \begin{subfigure}[b]{0.49\linewidth}
      \centering
      \resizebox{\linewidth}{!}{\begin{tikzpicture}[font=\Large]
    \node[switch] at (0, 0) (l0) {};
    \node[switch, right=1 of l0] (l1) {};
    \node[switch, right=1 of l1] (l2) {};
    \node[switch, right=1 of l2] (l3) {};
    \node[switch, above=2 of $(l0.center)!0.5!(l1.center)$] (s0) {};
    \node[switch, above=2 of $(l2.center)!0.5!(l3.center)$] (s1) {};

    \foreach \i in {0,1,2,3} {
        \node[right=-0.1 of l\i] {$L_{\i}$};
        \node[server, below  left=1 and -0.6 of l\i] (l\i-h0) {};
        \node[server, below right=1 and -0.6 of l\i] (l\i-h1) {};
        \path (l\i-h0) -- (l\i-h1) node[midway] {\bf...};
    }
    \foreach \i in {0,1} {
        \node[right=-0.1 of s\i] {$S_{\i}$};
    }
    % network links
    \begin{scope}[on background layer, every path/.style={gray, line width=1}]
        \draw (l0.center) -- (s0.center);
        \draw (l1.center) -- (s0.center);
        \draw (l2.center) -- (s0.center);
        \draw (l1.center) -- (s1.center);
        \draw (l2.center) -- (s1.center);
        \draw (l3.center) -- (s1.center);
        \draw[dashed] (s0.center) -- (l3.center);
        \draw[dashed] (s1.center) -- (l0.center);
        \foreach \i in {0,1,2,3} {
            \draw (l\i.center) -- (l\i-h0.center);
            \draw (l\i.center) -- (l\i-h1.center);
        }
    \end{scope}
    \begin{scope}[
        every path/.style={decorate, decoration={brace,mirror,amplitude=6pt,raise=0.1ex}, very thick,},
        every node/.style={xshift=-1ex, left, anchor=east}
    ]
        \foreach \i in {0,1,2,3} {
            \draw (l\i-h0.south) -- (l\i-h1.south)
                node[midway, below=0.2] {16 Hosts};
        }
    \end{scope}
    \begin{scope}[on background layer, every path/.style={darkred, line width=3}]
        \draw[->] (s0.center) -- (l1);
        \draw[->] (l1.center) -- (s1);
        \draw[->] (s1.center) -- (l2);
        \draw[->] (l2.center) -- (s0);
    \end{scope}
\end{tikzpicture}
}
      \subcaption{实验场景}
      \label{c3:s6:ss1:fig:sub1:deadlock scenario}
  \end{subfigure}
  \begin{subfigure}[b]{0.49\linewidth}
      \centering
      \includegraphics[width=\linewidth]{deadlock-avoidance.pdf}
      \subcaption{死锁发生时间CDF分布}
      \label{c3:s6:ss1:fig:sub1:cdf of deadlock onset time}
  \end{subfigure}
  \caption{死锁避免}
  \label{c3:s6:ss1:fig:deadlock avoidance}
\end{figure}

图\ref{c3:s6:ss1:fig:sub1:cdf of deadlock onset time}显示了不同机制下死锁开始时间的CDF分布。在SIH下基于DCQCN\cite{SIGCOMM15DCQCN}和PowerTCP\cite{NSDI22PowerTCP}的所有实验均发生死锁,相较之下,DSH在DCQCN下避免了96\%的死锁,在PowerTCP下完全避免死锁。这得益于DSH可以利用更多缓存空间吸纳突发从而避免PFC触发。

\subsubsection{附带损害消除}

引入PFC还可能导致无辜流性能受损\cite{SIGCOMM16RDMA},DSH同样可以通过避免PFC触发避免附加性能损害。

图\ref{c3:s6:ss1:fig:sub1:collateral damage scenario}所示的数据中心典型场景在相关工作中被广泛使用\cite{NSDI20PCN,SIGCOMM17NDP,SIGCOMM21TCD},其中所有的链路带宽为100Gbps,传播时延为$2 \mu s$。两条长流$F_0$和$F_1$分别从$H_0$和$H_1$发往$R_0$和$R_1$,当两条流的吞吐率各自稳定到50Gbps时,$H_2-H_{25}$同时生成24条突发流量发往$R_1$,每条流的大小为64KB。由于64KB小于BDP大小,拥塞控制机制无法及时对突发流进行调节。此时拥塞点位于$S_1$,$F_1$和所有突发流共同导致该拥塞的发生,而$F_0$并不导致拥塞,所以理想情况下无辜流$F_0$的吞吐率不应该受损。

\begin{figure}[H]
  \begin{subfigure}[b]{0.49\linewidth}
      \centering
      \resizebox{\linewidth}{!}{\begin{tikzpicture}[font=\large]
    \node[switch] at (0, 0) (s0) {};
    \node[above=0 of s0, font=\small] {$S_0$};

    \node[switch, right=1 of s0] (s1) {};
    \node[above=0 of s1, font=\small] {$S_1$};

    \node[server, above left=-0.1 and 1 of s0] (h0) {};
    \node[above=0 of h0, font=\small] {$H_0$};
    \node[server, below left=-0.1 and 1 of s0] (h1) {};
    \node[above=0 of h1, font=\small] {$H_1$};

    \node[server, below left= 1 and -0.1 of s1] (h2) {};
    \node[below=0 of h2, font=\small] {$H_2$};
    \node[server, below right=1 and -0.1 of s1] (h3) {};
    \node[below=0 of h3, font=\small] {$H_{25}$};
    \path (h2) -- (h3) node[midway] (dots) {\Large \bf ...};

    \node[server, above right=-0.1 and 1 of s1] (r0) {};
    \node[above=0 of r0, font=\small] {$R_0$};
    \node[server, below right=-0.1 and 1 of s1] (r1) {};
    \node[above=0 of r1, font=\small] {$R_1$};

    % network links
    \begin{scope}[on background layer, every path/.style={gray, line width=1}]
        \draw (h0.center) -- (s0.center) -- (s1.center) -- (r0.center);
        \draw (h1.center) -- (s0.center);
        \draw (r0.center) -- (s1.center) -- (r1.center);
        \draw (h2.center) -- (s1.center) -- (h3.center);
    \end{scope}

    % flows
    \begin{scope}[line width=2pt, rounded corners=10pt]]
        \draw[->, darkred]
            ([yshift=2mm] $(h0.center)!0.2!(s0.center)$)
            -- ([yshift=2mm] s0.center) node[midway, above, font=\small] {$F_0$}
            -- ([yshift=2mm] $(s0.center)!0.7!(s1.center)$);
        \draw[->, darkred]
            ([yshift=2mm] $(s1.center)!0.2!(r0.center)$)
            -- ([yshift=2mm] $(r0.center)!0.2!(s1.center)$);
        \draw[->, darkblue]
            ([yshift=-2mm] $(h1.center)!0.2!(s0.center)$)
            -- ([yshift=-2mm] s0.center) node[midway, below, font=\small] {$F_1$}
            -- ([yshift=-2mm] $(s0.center)!0.7!(s1.center)$);
        \draw[->, darkblue]
            ([yshift=-2mm] $(s1.center)!0.2!(r1.center)$)
            -- ([yshift=-2mm] $(r1.center)!0.2!(s1.center)$);
        \draw[->, darkblue]
            ([xshift=2mm] $(h2.center)!0.2!(s1.center)$)
            -- ([xshift=2mm] $(h2.center)!0.8!(s1.center)$);
        \draw[->, darkblue]
            ([xshift=-2mm] $(h3.center)!0.2!(s1.center)$) -- ([xshift=-2mm] $(h3.center)!0.8!(s1.center)$);
        \node[above=0.2 of dots, darkblue, font=\small] {突发};
    \end{scope}

\end{tikzpicture}
}
      \subcaption{实验场景}
      \label{c3:s6:ss1:fig:sub1:collateral damage scenario}
  \end{subfigure}
  \begin{subfigure}[b]{0.49\linewidth}
      \centering
      \includegraphics[width=\linewidth]{Figures/collateral-damage-None.pdf}
      \subcaption{流$F_0$的吞吐率(w/o CC)}
      \label{c3:s6:ss1:fig:sub1:throughput of f0 w/o cc}
  \end{subfigure}
  \begin{subfigure}[b]{0.49\linewidth}
    \centering
    \includegraphics[width=\linewidth]{Figures/collateral-damage-DCQCN.pdf}
    \subcaption{流$F_0$的吞吐率(DCQCN)}
    \label{c3:s6:ss1:fig:sub1:throughput of f0 dcqcn}
  \end{subfigure}
  \begin{subfigure}[b]{0.49\linewidth}
    \centering
    \includegraphics[width=\linewidth]{Figures/collateral-damage-PowerTCP.pdf}
    \subcaption{流$F_0$的吞吐率(PowerTCP)}
    \label{c3:s6:ss1:fig:sub2:collateral throughput of f0 powertcp}
  \end{subfigure}  
  \caption{附带损害消除}
  \label{c3:s6:ss1:fig:collateral damage}
\end{figure}

图\ref{c3:s6:ss1:fig:collateral damage}显示了$F_0$的吞吐率变化。SIH下$F_0$的吞吐率严重受损,其原因在于SIH中$X_{qoff}$阈值较低,所以很容易触发PFC暂停帧。当$S_0$被暂停发送时,$F_0$的数据传输也被暂停。相较之下,DSH可以有效避免$F_0$的吞吐率受损,这得益于DSH分配净空缓存的高效性,从而可以释放更多缓存空间作为共享缓存避免PFC触发。此外,图\ref{c3:s6:ss1:fig:sub1:throughput of f0 dcqcn}和\ref{c3:s6:ss1:fig:sub2:collateral throughput of f0 powertcp}结果显示,现行拥塞控制算法并不能避免PFC带来的附带损害,这是由于端到端拥塞控制算法至少需要一个RTT的时间才能对流量变化做出响应。因此,在第一个RTT的时间内缓存管理机制决定是否可以避免PFC触发。

\xsubsection{大规模网络性能测试}{Performance Evaluation in Large-scale Network}

本节通过在ns-3中搭建大规模数据中心网络,进一步测试DSH在大规模网络中的性能表现。

\subsubsection{拓扑结构}

通过ns-3搭建如图\ref{fig:c3:spine leaf topology}所示Spine-Leaf网络拓扑结构,其中Spine层有16个交换机,Leaf层有16个交换机,每个Leaf交换机下连接16个主机,Spine层交换机和Leaf层交换机之间两两进行全连接,所有链路带宽均为100Gbps,传播时延均为$2 \mu s$,则跨Spine传输的基准RTT为$16 \mu s$。网络中部署ECMP负载均衡机制。

\begin{figure}[H]
  \centering
  \includegraphics[width=0.7\textwidth]{spine_leaf_architecture.pdf}
  \caption{Spine-Leaf拓扑结构}
  \label{fig:c3:spine leaf topology}
\end{figure}

\subsubsection{交换芯片}

网络中的交换机模拟Broadcom Tomahawk交换芯片,配置同\ref{c3:s5:ss1:basic performance evaluation}所述,具体参数配置总结如图\ref{fig:c3:parameter setting in switch chip}所示。

\begin{figure}[H]
  \begin{table}[H]
      \begin{tabularx}{\textwidth}{YYYY}
      \toprule
          \textbf{参数} & \textbf{具体配置} & \textbf{参数} & \textbf{具体配置} \\
      \midrule
          端口数量 & 32 & 端口支持队列数 & 8 \\
          端口带宽 & 100Gbps & DT控制参数$\alpha$ & $1/16$ \\
          缓存容量 & 16MB & 私有缓存预留量 & 672KB \\
          $X_{qon}$偏移量$\delta_q$ & 0 & 净空缓存预留量 & 12MB \\
          $X_{pon}$偏移量$\delta_p$ & 0 & 队列调度策略 & SQ|DWRR \\
      \bottomrule
      \end{tabularx}
  \end{table}
  \caption{交换芯片相关配置}
  \label{fig:c3:parameter setting in switch chip}
\end{figure}


\subsubsection{传输层}

端到端拥塞控制算法主要采用DCQCN\cite{SIGCOMM15DCQCN}和PowerTCP\cite{NSDI22PowerTCP},其中的参数配置参考自开源仿真实现\cite{HPCCGitHub},具体配置如图\ref{fig:c3:parameter setting in transport protocol}所示。

\subsubsection{负载流量}

网络中产生的流量类型包括两种:背景流和突发流。背景流形式为一打一,其发送端和接收端均随机产生,流大小根据web search\cite{SIGCOMM10DCTCP}负载模型确定,流开始时间服从泊松分布,流量所属类别在1-7之间随机生成;突发流形式为多打一,16个发送端同时向同一个接收端发送大小为64KB的流量,发送端随机产生但保证与接收端不在同一个机架下,突发流为固定流量类别,背景流的类别随机从突发流以外类别产生,最大网络负载为90\%。

\begin{figure}[H]
  \begin{table}[H]
      \begin{tabularx}{\textwidth}{YYY}
      \toprule
          \textbf{参数} & \textbf{DCQCN} & \textbf{PowerTCP}\\
      \midrule
          基准RTT($\tau$) & $16\mu s$ & $16\mu s$ \\
          $K_{min}$ & $400KB$ & / \\
          $K_{max}$ & $1600KB$ & / \\
          $P_{max}$ & 1 & / \\
          $g$ & $1/16$ & / \\
          % EWMA增益$R_{AI}$ & $20Mbps$ & / \\
          % RP计时器($T$) & $55\mu s$ & / \\
          快速恢复次数($F$) & 5 & / \\
          目标利用率($\eta$) & / & 0.95 \\
          EWMA参数($\gamma$) & / & 0.9 \\
      \bottomrule
      \end{tabularx}
  \end{table}
  \caption{传输层相关配置}
  \label{fig:c3:parameter setting in transport protocol}
\end{figure}

\subsubsection{实验结果}

不同背景流负载下的背景流和突发流FCT结果如图\ref{c3:s6:ss1:fig:benchmark fct}所示,为了便于比较,图中将所有FCT结果相对于SIH进行归一化处理。结果显示DSH对于背景流和突发流的FCT均有明显提升,在DCQCN下,DSH可以将背景流和突发流的平均分别减小10.1\%和43.3\%;在Power TCP下分别减小31.1\%和57.7\%。

\begin{figure}[H]
  \begin{subfigure}[b]{0.49\linewidth}
      \centering
      \includegraphics[width=\linewidth]{Figures/benchmark-avg-fct-incast-dcqcn.pdf}
      \subcaption{突发流(DCQCN)}
      \label{c3:s6:ss1:fig:sub1:benchmark fct burst dcqcn}
  \end{subfigure}
  \begin{subfigure}[b]{0.49\linewidth}
      \centering
      \includegraphics[width=\linewidth]{Figures/benchmark-avg-fct-back-dcqcn.pdf}
      \subcaption{背景流(DCQCN)}
      \label{c3:s6:ss1:fig:sub2:benchmark fct back dcqcn}
  \end{subfigure}
  \begin{subfigure}[b]{0.49\linewidth}
    \centering
    \includegraphics[width=\linewidth]{Figures/benchmark-avg-fct-incast-powertcp.pdf}
    \subcaption{突发流(PowerTCP)}
    \label{c3:s6:ss1:fig:sub1:benchmark fct burst powertcp}
  \end{subfigure}
  \begin{subfigure}[b]{0.49\linewidth}
    \centering
    \includegraphics[width=\linewidth]{Figures/benchmark-avg-fct-back-powertcp.pdf}
    \subcaption{背景流(PowerTCP)}
    \label{c3:s6:ss1:fig:sub1:throughput of f0 powertcp}
  \end{subfigure} 
  \caption{不同背景流负载下的平均FCT}
  \label{c3:s6:ss1:fig:benchmark fct}
\end{figure}

在不同网络应用和拓扑结构下进一步测试DSH的流量传输性能。采用的流量模式包括:Web Search\cite{SIGCOMM10DCTCP}、Data Mining\cite{SIGCOMM09VL2}、Cache\cite{SIGCOMM15FB}以及Hadoop\cite{SIGCOMM15FB};网络拓扑包括:Spine-Leaf和Fat-tree胖树\cite{al2008scalable}(k=16)结构。不同负载下背景流的FCT结果如图\ref{c3:s6:ss1:fig:benchmark pattern and topology}所示,结果显示,在不同流量模式和网络拓扑下,DSH对于FCT均有不同程度改善。

\begin{figure}[H]
  \begin{subfigure}[b]{0.49\linewidth}
      \centering
      \includegraphics[width=\linewidth]{Figures/benchmark-pattern-avg-fct-back-dcqcn-mining.pdf}
      \subcaption{Spine-Leaf\ + \ Data mining}
      \label{c3:s6:ss1:fig:sub1:benchmark pattern Spine-Leaf mining}
  \end{subfigure}
  \begin{subfigure}[b]{0.49\linewidth}
      \centering
      \includegraphics[width=\linewidth]{Figures/benchmark-pattern-avg-fct-back-dcqcn-cache.pdf}
      \subcaption{Spine-Leaf\ + \ Cache}
      \label{c3:s6:ss1:fig:sub1:benchmark pattern Spine-Leaf cache}
  \end{subfigure}
  \begin{subfigure}[b]{0.49\linewidth}
    \centering
    \includegraphics[width=\linewidth]{Figures/benchmark-pattern-avg-fct-back-dcqcn-hadoop.pdf}
    \subcaption{Spine-Leaf\ + \ hadoop}
    \label{c3:s6:ss1:fig:sub1:benchmark pattern Spine-Leaf hadoop}
  \end{subfigure}
  \begin{subfigure}[b]{0.49\linewidth}
    \centering
    \includegraphics[width=\linewidth]{Figures/benchmark-fattree-avg-fct-back-dcqcn-cache.pdf}
    \subcaption{Fat-Tree\ + \ Web \ Search}
    \label{c3:s6:ss1:fig:sub1:benchmark pattern fat-tree search}
  \end{subfigure} 
  \caption{不同负载和拓扑下的平均FCT(DCQCN)}
  \label{c3:s6:ss1:fig:benchmark pattern and topology}
\end{figure}

\xsection{本章小结}{Brief Summary}
\label{c3:s7:brief summary}

本章首先描述片上缓存系统现有净空缓存管理策略SIH的具体机制,指出其在净空缓存分配上固有的低效性,分析SIH当前交换芯片带宽高速增长趋势下存在的缓存资源浪费和频繁PFC触发等问题。针对SIH静态隔离分配方式的低效性,提出了一种动态共享净空缓存分配机制DSH,DSH的基本思想即动态地为拥塞队列分配净空缓存,同时在不同队列之间共享分配的净空缓存,通过统计复用的方式减少净空缓存预留量,提高缓存利用效率。在缓存分配方面,DSH不再为每个队列静态预留最坏情况下的净空缓存需求量,而是仅为每个端口预留,以此减少净空缓存预留量;在流量控制方面,DSH结合队列级别流量控制和端口级别流量控制,在避免丢包的同时保证队列之间的隔离性。为了体现DSH的性能优势,本章通过理论分析和实验仿真对DSH的性能进行验证,理论上通过公式推导证明了DSH在突发吸纳能力上优于SIH;实验上通过设计不同网络场景和大规模网络环境验证了DSH在PFC避免、死锁避免和附带损害消除等方面更好的性能表现,以及在大规模网络场景不同流量负载、不同应用场景和不同网络拓扑下的网络性能提升。

% !TeX root = ../main.tex

\xchapter{面向异构缓存系统的动态共享净空缓存管理机制H-DSH}{Dynamic and Shared Headroom Allocation Scheme for Hybrid Buffer System}

本章首先指出异构缓存系统现行净空缓存管理策略存在的低效性和性能损害等问题,然后分析问题的主要来源,提出了一种适用于异构缓存系统架构的动态共享缓存管理策略H-DSH。H-DSH以片外缓存为中心,一方面,将片外缓存作为主要共享缓存空间进行动态分配,尽可能将缓存积累限制在片外缓存,以缓解片上缓存的容量压力;另一方面,在片上为端口静态预留保底净空缓存池避免丢包,同时在合适的时机将拥塞流量回流到片上,以避免片外缓存的带宽瓶颈。本章\ref{c4:s1:current buffer system scheme}节描述了异构缓存系统现行缓存管理策略H-SIH的具体机制。\ref{c4:s2:problem analysis}节指出H-SIH在片外缓存利用、PFC附带损害和长距离传输方面存在的问题并分析问题来源。\ref{c4:s3:hdsh design}节提出H-DSH的设计目标、基本思想、主要挑战以及各个模块的实现方案。\ref{c4:s4:hdsh implementation}节详细阐述了H-DSH的具体实现细节。\ref{c4:s5:hdsh evaluation}节通过实验验证H-DSH的基本性能表现和复杂网络场景适应性。\ref{c4:s6:brief summary}节总结本章主要工作。


\xsection{现行缓存管理机制}{Current Buffer System Scheme}
\label{c4:s1:current buffer system scheme}

异构缓存系统通常部署于路由器。不同于片上缓存系统,MMU需要同时对片上缓存和片外缓存进行管理,缓存管理策略的制定需要考虑片上缓存和片外缓存的不同物理特性。经过调研发现,主流厂商对异构缓存系统的普遍认识为以片上缓存为中心,片外缓存作为超额订购,如Cisco\cite{CiscoNcs5500}和Broadcom\cite{BCM88480}。对于异构缓存系统的无损缓存池,现行缓存管理机制H-SIH在净空缓存分配上仍然采用静态分配方式,在流量控制上通过结合全局流量控制避免片外带宽用尽而丢包。

\xsubsection{缓存分配}{Buffer Allocation}

\begin{figure}[H]
  \centering
  \includegraphics[width=0.7\textwidth]{Figures/buffer_partition_hsih.pdf}
  \caption{H-SIH缓存分区}
  \label{c4:s1:ss1:fig:hsih buffer partition}
\end{figure}

H-SIH在逻辑上将无损缓存池划分为私有缓存、共享缓存和净空缓存。其中私有缓存和净空缓存位于片上缓存,共享缓存同时存在于片上缓存和片外缓存。H-SIH对片上缓存和片外缓存的逻辑分区如图\ref{c4:s1:ss1:fig:hsih buffer partition}所示,异构缓存系统中私有缓存分配方式和片上缓存系统相同,以静态方式在片上缓存预留。净空缓存不以队列为单位预留,而是在片上预留全局净空缓存池。H-SIH中共享缓存分为两部分:片上缓存除私有缓存和净空缓存以外的剩余空间和整个片外缓存空间,片外共享缓存空间仅在拥塞队列之间共享。

对于不同缓存分区的管理,异构缓存系统与片上缓存系统类似。私有缓存空间优先占用,为每个队列分配几个报文的空间,如3KB;净空缓存空间在触发PFC暂停后开始占用,为每个队列分配最坏情况下的需求量,即公式(\ref{eqn:c2:headroom calculation})中的$\eta$大小。共享缓存空间在所有队列之间动态共享,H-SIH在缓存分配上以片上缓存为中心,空队列或者轻微拥塞队列优先占用片上共享缓存,拥塞队列占用片外共享缓存。具体地,每个队列可用的片上共享缓存空间通过动态阈值限制,达到阈值限制后判定为拥塞队列开始占用片外缓存空间,动态阈值由公式(\ref{eqn:c4:dt threshold in hybrid buffer system})计算得到:
\begin{equation}
  T(t)=\alpha \cdot (B_{os} - \sum_{i} \sum_{j} \omega_{os}^{i,j}(t))
  \label{eqn:c4:dt threshold in hybrid buffer system}
\end{equation}

\noindent 其中,$B_{os}$表示片上共享缓存分区大小,$\omega_{os}^{i,j}(t)$表示$t$时刻端口$i$中队列$j$占用片上共享缓存的总量

\xsubsection{流量控制}{Flow Control}
H-SIH的流量控制机制与标准PFC类似,仅触发条件存在差别。在片外缓存容量或带宽不足时引入全局流量控制,全局流量控制即触发片外缓存中所有队列的PFC控制帧,不需要引入额外机制,具体地,全局流量控制触发条件为满足以下状态之一:
\begin{equation}
  \begin{cases}
    \sum_{i} \sum_{j} q^{i,j}_{\text{off}}(t) \geqslant B_{\text{off}} \\
    Bw_{\text{used}} \geqslant \beta \cdot Bw_{\text{off}} \\
  \end{cases}
  \label{eqn:c4:global pause invoke condition}
\end{equation}

\noindent 其中$q^{i,j}_{\text{off}}(t)$表示$t$时刻端口$i$中队列$j$在片外缓存的队列长度,$B_{\text{off}}$和$Bw_{\text{off}}$分别表示片外缓存容量和带宽大小,$Bw_{\text{used}}$为当前片外带宽占用,$\beta$为片外带宽占用阈值参数。

在触发PFC恢复时,H-SIH不采用全局策略,而是独立地控制每个队列的恢复状态转换,队列PFC恢复阈值$X_{\text{on}}(t)$设置为:
\begin{equation}
  X_{\text{on}}(t) = T(t) - \delta
  \label{eqn:c4:pfc resume threshold}
\end{equation}

\noindent 与标准PFC不同的是,$X_{\text{on}}(t)$限制的是片上共享缓存和片外缓存队列长度之和,即满足如下条件:
\begin{equation}
  \omega_{s}^{i,j}(t) \leqslant X_{\text{on}}(t)
  \label{eqn:c4:resume condition}
\end{equation}


综上所述,H-SIH的流量控制机制可以由图\ref{c4:s1:ss1:fig:hsih state transition}中的状态机来描述,除触发条件外与\ref{c3:s1:ss2:flow control}节阐述的PFC机制类似,本节不再展开赘述。
\begin{figure}[H]
  \begin{subfigure}[b]{0.49\linewidth}
      \centering
      \includegraphics[width=\linewidth]{state_transition_hsih.pdf}
      \subcaption{入口队列}
      \label{c3:s3:ss4:fig:sub1:hsih ingress queue state transition}
  \end{subfigure}
  \begin{subfigure}[b]{0.49\linewidth}
      \centering
      \includegraphics[width=\linewidth]{state_transition_pfc_out.pdf}
      \subcaption{出口队列}
      \label{c3:s3:ss4:fig:sub2:hsih egress queue state transition}
  \end{subfigure}
  \caption{H-SIH流量控制状态机}
  \label{c4:s1:ss1:fig:hsih state transition}
\end{figure}

\xsubsection{MMU处理流程}{MMU Workflow}

H-SIH的处理逻辑在MMU中实现,主要逻辑包括缓存使用情况监测、报文准入决策和流量控制触发。MMU实时记录各个缓存分区的占用,报文到达时根据缓存使用情况对该报文进行决策,MMU在每个报文入队前和出队后检查PFC状态转换。具体地,MMU在报文入队前按照以下流程进行决策:

1)$q^{i,j}_{\text{on}}(t)<\phi$:MMU决策该报文进入私有缓存空间,即优先占用私有缓存。

2)$q^{i,j}_{\text{on}}(t)< \phi+T(t)$:MMU决策该报文进入片上共享缓存空间。即当前队列处于非拥塞状态,占用片上共享缓存空间。

3)$\sum_{i}\sum_{j}q^{i,j}_{\text{off}}(t) < B_{\text{off}} ~\bigwedge~ Bw_{\text{used}} < \beta \cdot Bw_{\text{off}}$:MMU决策该报文进入片外缓存空间。此时,片上缓存队列长度达到阈值,片外缓存可用带宽和容量足够,当前队列进入拥塞状态,开始占用片外共享缓存。

4)$\sum_{i}\sum_{j}q^{i,j}_{\text{off}}(t) \geqslant B_{\text{off}} ~\bigvee~ Bw_{\text{used}} \geqslant \beta \cdot Bw_{\text{off}}$:MMU决策该报文进入片上净空缓存。此时,片外容量或带宽用尽,触发全局流量控制,全局流量控制即片外缓存所有队列均触发PFC控制帧。

5)$q^{i,j}_{\text{on}}(t) < \phi +T(t)+\eta$:MMU决策报文进入片内净空缓存。当前队列触发PFC,片上净空缓存容纳控制帧生效之前到达的报文。

6)$q^{i,j}_{\text{on}}(t) \geqslant \phi +T(t)+\eta$:片上净空缓存溢出,MMU决策该报文丢弃。

\noindent 其中$\phi$表示给每个队列预留的私有缓存大小,$q^{i,j}_{\text{on}}(t)$表示$t$时刻端口$i$中队列$j$在片上缓存的队列长度。

在报文出队后,MMU需要检查每个队列,根据缓存占用情况触发PFC恢复。具体地,如果入口队列当前处于OFF状态且满足$\omega_{s}^{i,j}(t)<X_{\text{on}}(t)$时,MMU向上游设备发送一个恢复帧同时将入口队列转换为ON状态。

\xsection{问题分析}{Problem Analysis}
\label{c4:s2:problem analysis}

本节指出异构缓存系统中现行缓存分配机制存在的突发吸纳受限于片外带宽、流量控制损害线速吞吐和长距离传输存在丢包问题。结合片上缓存和片外缓存各自的物理特性以及无损网络部署发展趋势分析了上述问题存在的来源以及寻求更高效缓存分配策略的必要性。

\xsubsection{突发吸纳受限于片外带宽}{Burst Absorption is Limited by Off-chip Bandwidth}

异构缓存系统包括片上缓存和片外缓存两部分缓存空间,异构性在于片上缓存和片外缓存的不同物理特性。如图\ref{c4:s1:ss1:fig:hybrid buffer system features}所示,片上缓存集成在芯片内部,距离处理器核心很近,可以通过高速的内部总线实现快速访问,因此片上缓存带宽可以满足线速转发的需求。但是受限于芯片面积和能耗,片上缓存的容量有限,通常只有几百KB到几十MB的大小。片外缓存在容量和带宽上与片上缓存彼此互补:一方面,片外缓存独立于处理芯片之外,可以通过DRAM实现GB级别的缓存容量;另一方面,片外缓存需要通过数据总线与处理芯片通信,访问速度相对较慢,即使利用HBM技术\cite{jedecHBM2E,kim2019design}进行部署,读写带宽通常也只能达到转发带宽的一半左右\cite{Ciscohybridbuffer,}。

\begin{figure}[H]
  \centering
  \includegraphics[width=0.85\textwidth]{Figures/hybrid-buffer-system-features.pdf}
  \caption{异构缓存系统特点}
  \label{c4:s1:ss1:fig:hybrid buffer system features}
\end{figure}

H-SIH在缓存分配上以片上缓存为中心,只有在发生拥塞时才占用片外缓存空间。具体地,非拥塞或轻度拥塞队列占用片上缓存,拥塞队列占用片外缓存空间。H-SIH通过队列长度来识别拥塞队列,即队列长度大于动态阈值$T(t)$时识别为拥塞队列开始决策进入片外缓存。该机制在少量队列拥塞的情况下可以保证片外缓存吸纳队列积累,但是受限于片外缓存读写带宽,在带宽瓶颈时会由于片外写带宽不足导致突发吸纳能力受限。

具体地,H-SIH在片外带宽瓶颈时触发全局流量控制避免丢包,全局流量控制对所有流量触发无差别PFC暂停,严重缩减了片外缓存的可用空间。为了定量说明H-SIH在片外带宽瓶颈时存在的片外缓存利用低效性,本节通过ns-3搭建星型拓扑网络,其中16台主机作为发送端,另外16台主机作为接收端,链路带宽和时延分别为100Gbps和1$us$。背景流基于web \ search\cite{SIGCOMM10DCTCP}模型随机产生,流开始时间服从泊松分布同时将链路负载控制到90\%;突发流形式为十六打一,突发大小以4MB递增,目的主机随机挑选,每经过1$ms$产生一次突发。片外缓存队列长度如图\ref{c4:s1:ss1:fig:dram qlen motivation}所示,在片外带宽瓶颈场景下,片外缓存中的突发吸纳量最高只能达到10MB左右,大小仅突发总量的四分之一。

\begin{figure}[H]
  \centering
  \includegraphics[width=0.5\textwidth]{Figures/burst-absorption-motivation.pdf}
  \caption{H-SIH在不同突发大小下的片外缓存队列长度}
  \label{c4:s1:ss1:fig:dram qlen motivation}
\end{figure}

片外缓存利用低效性的主要来源在于H-SIH在片外带宽瓶颈时的粗粒度流量控制。H-SIH在触发全局流量控制时无差别地对所有流量触发PFC暂停,无疑会损害此时处于活跃状态的正常流量和突发流量,进而限制了片外缓存本身具有的突发吸纳能力。


\xsubsection{流量控制损害线速吞吐}{Throughput Damage from Flow Control}

\begin{figure}[H]
  \centering
  \resizebox{0.5\linewidth}{!}{\begin{tikzpicture}[font=\large]
    \node[switch] at (0, 0) (s0) {};
    \node[above=0 of s0, font=\small] {$R_0$};

    \node[switch, right=1 of s0] (s1) {};
    \node[above=0 of s1, font=\small] {$R_1$};

    \node[server, above left=-0.1 and 1 of s0] (h0) {};
    \node[above=0 of h0, font=\small] {$H_0$};
    \node[server, below left=-0.1 and 1 of s0] (h1) {};
    \node[above=0 of h1, font=\small] {$H_1$};

    \node[server, below left= 1 and -0.1 of s1] (h2) {};
    \node[below=0 of h2, font=\small] {$H_2$};
    \node[server, below right=1 and -0.1 of s1] (h3) {};
    \node[below=0 of h3, font=\small] {$H_{17}$};
    \path (h2) -- (h3) node[midway] (dots) {\Large \bf ...};

    \node[server, above right=-0.1 and 1 of s1] (r0) {};
    \node[above=0 of r0, font=\small] {$H_{19}$};
    \node[server, below right=-0.1 and 1 of s1] (r1) {};
    \node[above=0 of r1, font=\small] {$H_{18}$};

    % network links
    \begin{scope}[on background layer, every path/.style={gray, line width=1}]
        \draw (h0.center) -- (s0.center) -- (s1.center) -- (r0.center);
        \draw (h1.center) -- (s0.center);
        \draw (r0.center) -- (s1.center) -- (r1.center);
        \draw (h2.center) -- (s1.center) -- (h3.center);
    \end{scope}

    % flows
    \begin{scope}[line width=2pt, rounded corners=10pt]]
        \draw[->, darkred]
            ([yshift=2mm] $(h0.center)!0.2!(s0.center)$)
            -- ([yshift=2mm] s0.center) node[midway, above, font=\small] {$F_0$}
            -- ([yshift=2mm] $(s0.center)!0.7!(s1.center)$);
        \draw[->, darkred]
            ([yshift=2mm] $(s1.center)!0.2!(r0.center)$)
            -- ([yshift=2mm] $(r0.center)!0.2!(s1.center)$);
        \draw[->, darkblue]
            ([yshift=-2mm] $(h1.center)!0.2!(s0.center)$)
            -- ([yshift=-2mm] s0.center) node[midway, below, font=\small] {$F_1$}
            -- ([yshift=-2mm] $(s0.center)!0.7!(s1.center)$);
        \draw[->, darkblue]
            ([yshift=-2mm] $(s1.center)!0.2!(r1.center)$)
            -- ([yshift=-2mm] $(r1.center)!0.2!(s1.center)$);
        \draw[->, darkblue]
            ([xshift=2mm] $(h2.center)!0.2!(s1.center)$)
            -- ([xshift=2mm] $(h2.center)!0.8!(s1.center)$);
        \draw[->, darkblue]
            ([xshift=-2mm] $(h3.center)!0.2!(s1.center)$) -- ([xshift=-2mm] $(h3.center)!0.8!(s1.center)$);
        \node[above=0.2 of dots, darkblue, font=\small] {突发};
    \end{scope}

\end{tikzpicture}
}
  \caption{受害流场景}
  \label{c4:s1:ss1:fig:victim flow scenario}
\end{figure}

触发PFC控制帧会产生连锁反应,瓶颈链路上发生的拥塞可能会导致其它转发路径上的流量性能受损。考虑图\ref{c4:s1:ss1:fig:victim flow scenario}中的受害流场景,20台主机通过两台路由器连接,所有的链路带宽为100Gbps,链路时延为$1us$,其中有两条长流$F_0$途径$H_0 \rightarrow R_0 \rightarrow R_1 \rightarrow H_{19}$转发,$F_1$途径$H_1 \rightarrow R_0 \rightarrow R_1 \rightarrow H_{18}$,$F_0$和$F_1$属于同一流量类别所以进入同一个队列。在稳态情况下,$F_0$和$F_1$应当公平共享链路$R_0-R_1$带宽。在$t=0$时刻,突发流量同时从$H_2-H_{17}$以线速100Gbps发往$R_1$经路由转发给$H_{18}$。所有突发流量达到$R_1$后,瓶颈链路在于$R_1-H_{18}$,流量会在$R_1$的缓存中不断积累,直至触发PFC。

\begin{figure}[H]
  \begin{subfigure}[b]{0.49\linewidth}
      \centering
      \includegraphics[width=\linewidth]{Figures/collateral-damage-wo-motivation.pdf}
      \subcaption{流$F_0$的吞吐率(w/o CC)}
      \label{c3:s6:ss1:fig:sub1:f0 throughput w/o cc}
  \end{subfigure}
  \begin{subfigure}[b]{0.49\linewidth}
      \centering
      \includegraphics[width=\linewidth]{Figures/collateral-damage-TcpNewReno-motivation.pdf}
      \subcaption{流$F_0$的吞吐率(TCP New-Reno)}
      \label{c3:s6:ss1:fig:sub1:throughput of f0 new reno}
  \end{subfigure}
  \caption{无辜流量吞吐受损}
  \label{c3:s6:ss1:fig:throughput damage}
\end{figure}

本节通过在ns-3仿真平台模拟上述网络场景,在PFC和PFC+TcpNewReno拥塞管理机制下测试H-SIH的性能,$F_1$的吞吐率结果如图\ref{c3:s6:ss1:fig:throughput damage}所示。在突发流量到达之前,$F_0$的吞吐率稳定在50Gbps左右,突发流量达到触发PFC之后,$F_0$的吞吐率下降至接近0。$F_0$的转发路径不包括瓶颈链路$R_1-H_{18}$,但是其吞吐率仍然受其拥塞影响,$F_0$成为该场景下的受害流。由于突发流和背景流$F_1$的瓶颈链路位于$R_1-H_{18}$,发生拥塞后会触发其上行链路的PFC暂停,即$R_1$向其上行链路发送暂停帧,进而拥塞传播到$R_0$触发$R_0$向上游发送暂停帧。最终,PFC传播到发送端导致$F_0$和$F_1$暂停发送,$F_0$吞吐率受损。

受害流性能受损来源于PFC暂停帧带来的连锁反应,在基于PFC的无损网络中无法完全消除,但是在高效的缓存管理机制下可以有效避免。能否有效避免受害流性能损害在于缓存管理机制是否可以在不触发PFC的前提下完全吸纳突发流量。


\xsubsection{长距离传输存在丢包}{Packet Loss in Long-haul Transmission}

随着数据中心网络支持业务应用的不断发展,跨数据中心流量不断增长并且提出了更高的网络服务质量要求\cite{zhao2023deterministic}。对于分布式应用,网络成为其提供高质量和高可靠性服务的关键性能瓶颈,如分布式存储中跨计算集群和存储集群之间的网络通信。

RDMA通过将网络协议栈卸载到网卡降低CPU开销,可以在接近于0的CPU开销下实现高带宽和低时延网络传输。因此,RDMA技术在数据中心内部广泛部署。随着数据中心分布式应用相关技术的进一步发展,高带宽和超低时延的跨数据中心长距离传输成为其提升服务质量的关键。因此,跨数据中心传输对于RDMA技术的部署需求日益迫切。

图\ref{c4:s1:ss1:fig:data center typical topology}中显示了数据中心网络的经典拓扑结构。所有服务器通过Clos架构连通,数据中心内部采用三级交换架构,数据中心之间通过路由器和长距离链路连接。取决于数据中心的地理位置,长距离链路的长度可以达到数十甚至数百千米,使其传播时延增长到毫秒级别\cite{bai2023empowering}。

\begin{figure}[H]
  \centering
  \includegraphics[width=0.85\linewidth]{Figures/data_center_architecture.pdf}
  \caption{数据中心网络典型拓扑结构}
  \label{c4:s1:ss1:fig:data center typical topology}
\end{figure}

长距离传输对于缓存容量提出了更高的要求。最坏情况下的PFC净空缓存需求量计算如公式(\ref{eqn:c2:headroom calculation})所示,其大小与链路带宽和传播时延成正相关关系。长距离链路的传播时延给PFC净空缓存的预留带来了很大的挑战,长距离传输需要更大的净空缓存空间才能保证无丢包,如100Gbps带宽和$1ms$传播时延的长距离链路,H-SIH需要给每个队列预留的净空缓存量为$\eta=2(100\text{Gbps} \times 1ms + 1500\text{B})+\text{3840B}\approx 25\text{MB}$。因此,对于长距离传输中的PFC部署,纯片上缓存系统无法满足所有队列的净空缓存需求,需要异构缓存系统提供更大的缓存容量。

\begin{figure}[H]
  \centering
  \includegraphics[width=0.5\linewidth]{Figures/long-haul-transmission-motivation.pdf}
  \caption{H-SIH在不同链路时延下的丢包率}
  \label{c4:s1:ss1:fig:long haul packet loss}
\end{figure}


片外缓存可以保证足量的净空缓存分配,但是片外净空缓存的占用仍然受限于片外带宽。H-SIH在片外带宽瓶颈时触发全局流控可以保证无丢包是基于净空缓存需求总量小于片上缓存容量的假设,然而,随着PFC控制帧传输距离增加,需要的净空缓存总量不断增加,片外缓存空间无法完全避免净空缓存溢出而丢包。本节通过ns-3模拟图\ref{c4:s1:ss1:fig:data center typical topology}中的跨数据中心长距离传输场景,其中四台主机通过一个路由器连接,所有链路带宽100Gbps,背景流基于web \ search\cite{SIGCOMM10DCTCP}模型随机产生,流开始时间服从泊松分布,将链路负载控制到0.9;突发流形式为三打一,突发大小为8MB,目的主机随机挑选,每经过1$ms$产生一次突发,链路时延在1$us$-2$ms$范围内递增,图\ref{c4:s1:ss1:fig:long haul packet loss}显示了H-SIH不同链路时延下的丢包结果,结果显示H-SIH在长距离传输场景下存在丢包,而且随着传输距离增加,丢包问题变得更加严重。

H-SIH的在长距离传输场景中的丢包问题主要来源于净空缓存分配和片外带宽利用的不合理性。一方面,H-SIH在片内预留净空缓存,忽略了长距离传输场景下可能存在的净空缓存需求量大于片内缓存容量问题,在净空缓存分配时未能充分发挥片外缓存的容量优势;另一方面,H-SIH在片外带宽用尽时触发全局流量控制,触发全局流量控制会导致所有拥塞队列同时占用净空缓存,在净空缓存不足的情况下,丢包问题将进一步加剧。


\xsection{H-DSH机制设计}{H-DSH Design}
\label{c4:s3:hdsh design}

针对H-SIH存在的问题,本文提出了一种面向异构缓存系统的动态共享净空缓存分配机制H-DSH,H-DSH以片外缓存为中心,将片外缓存作为共享缓存空间进行动态分配,在进行净空缓存分配时优先利用片外缓存空间,以充分发挥片外缓存的容量优势。同时通过概率模型主动限制部分拥塞流量对片外缓存的占用,有效降低片外带宽瓶颈出现的概率。本节主要阐述H-DSH的设计目标、基本设计思想、面临的主要挑战以及具体设计细节。

\xsubsection{设计目标}{Design Goals}

考虑到异构缓存系统不同架构、片上和片外缓存不同物理特性以及部署环境流量和时延复杂性等因素,异构缓存系统需要设计一个更加灵活且高效的缓存管理策略,新的缓存管理策略需要同时满足以下特性:

% \subsubsection{避免丢包导致性能受损}
1)避免丢包导致性能受损

丢包带来的重传会导致RDMA性能受损,避免丢包同样是异构缓存系统净空缓存管理策略的首要目标。在缓存容量和读写带宽充足的情况下,新策略的目标在于分配足够的净空缓存以保证无损传输;在缓存容量或带宽受限的情况下,新策略的目标在于尽量减少丢包数量以降低网络性能损害。

% \subsubsection{提供更高突发吸纳能力}
2)提供更高突发吸纳能力

相较于数据中心内部网络环境,跨数据中心长距离传输场景中的流量表现出更强的突发性和不可预测性。为了确保突发流量的传输性能,异构缓存系统需要具备足够的容量和处理能力,有效降低突发流量对网络传输性能的影响。因此,新的净空缓存管理策略需要基于异构缓存系统为长距离传输场景提供足够的突发吸纳能力。

% \subsubsection{避免PFC损害线速吞吐}
3)避免PFC损害线速吞吐

PFC存在的头阻问题可能导致无辜流的暂停发送,进而导致正常流量吞吐率减小、时延增加和不公平性等一系列性能问题。线速吞吐是网络性能的重要指标之一,高性能网络传输需要无损缓存系统提供线速吞吐能力。新的净空缓存管理策略需要在利用PFC避免丢包的基础上避免PFC对线速吞吐造成传输性能损害。

% \subsubsection{发挥片外缓存的容量优势}
4)发挥片外缓存的容量优势

相较于片上缓存,片外缓存具有明显的容量优势。考虑到片外缓存带宽敏感,片外缓存更适合存储大容量需求且小带宽需求型流量。在带宽受限的条件下充分利用片外缓存容量是进一步提高异构缓存系统缓存管理效率的关键。因此,新策略需要实现片外缓存的高效利用。

% \subsubsection{提供长距离传输的扩展性}
5)提供长距离传输的扩展性

随着长距离传输网络服务质量需求的不断提高,进一步将RDMA扩展至跨数据中心范围成为新的发展趋势。由于长距离链路带来更大的传播时延,支持PFC需要更多的净空缓存空间。单纯利用片上缓存无法满足长距离传输的需求。新的净空缓存管理策略需要具备一定的传输距离扩展性,以应对RDMA扩展需求。


\xsubsection{基本思想}{Key Ideas}
\label{c4:s3:ss2:key ideas}

为实现上述目标,H-DSH在设计上主要遵循以下基本思想,通过高效的净空缓存分配方式充分发挥异构缓存的利用效率,同时结合流量识别机制和主动片外带宽调节机制提升无损传输性能:

\subsubsection{H-DSH以动态共享方式统一管理片上缓存和片外缓存}

对于带宽敏感的片外缓存而言,静态预留缓存空间意味着片外读写带宽的预留,所以片外缓存空间不适合进行静态分配。与传统缓存分区相统一,H-DSH将片外缓存作为共享缓存扩展空间,以动态方式分配共享缓存,同时将片外缓存的共享范围扩展到所有队列。在读写需求不超过片外提供带宽的情况下,H-DSH对于片外共享缓存空间的分配方式与片内共享缓存空间一致。另外,H-DSH通过动态共享的方式分配净空缓存空间,减少整个缓存系统的静态空间,利用统计复用的思想提高缓存利用率。

\subsubsection{H-DSH进一步结合流量特征信息进行缓存位置决策}

H-DSH在设计上充分考虑到片上缓存和片外缓存在容量和带宽上的差异。在缓存的定位上,片上缓存适用于存储带宽敏感容量不敏感型流量,片外缓存则适合存储容量敏感带宽不敏感型流量。具体地,片内缓存应该存储能够及时排出的流量,该类流量不会造成过长的队列积累,同时单位时间带宽需求\footnote{便于说明,此处定义流量的单位时间带宽需求为该流量在缓存中的读写数据量与持续时间的比值,用于表示流量的带宽敏感程度。}更大;相对地,片外缓存应该存储在缓存中停留时间长的流量。H-DSH在队列长度的基础上进一步提取流量特征信息为缓存位置决策提供指导。

\subsubsection{H-DSH通过提前触发PFC避免出现片外带宽瓶颈}

当片外带宽瓶颈出现时,片外缓存空间的进一步利用会受其限制,因此,提高片外缓存利用率的关键在于有效解决片外带宽瓶颈问题。当片外带宽资源紧张时,缓存管理机制可以通过触发PFC来缓解片外带宽压力,但是PFC生效需要一定的时间延迟,导致该处理措施具有一定的滞后性。为了克服PFC的滞后性,H-DSH通过在片外缓存中挑选部分拥塞队列,在片外带宽用尽之前提前触发其PFC暂停,降低片外带宽瓶颈发生的概率。

\subsubsection{H-DSH优先利用片外缓存空间分配共享净空缓存}

考虑到长距离传输不断增加的净空缓存容量需求与现实片上缓存容量受限之间的矛盾,通过静态预留净空缓存的方式完全避免丢包已经不再现实。H-DSH将净空缓存空间进一步划分为静态净空缓存和共享净空缓存,共享净空缓存优先从片外缓存分配;在片上缓存容量足够时,在片上为每个端口预留$\eta$大小的净空缓存,片上缓存容量不足时则尝试从共享缓存空间分配。

\xsubsection{主要挑战}{Main Challenges}

相对于在数据中心网络内部部署的片上缓存系统,异构缓存系统本身以及部署环境存在更加显著的异构性。一方面,异构缓存系统本身固有片上缓存和片外缓存异构性,缓存管理策略需要将片外带宽作为核心资源进行管理;另一方面,在异构缓存系统的部署场景中,长距离链路会带来明显的时延异构性,时延增加同时带来更多的净空缓存需求。这些对于缓存管理策略的设计提出了全新的挑战,异构缓存系统缓存管理策略的设计主要面临以下挑战:

\subsubsection{如何利用有限信息识别流量敏感性特征}
%有限可用资源下的流量敏感特征识别

网络流量具有很强的实时变化性和不可预测性,缓存管理策略无法实现准确的流量识别和实时的流量预测。有效的流量特征识别可以显著提高缓存管理的效率,如何利用有限的可用资源(包括硬件资源和流量信息等)实现更高准确度的流量识别是H-DSH实现高效性的重要挑战。根据\ref{c4:s3:ss2:key ideas}提出的基本思想,H-DSH至少需要实现队列级别的流量容量以及带宽敏感性特征识别。因此,H-DSH需要在有限的可用资源下实现对流量敏感性特征的迅速且准确识别,通过片上缓存和片外缓存队列长度等状态变化和特征提取有效信息。

\subsubsection{如何实现适度且有效的受害流量挑选}
%适度且有效的受害流量挑选

H-DSH通过提前触发部分拥塞流量的PFC缓解片外带宽压力,在进行受害流量挑选时需要适度且有效。一方面,过度挑选受害流量会损害其传输性能,同时影响片外带宽的利用效率,进而降低片外缓存的使用效率;另一方面,受害流量挑选不足或不合理则无法有效减小片外缓存读写带宽需求,在片外带宽用尽时可能导致严重的性能下降。另外,被挑选流量的队列状态对于传输性能也会产生影响,如暂停短队列可能导致其中流量吞吐率受损。因此,H-DSH需要权衡以上各点实现适度且高效的受害流量挑选策略。

\subsubsection{如何在净空缓存预留不足时避免丢包}

PFC的净空缓存需求量与链路时延正相关,在长距离传输环境中支持PFC需要预留更多的净空缓存空间,容量敏感的片上缓存空间无法满足不断增长的净空缓存需求。因此,如何在净空缓存预留不足的情况下避免丢包成为异构缓存系统缓存管理策略面临的新挑战。在片上缓存系统中,无损传输和长距离传输具有明显的互斥关系,异构缓存系统可以解决长距离传输的高净空缓存容量需求,但是片外缓存的带宽限制会带来丢包风险,即片外的净空缓存空间可能因片外带宽不足而无法使用,从而导致丢包。因此,H-DSH在设计上需要进一步考虑净空缓存需求超出片上缓存容量的场景,通过高效的缓存分配与管理机制进一步扩展传输距离。


\xsubsection{具体机制}{Design Details}

本节描述H-DSH相关机制的具体设计和实现思路。包括H-DSH的缓存分区结构、缓存分配与管理、流量控制、流量敏感性识别以及MMU处理流程。

\subsubsection{缓存分区结构}

\begin{figure}[H]
  \centering
  \includegraphics[width=0.65\linewidth]{Figures/buffer_partition_hdsh.pdf}
  \caption{H-DSH缓存分区}
  \label{c4:s1:ss1:fig:hdsh buffer partition}
\end{figure}

% \todo{缓存分区图过大}

H-DSH的缓存分区结构如图\ref{c4:s1:ss1:fig:hdsh buffer partition}所示,整体上与传统缓存分区相统一。片上缓存分区结构与H-SIH基本一致,划分为私有缓存、共享缓存和保底净空缓存池三部分。区别在于片外缓存全部划分为共享缓存,作为片上共享缓存分区在片外的扩展空间,在所有队列之间共享,保底净空缓存池为所有端口全局预留。

H-DSH将片外缓存作为共享缓存的扩展空间,且分配方式与片上共享缓存保持一致。这样设计主要有两方面原因:一方面,H-DSH旨在将所有的静态缓存空间预留在片上,由于片外缓存带宽受限,在片外静态预留缓存空间无法保证其可用性;另一方面,将片外缓存作为共享缓存分区的一部分可以降低缓存管理策略的复杂性,不考虑片外带宽瓶颈的情况下,片外共享缓存的分配与管理与片上共享缓存并无差异。

% \todo{保底净空缓存池}

\subsubsection{缓存分配}

与片上缓存系统类似,H-DSH仍然在片上为每个队列静态预留少量私有缓存空间,同时在可用的片上缓存容量内尝试为每个端口静态预留$\eta$大小的净空缓存作为全局净空缓存池,全局净空缓存池用于吸收触发端口级别流量控制后到达的流量。值得注意的是,对于逐端口净空缓存需求总量超出片上缓存容量的情况,H-DSH仅在片上预留适量静态净空缓存池并使其在所有端口之间共享。

对于共享缓存空间,H-DSH根据需要为每个队列动态分配。共享缓存空间包括片上共享缓存和片外共享缓存两部分,考虑到片上缓存和片外缓存的不同物理特性,H-DSH使用不同的策略限制每个队列在其中的可用空间。

对于容量敏感的片上共享缓存,H-DSH使用动态阈值$T(t)$限制其占用,$T(t)$计算方式与H-SIH相同,即公式(\ref{eqn:c4:dt threshold in hybrid buffer system})。区别在于H-DSH利用该阈值限制片上和片外共享缓存队列长度之和,如下式(\ref{eqn:c4:dt threshold limit})所示:
\begin{equation}
  \sum_{i} \sum_{j} \omega_{s}^{i,j}(t) < T(t)
  \label{eqn:c4:dt threshold limit}
\end{equation}

\noindent{其中,$ω_s^{i,j}(t)$为端口$i$队列$j$的片上共享缓存和片外共享缓存占用的总和,其中蕴含的原理为H-DSH利用片内和片外队列总长度识别拥塞队列,在队列拥塞时限制占用片上缓存空间。}

对于容量不敏感的片外共享缓存,在片外带宽压力较小时,H-DSH仅通过静态阈值限制其占用,静态阈值的设置可避免队列长度过长即可,如16MB;在片外带宽压力增加到一定程度时,H-DSH开始在片外挑选受害队列,通过提前触发其PFC缓解带宽压力,每个队列以一定概率触发PFC。为了在规避片外带宽瓶颈的同时避免引入性能损害,H-DSH进一步结合队列长度和片外带宽使用情况计算每个队列的PFC触发概率,具体地,片外带宽使用程度超过一定阈值后(如60\%),队列触发PFC的概率正比于队列长度和片外带宽使用率。

片上预留的保底净空缓存空间只有在触发端口级别流控时才开始占用。对于队列级别流控,H-DSH在其触发时动态地为队列分配共享净空缓存空间。共享净空缓存的分配和管理方式与普通共享缓存相同,仅存在用途上的差别。

\subsubsection{流量控制}

与\ref{c3:s3:ss4:dsh mechanisms}节中的流量控制描述类似,H-DSH同样通过结合队列级别流量控制和端口级别流量控制减少净空缓存预留量。相对于片上缓存系统,异构缓存系统中缓存占用情况更复杂,因而H-DSH中流量控制的阈值设置复杂度更高。

对于队列级别流量控制,其状态转换过程与PFC机制类似,差别在于暂停和恢复阈值的设置。H-DSH中触发队列级别暂停的条件包括缓存容量受限和缓存带宽受限两种情况:缓存容量受限即共享缓存占用达到阈值限制(包括片上和片外缓存),此时队列拥塞程度较大,通过触发PFC限制其队列长度;缓存带宽受限即片外带宽占用超过阈值,此时片外带宽压力较大,通过提前触发部分队列的PFC避免片外带宽用尽。队列级别恢复的要求即尽量不影响该队列的吞吐率,与片上缓存系统相统一,H-DSH将$X_{qon}$阈值设置为:
\begin{equation}
  X_{qon}(t) = T(t) - \delta_q
  \label{eqn:c4:qon threshold}
\end{equation}

\noindent{其中$T(t)$即片上共享缓存的动态阈值,需要注意的是,该阈值限制的是共享缓存空间占用,即片上和片外共享缓存之和低于该阈值时触发PFC恢复,其中蕴含的原理包含两方面:一方面,将$T(t)$作为队列流控恢复阈值可以保证发送恢复帧时有足够的队列长度,从而可以保证恢复帧生效时延内的吞吐率;另一方面,通过共享缓存总占用量与恢复阈值比较可以防止PFC恢复时片外队列较长,避免过早PFC恢复导致拥塞加重。}

理想情况下,端口级别流量控制应该作为队列级别流控的备选措施。在片外带宽压力较小时,片外缓存足够支持队列级别流量控制净空缓存占用,此时无需触发端口级别流量控制;在片外带宽压力较大且端口拥塞程度较高时需要触发端口级别暂停。具体地,H-DSH在挑选受害流量触发其队列级别流量控制时,进一步检查该队列所在端口的拥塞程度,如果端口内拥塞程度较高则触发端口级别暂停。因此,H-DSH将端口暂停阈值$X_{\mathit{poff}}$设置为:
\begin{equation}
  X_{\mathit{poff}}(t) = N_q \times T(t)
  \label{eqn:c4:poff threshold}
\end{equation}

\noindent 其中$N_q$为端口中的队列数,与公式(\ref{eqn:c4:qon threshold})中$X_{qon}$阈值不同,该阈值限制的是该端口中所有队列的片上共享缓存空间占用,即在该端口中所有队列片上共享缓存占用总和超过阈值$X_{\mathit{poff}}$时触发整个端口的暂停帧。与端口级别暂停阈值的设置方式同理,端口级别恢复阈值设置为:
\begin{equation}
  X_{pon}(t) = X_{\mathit{poff}}(t) - \delta_p
  \label{eqn:c4:pon threshold}
\end{equation}


\subsubsection{流量敏感性识别}

\begin{figure}[H]
  \centering
  \includegraphics[width=0.6\linewidth]{Figures/state_transition_flow.pdf}
  \caption{流量敏感性识别状态机}
  \label{c4:s1:ss1:fig:flow state transition}
\end{figure}

为了充分发挥片上和片外缓存各自的物理特性优势,H-DSH根据流量特征基于报文进行分类。对应于片上和片外缓存定位,H-DSH将报文划分为带宽敏感型和容量敏感型。带宽敏感型报文对于缓存容量需求较小,但在短时间内的带宽需求较大,该类型报文适合在片上缓存存储,具体包括队首、短队列和微突发中能够及时转发的报文,以及所在队列快速排空的报文;容量敏感型报文单位时间带宽需求较小,但缓存积累量较大,以此更适合在片外缓存存储,具体包括净空缓存、队尾以及被暂停队列中缓存停留时间长的报文。基于以上认识,H-DSH针对异构缓存系统提出一个流量敏感性识别机制,该机制可由图\ref{c4:s1:ss1:fig:flow state transition}中的状态机描述。

% \todo{状态机解释}

\subsubsection{MMU处理流程}

综合以上机制,H-DSH的具体工作流程如图\ref{c4:s3:ss4:fig:hdsh ingress flow control}的状态机所描述。在入口侧,每个入口队列存在两种状态:

(1)QON:表示该入口队列处于非拥塞状态。在该状态下,上游端口允许向该队列发送对应类别的流量,此时收到的数据包将会被存放到私有缓存空间或者共享缓存空间中。

(2)QOFF:表示该入口队列处于拥塞状态。在该状态下,上游端口对应类别的流量被暂停发送,收到的数据包将会被存放到共享缓存中,即共享净空缓存。

\begin{figure}[H]
  \begin{subfigure}[b]{0.49\linewidth}
      \centering
      \includegraphics[width=7cm]{state_transition_pfc_in_queue_hdsh.pdf}
      \subcaption{队列级别流量控制}
      \label{c4:s3:ss4:fig:sub1h:dsh ingress queue state transition}
  \end{subfigure}
  \begin{subfigure}[b]{0.49\linewidth}
      \centering
      \includegraphics[width=7cm]{state_transition_pfc_in_port_hdsh.pdf}
      \subcaption{端口级别流量控制}
      \label{c4:s3:ss4:fig:sub2:hdsh ingress port state transition}
  \end{subfigure}
  \caption{H-DSH出口侧流量控制状态机}
  \label{c4:s3:ss4:fig:hdsh ingress flow control}
\end{figure}

每个入端口同样存在两种端口级别的状态:

\setcounter{paragraph}{0}
(1)PON:表示该入端口处于非拥塞状态。在该状态下,上游端口允许向其发送任意类别的流量(对应类别没有被队列级别流量控制暂停),此时收到的数据包将会被存放到私有缓存或者共享缓存分区中。

(2)POFF:表示该入端口处于拥塞状态。在该状态下,上游端口所有类别的流量均被暂停发送,此时收到的数据包将会被存放到保险净空缓存中。

出口侧的具体工作流程同图\ref{c3:s3:ss4:fig:dsh egress flow control},状态转换过程完全相同,此处不再展开赘述。

具体地,初始队列长度为0时,入口队列处于QON状态。后续报文到达时可能存在三种状态:\ding{172}PON$+$QON,\ding{173}PON$+$QOFF,\ding{174}POFF。根据所处状态和缓存占用情况进行对应处理,MMU按照顺序进行判断,只有在前面条件均不满足的情况下才会进行后续判断。对于状态\ding{172},此时所在队列和端口均处于ON状态,端口和队列级别暂停均未触发:

\setcounter{paragraph}{0}

(1)$q^{i,j}_{\text{on}}(t)<\phi$:MMU决策该报文进入私有缓存空间,即片上队列长度小于私有缓存大小,优先占用私有缓存空间。

(2)$q^{i,j}(t)< \phi+T(t)$:MMU决策该报文进入片上共享缓存空间。即当前队列共享缓存占用不超过片上共享缓存动态阈值,占用片上共享缓存空间。

(3)$bw_{\text{used}} \geqslant \beta \cdot bw ~\bigvee ~q_{\text{off}}(t) \geqslant B_{\text{off}}$:MMU决策该报文进入片内净空缓存空间。此时片外带宽或容量用尽,触发端口级别暂停以避免丢包。

(4)$bw_{\text{used}} \geqslant \beta_{c} \cdot bw$:MMU决策该报文进入片外共享缓存空间,同时通过概率触发其队列级别暂停,若触发队列级别暂停则进一步判断是否触发端口级别暂停,即公式(\ref{eqn:c4:poff threshold})。其中$\beta_c \cdot bw$为片外带宽过载阈值。

(5)$q^{i,j}_{\text{off}}(t) < T_{\text{off}}$:MMU决策该报文进入片外共享缓存空间。即片外共享缓存占用未达到阈值限制,占用片外共享缓存空间。

对于状态\ding{173},此时端口处于PON状态,队列处于QOFF状态,队列级别暂停触发:

\setcounter{paragraph}{0}

(1)$bw_{\text{used}} < \beta \cdot bw ~\bigwedge ~q_{\text{off}}(t) \geqslant B_{\text{off}}$:MMU决策该报文进入片外共享缓存空间。即片外共享缓存空间可用,优先在片外分配共享净空缓存。

(2)$q_{\text{on}}^{i,j}(t)< \phi+T(t)$:MMU决策该报文进入片上共享缓存空间。此时片上共享缓存存在可用空间,则在片上分配共享净空缓存。

(3)$q_{\text{on}}^{i,j}(t) \geqslant \phi+T(t)$:MMU决策该报文进入片上净空缓存空间,同时触发端口级别暂停。

对于状态\ding{174},此时端口处于POFF状态,端口级别暂停触发:

\setcounter{paragraph}{0}

(1)$q_{h}(t) \leqslant H$:MMU决策该报文进入片上净空缓存空间,其中$q_{h}(t)$为所有端口的净空缓存占用总量,$H$为净空缓存池大小。

(2)$q_{h}(t) > H$:MMU决策将该报文丢弃。即净空缓存溢出,报文丢弃。


\xsection{H-DSH具体实现}{H-DSH Implementation}
\label{c4:s4:hdsh implementation}

本节讨论H-DSH的硬件资源和软件算法上的易实现性。相较于H-SIH,H-DSH增加的硬件资源开销可以接受,H-DSH的算法实现大多为简单比较逻辑,在软件上易于实现。


\xsubsection{流量控制实现}{Implementation of Flow Control}

关于队列级别流量控制和端口级别流量控制以及二者的整合具体实现,\ref{c3:s5:dsh implementation}节已经详细描述,本节不再展开赘述。其中,队列级别流控仅修改控制帧触发状态即可,端口级别流控需要引入一个乘法器和一个减法器用于阈值计算,整合二者需要在出口为每个队列和端口维护一个状态,每个队列的暂停状态通过一个与或门即可获得。


\xsubsection{额外资源开销}{Additional Resource Overhead}

相较于H-SIH,H-DSH引入的额外资源开销在可接受范围内。H-DSH引入的额外资源开销包括运算器、寄存器和内存。其中,运算器开销来源于流量控制和缓存决策时的阈值比较,H-DSH仅引入少量比较和简单运算操作;寄存器开销在于实时统计片外带宽使用情况,需要增加一个寄存器存储当前的带宽占用,为了避免引入计时器,H-DSH每存储一定数量报文后重新计算其大小;内存开销主要在于维护每个队列在片外缓存的长度,额外内存开销不超过2MB。另外,对于H-DSH中概率触发队列暂停的实现,不需要引入硬件随机数生成器,而是通过软件算法生成随机数。综上,H-DSH在仅引入少量额外资源开销,在硬件上具有可实现性。

\xsubsection{详细算法实现}{H-DSH Algorithm}

H-DSH在算法实现上可以划分为两个模块:入队模块和出队模块。入队模块负责报文入队之前的决策以及流量控制Pause触发,出队模块负责报文出队后的缓存释放以及流量控制Resume触发。

\begin{algorithm}[H]
  \small
  \SetAlCapFnt{\small}
  \SetAlCapNameFnt{\small}
  \Input{待处理报文$packet$}
  \Output{存储位置决策$decision$}

  \SetAlgoVlined
  \newcommand{\algorithmicgoto}{\textbf{go to}}
  \newcommand{\Goto}[1]{\algorithmicgoto~\ref{#1}}

  获取报文大小$pkt\_size$,入端口$pid$,队列$qid$\;
  \uIf{$port\_status[pid]=OFF$}{\label{poff}
    \uIf{$headroom\_left \geqslant pkt\_size$}{
      $decision \leftarrow TO\_HEADROOM$\;
    }
    \Else{
      $decision \leftarrow DROP$\;
    }
  }
  \uElseIf{$queue\_status[pid][qid]=OFF$}{\label{qoff}
    \uIf{$dram\_left \geqslant pkt\_size \And bw\_used < bw$}{
      $decision \leftarrow TO\_OFF\_SHARED$\;
    }
    \uElseIf{$pkt\_size + sram\_qlen[pid][qid] <= \varphi + T(t)$}{
      $decision \leftarrow TO\_ON\_SHARED$\;
    }
    \Else{
      $port\_status[pid] \leftarrow OFF$\;
      向上游$pid$端口发送端口暂停帧\;
      \Goto{poff};
    }
  }
  \uElseIf{$pkt\_size+sram\_qlen[pid][qid]<\varphi$}{
    $decision \leftarrow TO\_RESERVED$\;
  }
  \uElseIf{$pkt\_size+shared\_bytes[pid][qid]<T(t)$}{
    $decision \leftarrow TO\_ON\_SHARED$\;
  }
  \uElseIf{$dram\_left < pkt\_size \Or bw\_used \geqslant bw \Or dram\_qlen[pid][qid] \geqslant T_d$}{
    $queue\_status[pid][qid] \leftarrow OFF$\;
    向上游$pid$端口$qid$队列发送队列暂停帧\;
    \Goto{qoff}\;
  }
  \uElseIf{$bw\_used \geqslant cg\_bw \And \textnormal{CheckSelectedPause}(pid, qid)$}{
    \uIf{$sram\_port\_shared[pid] < N_q \times T(t)$}{
      $queue\_status[pid][qid] \leftarrow OFF$\;
      向上游$pid$端口$qid$队列发送队列暂停帧\;
      \Goto{qoff}\;
    }
    \Else{
      $port\_status[pid] \leftarrow OFF$\;
      向上游$pid$端口发送端口暂停帧\;
      \Goto{poff}\;      
    }
  }
  \Else{
    $decision \leftarrow TO\_OFF\_SHARED$\;
  }
  \Return $decision$\;
  \caption{MMU入队模块决策算法}
  \label{alg:c4:s4:ss3:enqueue module algorithm}
\end{algorithm}

\subsubsection{入队模块}
入队模块算法由MMU在报文入队前触发,主要处理过程包括报文缓存决策和流量控制触发。算法\ref{alg:c4:s4:ss3:enqueue module algorithm}描述了H-DSH入队模块决策算法的伪代码。具体地,对于每个新到达报文首先获取其大小、入端口和入口队列(第1行),然后判断当前端口是否处于暂停状态,如果处于暂停状态且净空缓存空间足够则进入全局的静态净空缓存池中(第3-4行),净空缓存溢出则丢弃当前报文(第6行)。净空缓存不是按照最坏情况给每个端口预留,而是预留一个全局的净空缓存池,所有端口共享其空间,在预留总量少于所有端口净空缓存需求量总和时可能由于净空缓存溢出而丢包。如果当前端口不处于暂停状态则进一步查看入口队列是否处于暂停状态,在入口队列暂停状态下,MMU需要为该队列分配共享净空缓存,在片外容量和带宽足够时首先尝试在片外分配(第8-9行),片外无法分配则从该队列的片内共享缓存空间分配(第10-11行),片内和片外均无法完成分配时触发端口级别暂停,存储位置在转入端口暂停状态后进一步决策(第13-14行)。在端口和队列都处于非暂停状态时,首先查看私有缓存空间是否可用,可用则优先占用私有缓存(第16行)。不可用则检查片上共享缓存空间,可用则优先占用片内共享缓存空间(第18行),不可用则进一步决策片外共享缓存空间,在片外容量或带宽可用时必然触发队列级别暂停(第20行),在片外带宽压力较大时概率触发队列级别暂停(第24行),同时判断当前端口的片上共享缓存占用,超过阈值则进一步触发端口级别暂停(第27行),触发流量控制之后的缓存位置由对应状态处理逻辑进行决策(第21,25,28行)。片外容量和带宽均不受限的情况下决策进入片外共享缓存空间(第30行)。

\subsubsection{出队模块}
出队模块算法在报文发送出队之后触发,出队模块根据缓存占用情况决定出队报文缓存分区,更新对应计数器释放缓存空间,同时检测是否需要触发流量控制恢复。算法\ref{alg:c4:s4:ss4:dequeue module algorithm}描述了H-DSH出队模块处理算法的伪代码。对于每个出队报文首先获取其大小、入端口、入口队列和存储位置(第1行),其中存储位置$location$表示报文存储在片上缓存还是片外缓存。对于片外缓存,其与片上缓存物理隔离,报文出队时更新片外相关计数器(第3-4行);对于片上缓存,MMU维护每个缓存分区的缓存占用的计数器,并不记录每个报文所在的缓存分区,出队模块进行的缓存释放操作为逻辑上的缓存释放,即更新对应分区的计数器。出队时缓存分区释放的顺序依次为:净空缓存$\rightarrow$共享缓存$\rightarrow$私有缓存,出队模块依次计算上述三个分区的出队字节数并更新对应的计数器(第7-8行,第10-12行,第14-16行)。报文出队后会导致队列长度减小或者暂停阈值增加,此时进行端口级别和队列级别的恢复检测。出队模块首先对每个处于暂停状态的端口进行检查,如果净空缓存排空且端口共享缓存占用低于$X_{pon}$阈值则向上游设备发送$pid$对应端口的恢复帧同时恢复PON状态(第19-21行);之后进一步检查每个PON状态端口中的QOFF状态队列,如果共享缓存占用低于$X_{qon}$阈值则向上游设备发送$qid$对应队列的恢复帧。


\begin{algorithm}[H]
  \small
  \SetAlCapFnt{\small}
  \SetAlCapNameFnt{\small}
  \Input{待处理报文$packet$}

  \SetAlgoVlined

  获取报文大小$pkt\_size$,入端口$pid$,队列$qid$和存储位置$location$\;
  \If{$location=OFF\_CHIP\_BUFFER$}{
    $dram\_qlen[pid][qid] \leftarrow dram\_qlen[pid][qid] - pkt\_size$\;
    $shared\_bytes[pid][qid] \leftarrow shared\_bytes[pid][qid] - pkt\_size$\;
  }
  \Else{
    \Comment{更新净空缓存占用计数器}
    $from\_headroom \leftarrow \min(headroom\_bytes[pid], pkt\_size)$\;
    $headroom\_bytes[pid] \leftarrow headroom\_bytes[pid] - from\_headroom$\;

    \Comment{更新共享缓存占用计数器}
    $from\_shared \leftarrow \min(sram\_shared[pid][qid], pkt\_size-from\_headroom)$\;
    $sram\_shared[pid][qid] \leftarrow sram\_shared[pid][qid] - from\_shared$\;
    $shared\_bytes[pid][qid] \leftarrow shared\_bytes[pid][qid] - from\_shared$\;

    \Comment{更新私有缓存占用计数器}
    $left\_bytes \leftarrow pkt\_size-from\_headroom-from\_reserved$\;
    $from\_reserved \leftarrow \min(reserved\_bytes[pid][qid], left\_bytes)$\;
    $reserved\_bytes[pid][qid] \leftarrow reserved\_bytes[pid][qid] - from\_reserved$\;
  }

  \Comment{检查端口级别流控恢复}
  \ForAll{$port\_status[p]=OFF$}{
    \If{$headroom\_bytes[q] \leqslant 0 \And sram\_shared[p] < N_q \times T(t) - \delta_p$}{
      $port\_status[p] \leftarrow ON$\;
      向上游设备发送端口$p$的恢复帧\;
    }
  }
  \Comment{检查队列级别流控恢复}
  \ForAll{$port\_status[p]=ON \And queue\_status[p][q]=OFF$}{
    \If{$shared\_bytes[p][q]<T(t)-\delta_q$}{
      $queue\_status[p][q] \leftarrow ON$\;
      向上游设备发送端口$p$队列$q$的恢复帧\;
    }
  }  

  \caption{MMU出队模块处理算法}
  \label{alg:c4:s4:ss4:dequeue module algorithm}
\end{algorithm}


\xsection{H-DSH性能测试}{H-DSH Evaluation}
\label{c4:s5:hdsh evaluation}

本节通过在ns-3仿真平台测试H-DSH的性能,实验结果总结如下,与现有机制H-SIH相比:

1)H-DSH对片外缓存的突发吸纳量提升超过3倍。

2)H-DSH可以有效减少PFC触发,避免流量吞吐受损。

3)H-DSH最高可以将平均流完成时间减少14.1\%。

4)H-DSH对无损传播距离的扩展超过4倍。

5)H-DSH在不同流量模式和不同拥塞控制算法下均可以有效减少平均FCT。

\xsubsection{网络和参数配置}{Network and Parameter Configuration}

测试环境均采用图\ref{c4:s1:ss1:fig:star topology}所示的星型拓扑,所有主机通过一个转发设备连接,其中左侧主机作为发送端,右侧主机作为接收端,整个网络拓扑中链路带宽配置为100Gbps,链路传播时延主要配置为1$us$,此时基准RTT为4$us$,$\eta=57640$B。转发设备每个端口支持8个队列,其中队列0为最高优先级,用于存储ACK确认和PFC暂停/恢复帧等控制报文,剩余7个队列之间采用DWRR调度策略。转发设备R部署异构缓存系统,片上缓存容量24MB,片外缓存容量10GB,片外缓存读写总带宽2Tbps。

\begin{figure}[H]
  \centering
  \includegraphics[width=0.6\linewidth]{Figures/star_topology.pdf}
  \caption{测试环境拓扑结构}
  \label{c4:s1:ss1:fig:star topology}
\end{figure}

缓存管理机制的参数按照如下配置:其中私有缓存大小为3072B,片上缓存动态阈值$\alpha$均配置为1,片外缓存静态阈值大小为16MB,恢复阈值偏移量$\delta_p$和$\delta_q$均设置为0。另外,对于H-SIH,净空缓存空间为每个队列预留$\eta$大小,32个端口下的净空缓存总预留量为$\eta \times N_p \times N_q = 57640\text{B} \times 32 \times 8 \approx 14\text{MB}$,片外带宽占用阈值$\beta = 0.9$;对于H-DSH,净空缓存空间为每个端口预留$\eta$大小,32个端口需要预留的总量为$\eta \times N_p= 57640\text{B} \times 32 \approx 1.75\text{MB}$,片外带宽占用阈值$\beta_f = 0.9$,片外带宽过载阈值$\beta_c = 0.7$。相关参数配置总结如图\ref{fig:c4:parameter setting}所示。

\begin{figure}[H]
  \begin{table}[H]
      \begin{tabularx}{\textwidth}{YYYY}
      \toprule
          \textbf{参数} & \textbf{具体配置} & \textbf{参数} & \textbf{具体配置} \\
      \midrule
          片上缓存容量 & 24MB & 端口支持队列数 & 8 \\
          片外缓存容量 & 10GB  & DT控制参数$\alpha$ & 1\\
          片外缓存总带宽 & 2Tbps & 片外缓存静态阈值 & 16MB \\
          $X_{qon}$偏移量$\delta_q$ & 0 & 片外带宽占用阈值 & 0.9 \\
          $X_{pon}$偏移量$\delta_p$ & 0 & 片外带宽拥塞阈值 & 0.7 \\
      \bottomrule
      \end{tabularx}
  \end{table}
  \caption{交换芯片相关配置}
  \label{fig:c4:parameter setting}
\end{figure}

% \begin{tcolorbox}[height=6cm,colback=black!5!white,colframe=blue!75!black]
%   参数配置总结表格
% \end{tcolorbox}

% \todo{参数配置总结表格}

\xsubsection{突发吸纳提升}{Burst Absorption Imporvement}
\label{c4:s5:ss2:off-chip buffer utilization}

为了验证H-DSH对于片外缓存的利用效率,本节设计实验测试H-DSH在不同突发大小下的片外缓存使用。拓扑中包括发送端和接收端主机各16个,其中背景流采用web \ search\cite{SIGCOMM10DCTCP}模式,流发送时间服从泊松分布,发送端和接收到随机挑选,将链路负载控制到0.9;突发流形式为十六打一,由16个源主机同时发往同一个目的主机,目的主机和流量类别随机产生,连续两次突发间隔时间100$us$。在不同的突发大小下,H-SIH和H-DSH的片外缓存队列长度如图\ref{c4:s1:ss1:fig:burst absorption}所示。实验结果表明,相对于H-SIH,H-DSH可以将片外缓存的突发吸纳量增加超过3倍。

\begin{figure}[H]
  \centering
  \includegraphics[width=0.5\linewidth]{Figures/burst-absorption.pdf}
  \caption{不同突发大小下的片外缓存队列长度}
  \label{c4:s1:ss1:fig:burst absorption}
\end{figure}

\xsubsection{PFC触发避免}{PFC Avoidance}
\label{c4:s5:ss2:pfc avoidance hdsh}

为了验证H-DSH的PFC避免能力,本节首先在与\ref{c4:s5:ss2:off-chip buffer utilization}节相同的实验环境下进行测试,不同背景流负载下的PFC暂停时长如图\ref{c3:s3:ss4:fig:sub1:web flow pfc avoidance}所示,实验结果表明,相对于H-SIH,H-DSH可以有效减少背景流的PFC触发。在0.9的背景流负载下进一步测试H-DSH对突发流的PFC避免效果,不同突发大小下的PFC暂停时长如图\ref{c3:s3:ss4:fig:sub1:incast flow pfc avoidance}所示,实验结果表明,相对于H-SIH,H-DSH同样可以有效避免突发流的PFC触发,同时随突发大小不断增加,H-DSH的PFC避免效果更显著。

\begin{figure}[H]
  \begin{subfigure}[b]{0.49\linewidth}
      \centering
      \includegraphics[height=5.3cm]{Figures/pfc-avoidance-web.pdf}
      \subcaption{不同背景流负载下的暂停时长}
      \label{c3:s3:ss4:fig:sub1:web flow pfc avoidance}
  \end{subfigure}
  \begin{subfigure}[b]{0.49\linewidth}
      \centering
      \includegraphics[height=5.3cm]{Figures/pfc-avoidance-incast.pdf}
      \subcaption{不同突发大小下的暂停时长}
      \label{c3:s3:ss4:fig:sub1:incast flow pfc avoidance}
  \end{subfigure}
  \caption{PFC避免}
  \label{c3:s3:ss4:fig:hdsh pfc avoidance}
\end{figure}

\xsubsection{PFC损害消除}{Collateral Damage Avoidance}

PFC触发可能导致无辜流吞吐率受损,本节通过模拟图\ref{c4:s1:ss1:fig:victim flow scenario}中的受害流场景验证H-DSH的PFC损害避免能力。其中突发流和背景流$F_1$的瓶颈链路位于$R_1-H_{18}$,发生拥塞后会触发其上行链路的PFC暂停,即$R_1$向其上行链路发送暂停帧,进而拥塞传播到$R_0$触发$R_0$向上游发送暂停帧,尽管$F_0$的转发路径不包括瓶颈链路$R_1-H_{18}$,仍然受PFC损害成为受害流。在PFC和PFC+TCP New-Reno拥塞管理机制下测试H-DSH性能,$F_0$的吞吐率结果如图\ref{c3:s3:ss4:fig:throughput damage avoidance}所示。相较于H-SIH,H-DSH可以有效避免受害流$F_0$的吞吐性能损害,这得益于H-DSH的高效缓存分配策略可以有效避免PFC触发。

\begin{figure}[H]
  \begin{subfigure}[b]{0.49\linewidth}
      \centering
      \includegraphics[height=5.3cm]{Figures/collateral-damage-wo.pdf}
      \subcaption{受害流$F_0$吞吐率(w/o CC)}
      \label{c3:s3:ss4:fig:sub1:w/o cc f0 throughput}
  \end{subfigure}
  \begin{subfigure}[b]{0.49\linewidth}
      \centering
      \includegraphics[height=5.3cm]{Figures/collateral-damage-TcpNewReno.pdf}
      \subcaption{受害流$F_0$吞吐率(TCP New-Reno)}
      \label{c3:s3:ss4:fig:sub1:new reno f0 throughput}
  \end{subfigure}
  \caption{PFC损害消除}
  \label{c3:s3:ss4:fig:throughput damage avoidance}
\end{figure}

\xsubsection{流传输性能改善}{FCT Imporvement}
\label{c4:s5:ss4:fct improvement}

为了显示H-DSH对于网络流量整体传输性能的改善,本节采用与\ref{c4:s5:ss2:off-chip buffer utilization}节相同网络拓扑和流量模式,突发总大小配置为8MB,进一步测试H-SIH和H-DSH下的流完成时间,FCT结果如图\ref{c4:s1:ss1:flow fct}所示。结果显示,相对于H-SIH,H-DSH可以显著减少平均FCT,而且随着网络负载增加改善效果更加明显,这得益于H-DSH在片外带宽压力较大时的流量挑选,通过提前暂停部分队列避免片外带宽用尽导致严重性能损害。

\begin{figure}[H]
  \centering
  \includegraphics[width=0.5\linewidth]{Figures/fct-evaluation.pdf}
  \caption{不同背景流负载下的流完成时间}
  \label{c4:s1:ss1:flow fct}
\end{figure}

\xsubsection{复杂场景适应性}{Complex Network Scenes Adaptability}
本节进一步验证H-DSH在复杂网络场景下的适应性,主要涉及H-DSH在不同传输距离、流量模式和拥塞控制机制下的扩展性。

\subsubsection{流量模式}

\begin{figure}[H]
  \begin{subfigure}[b]{0.47\linewidth}
      \centering
      \includegraphics[width=\linewidth]{Figures/traffic-pattern-cdf.pdf}
      \subcaption{不同模式流量分布}
      \label{c3:s6:ss1:fig:sub1:traffic pattern cdf}
  \end{subfigure}
  \begin{subfigure}[b]{0.47\linewidth}
      \centering
      \includegraphics[width=\linewidth]{Figures/fct-evaluation-pattern-cache.pdf}
      \subcaption{Cache}
      \label{c3:s6:ss1:fig:sub1:fct cache}
  \end{subfigure}
  \begin{subfigure}[b]{0.47\linewidth}
    \centering
    \includegraphics[width=\linewidth]{Figures/fct-evaluation-pattern-hadoop.pdf}
    \subcaption{Hadoop}
    \label{c3:s6:ss1:fig:sub1:fct hadoop}
  \end{subfigure}
  \begin{subfigure}[b]{0.47\linewidth}
    \centering
    \includegraphics[width=\linewidth]{Figures/fct-evaluation-pattern-mining.pdf}
    \subcaption{Data Mining}
    \label{c3:s6:ss1:fig:sub1:fct mining}
  \end{subfigure}  
  \caption{不同流量模式下的流完成时间}
  \label{c3:s6:ss1:fig:fct different pattern}
\end{figure}

web search流量是网络中主要流量形式之一\cite{SIGCOMM10DCTCP},\ref{c4:s5:ss4:fct improvement}节验证了H-DSH在该流量模式下的传输性能提升。本节在相同网络拓扑和配置下进一步测试H-DSH对于其它流量模式的适应性,包括Data Mining\cite{SIGCOMM09VL2}、Hadoop\cite{SIGCOMM15FB}和Cache\cite{SIGCOMM15FB},上述流量模式中流量分布特点如图\ref{c3:s6:ss1:fig:sub1:traffic pattern cdf}所示。不同流量模式下的FCT结果如图\ref{c3:s6:ss1:fig:fct different pattern}所示,结果表明H-DSH对于不同流量模式具有良好的适应性,在不同流量模式下均有不同程度FCT改善。

\subsubsection{传输距离}

针对异构缓存系统在跨数据中心长距离传输网络场景下的应用需求,本节基于ns-3模拟长距离传输网络场景,进一步测试H-DSH的传输距离扩展性。考虑到跨数据中心转发设备端口数需求通常较小,本节采用的星型拓扑中转发设备仅连接8个主机,链路带宽设置为100Gbps,为保证异构缓存系统收敛性,片外缓存读写总带宽设置为200Gbps,全局净空缓存池大小设置为16MB。实验背景流通过web search模式产生且服从泊松分布,其中链路负载配置为90\%,突发流以4MB总大小同时从三个发送端发往同一个接收端,突发间隔为1$ms$。整个实验仿真运行100$ms$。不同链路时延下的丢包率统计结果如图\ref{c4:s1:ss1:lossless distance}所示。结果表明,相对于H-SIH,H-DSH对无损传输距离的扩展超过4倍,同时将长距离传输RTT支持范围扩展到毫秒级别。另外,在RTT增加到40ms时,H-DSH仍然可以实现接近于零的丢包率。

\begin{figure}[H]
  \centering
  \includegraphics[width=0.48\linewidth]{Figures/long-haul-transmission.pdf}
  \caption{不同链路时延下的丢包率}
  \label{c4:s1:ss1:lossless distance}
\end{figure}

\subsubsection{拥塞控制}

\begin{figure}[H]
  \begin{subfigure}[b]{0.47\linewidth}
      \centering
      \includegraphics[width=\linewidth]{Figures/fct-evaluation-cc-TcpLinuxReno.pdf}
      \subcaption{TCP Reno}
      \label{c3:s6:ss1:fig:sub1:fct tcp reno}
  \end{subfigure}
  \begin{subfigure}[b]{0.47\linewidth}
      \centering
      \includegraphics[width=\linewidth]{Figures/fct-evaluation-cc-TcpBic.pdf}
      \subcaption{TCP Bic}
      \label{c3:s6:ss1:fig:sub1:fct tcp bic}
  \end{subfigure}
  \begin{subfigure}[b]{0.47\linewidth}
    \centering
    \includegraphics[width=\linewidth]{Figures/fct-evaluation-cc-TcpCubic.pdf}
    \subcaption{TCP Cubic}
    \label{c3:s6:ss1:fig:sub1:fct tcp cubic}
  \end{subfigure}
  \begin{subfigure}[b]{0.47\linewidth}
    \centering
    \includegraphics[width=\linewidth]{Figures/fct-evaluation-cc-TcpBbr.pdf}
    \subcaption{TCP BBR}
    \label{c3:s6:ss1:fig:sub1:fct tcp bbr}
  \end{subfigure}  
  \caption{不同拥塞控制机制下的流完成时间}
  \label{c3:s6:ss1:fig:fct different cc}
\end{figure}

针对复杂网络环境可能存在的异构拥塞控制机制,本节进一步验证H-DSH在不同拥塞控制机制下的适应性。测试环境拓扑结构、网络配置和流量模式与\ref{c4:s5:ss4:fct improvement}节完全相同,测试拥塞控制算法包括TcpReno、Bic、Cubic和BBR,平均FCT结果如图\ref{c3:s6:ss1:fig:fct different cc}所示,实验结果表明,H-DSH对于不同拥塞控制机制具有良好的扩展性,和不同拥塞控制算法配合均能有效减小平均FCT,改善网络性能。


% \todo{DCTCP}

\xsection{本章小结}{Brief Summary}
\label{c4:s6:brief summary}

本章首先描述异构缓存系统现行缓存管理策略H-SIH的具体机制,经过分析和实验指出H-SIH存在的片外缓存利用低效性、PFC损害传输性能和长距离传输丢包问题。针对以上问题和RDMA跨数据中心部署趋势,提出了一种适用于异构缓存系统架构的动态共享缓存管理策略H-DSH,H-DSH以片外缓存为中心,将片外缓存作为共享缓存空间动态分配,在进行净空缓存分配时优先利用片外缓存空间,充分利用片外缓存空间。在缓存位置决策时进一步识别流量敏感性以发挥片上缓存和片外缓存各自优势。同时通过提前限制部分拥塞流量对片外缓存的占用,有效降低片外带宽瓶颈出现的概率。最后,本章设计实验验证H-DSH的性能表现,实验结果表明H-DSH在片外缓存利用效率、突发吸纳能力、PFC附带损害避免和流量传输性能等方面均有更好的性能表现,同时验证H-DSH在长距离传输、不同流量模式和异构拥塞控制的复杂网络场景中具有良好的适应性。

\clearpage
% !TeX root = ../main.tex

\xchapter{总结与展望}{Conclusions and Future Work}

\xsection{论文工作总结}{Conclusions}

本文研究无损网络中的净空缓存管理策略,针对无损网络转发设备的不同缓存架构,分别展开片上缓存系统和异构缓存系统的净空缓存管理机制的研究和实现。针对SIH存在的净空缓存挤压共享缓存空间和PFC频繁触发问题,本文设计了面向片上缓存系统的动态共享净空缓存管理机制DSH;针对H-SIH存在的突发吸纳受限于片外带宽、全局流控损害吞吐和长距离传输丢包问题,本文设计了面向异构缓存系统的动态共享净空缓存管理机制。仿真实验结果表明,相对于SIH,DSH可以显著改善流量传输性能,同时有效减少PFC触发,从而避免PFC的网络性能损害;相对于H-SIH,H-DSH可以显著改善流量传输性能,保证流量线速吞吐,同时进一步扩展无损传输距离。

本文的主要工作总结为以下三个方面:

1)详细研究了片上缓存系统和异构缓存系统架构和物理特性,着重分析了无损网络现有的净空缓存管理机制的工作机制及其在数据中心网络带宽增加和业务服务需求提高趋势下的性能问题。详细调研了国内外学术界和工业界对于无损传输相关问题的研究现状,将现有解决方案分为端到端拥塞控制方案、缓存管理方案和PFC损害消除方案,并分析了各方案的局限性。

2)针对片上缓存系统现有净空缓存管理机制SIH存在的净空缓存挤压共享缓存空间和PFC频繁触发问题,本文提出了面向片上缓存系统的动态共享净空缓存管理机制DSH。其中,动态性体现在DSH根据队列拥塞状态动态分配净空缓存;共享性体现在保底净空缓存在端口内不同队列之间的共享和共享净空缓存在所有队列之间的共享。为了验证DSH的性能,本文通过ns-3搭建片上缓存系统并实现DSH缓存管理模块。实验结果表明,在大规模网络中,DSH最多可以将长流的流完成时间降低31.1\%,短流的流完成时间降低57.7\%,同时DSH可以有效减少PFC触发,避免PFC造成网络性能损害。

3)针对异构缓存系统现有净空缓存管理机制H-SIH存在的突发吸纳受限于片外带宽、全局流控损害吞吐和长距离传输丢包问题,本文提出了面向异构缓存系统的净空缓存管理机制H-DSH。在缓存分配上,H-DSH以片外缓存为中心,将片外缓存作为共享缓存空间,扩展共享缓存容量,同时优先从片外缓存动态分配共享净空缓存,从而缓解片上净空缓存分配压力。在缓存决策上,H-DSH结合流量敏感性特征进行缓存位置决策,充分发挥片上和片外缓存的带宽和容量优势。在流量控制上,H-DSH主动提前触发部分拥塞流量的暂停帧,避免片外带宽瓶颈导致性能受损。本文通过ns-3实现异构缓存系统并实现H-DSH缓存管理模块。实验结果表明,H-DSH可以将片外缓存的突发吸纳量增大3倍,避免流量吞吐受损,最高可以将平均FCT减小14.8\%,同时对无损传输距离提升超过4倍。

\xsection{未来工作展望}{Future Work}

本文针对无损网络净空缓存管理效率和性能问题进行研究,设计并实现了面向片上缓存系统的动态共享净空缓存管理机制DSH和面向异构缓存系统的动态共享净空缓存管理机制H-DSH。由于时间和条件限制,本文的研究工作仍然存在改进空间,未来可以从以下方面进一步完善:

1)本文提出的DSH机制通过引入端口级别流量控制减少净空缓存静态预留量,整端口的暂停可能损害队列之间的隔离性,后续工作需要进一步通过理论分析和实验测试验证DSH下的队列隔离性。

2)本文提出的H-DSH可以将无损传输支持的RTT范围扩展到毫秒级别,但是随着跨数据中心传输距离进一步扩展和分布式应用网络服务需求不断增加,未来长距离链路时延可能增加至几十甚至几百毫秒,H-DSH的净空缓存分配机制需要进一步优化。

3)准确的流量识别可以进一步逼近异构缓存系统缓存管理性能边界,未来可以进一步开展流量识别方面的研究,结合片上缓存和片外缓存物理特性进行流量识别,为缓存决策提供有效信息。

4)本文通过ns-3网络仿真平台实现片上缓存系统和异构缓存系统以缓存管理模块,通过仿真实验验证了DSH和H-DSH的性能优势,但是没有在真实硬件设备中进行机制实现和实验测试,未来可以进一步在真实网络环境中验证机制性能。
% \include{Main_Spine/c6}
\thesisbodyend

% 致谢 修改 Main_Miscellaneous/acknowledegment.tex 中的内容
% Acknowledgement, Rewrite your content in Main_Miscellaneous/acknowledegment.tex
\thesisacknowledegment

% [自动生成] 参考文献
%    默认使用 References/reference.bib,手动指定请在导言区 \addbibresource 处指定数据库。
%    任意可选参数会在参考文献后生成一个非常大的页面显示所有参考文献条目,可以直接复制粘贴到学位论文提交系统中
% [Auto Generate] Bibliography, Default file is References/reference.bib, change argument in \addbibresource for manual specification
\thesisbibliography
% \thesisbibliography[onepage]

% 附录(有几个附录就导入几个文件(不加.tex后缀)),
% Appendi(x/ies), argument should not have the .tex suffix
% \thesisappendix{Main_Miscellaneous/appendix_a,
%                 Main_Miscellaneous/appendix_b}

% 攻读学位期间取得的研究成果
% 添加 [auto] 参数则读取通过 \addachivementresource 添加的数据库,否则读取 Main_Miscellaneous/achievement.tex 的内容 注意盲审时需要手动修改格式
% 自动读取的数据库中 若条目含有 AUTHOR+an = {X=highlight} 则第 X 位作者会被加粗
% Achievement, argument [auto] will load data from database added by \addachivementresource, or the data from Main_Miscellaneous/achievement.tex
\thesisachivements
% \thesisachivements[auto]

% 答辩委员会决议 修改 Main_Miscellaneous/decision.tex 中的内容
% Decision, Rewrite your content in Main_Miscellaneous/decision.tex
\thesisdecision

% [自动生成] 常规评阅人名单 需要手动指定两个数字作为'本学位论文共接受 {#1} 位专家评阅,其中常规评阅人 {2}名'内容参数
% [Auto Generate] Reviewer List, set two number as the content in this page 
\thesisreviewers{7}{5}

% [自动生成] 独创性声明
% [Auto Generate] Originality Declaration
\thesisdeclarations

\end{document}