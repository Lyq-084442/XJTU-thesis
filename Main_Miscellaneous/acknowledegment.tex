% !TeX root = ../main.tex

研究生三年转瞬即逝,本文的完成也标志着我的研究生生涯的结束。在这三年年的时间里,我的理论知识、专业技能和实践能力都得到了很大的提升,同时对自己现阶段的不足和未来规划也有了更清晰的认知。在这里,我真诚地感谢母校的栽培,感谢研究生期间所有老师、同学和朋友给与我的关心与帮助。

从本课题开题到现在论文即将完结,这个过程中有遇到困难时的沮丧,也有取得进展时的欣喜,能够克服所有难题,顺利完成毕业设计,首先要感谢我的指导老师单丹枫老师,单老师渊博的知识、严谨的态度和强大的人格魅力令我钦佩和敬仰。感谢单老师耐心地为我们讲解遇到的问题,给我们传授科研经验,时刻关注我们的科研进展,同时在生活中对我们的关爱和照顾。本论文的完成离不开单老师的帮助,老师教会了我们很多东西, 由衷地感谢单老师的付出。

同时我要感谢小组内的吴晨、王宙、郑鼎方、胡世豪、蒋林冰等师兄师姐,他们给我传授了很多科研上的经验,为我解答了很多遇到的问题,实验室的梁泽宇、孙寅鑫、党一涵、周高升、孙斌等同学和彭光宇、杨舜磊、张琦、李云广等师弟师妹,还有我的舍友金吉磊、王冠一和党博文,在这个过程中和我一起讨论,帮我解决了很多问题,互相分享喜怒哀乐,彼此成长了很多。

还要感谢我的家人,感谢你们给我的关心和支持,你们对我的关爱是我保持进取的不竭动力,你们的健康和快乐是我最大的心愿。

最后, 对评审本文的专家和教授表示由衷的感谢。